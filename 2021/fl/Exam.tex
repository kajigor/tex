\documentclass[12pt]{article}
\usepackage[left=2cm,right=2cm,top=1cm,bottom=1cm,bindingoffset=0cm]{geometry}
\usepackage[utf8x]{inputenc}
\usepackage[english,russian]{babel}
\usepackage{cmap}
\usepackage{amssymb}
\usepackage{amsmath}
\usepackage{pifont}
\usepackage{tikz}
\usepackage{verbatim}
\usepackage{enumitem}
\usepackage{hyperref}

\pagenumbering{gobble}

\begin{document}

\begin{center}
  \LARGE Формальные языки

  \Large вопросы к экзамену
\end{center}

\begin{enumerate}
\item {Контекст, в котором возникают формальные языки. Метаязык описания языков. Алфавит, цепочка (строка), операции над цепочками (конкатенация, обращение, степень, длина), их свойства.}
  \item {Форма Бэкуса-Наура, расширенная форма Бэкуса-Наура, синтаксические диаграммы Вирта как примеры метаязыков. Примеры описания языков с использованием метаязыков. }
  \item {Формальный язык, формальная грамматика (по Хомскому), примеры. Непосрественная выводимость, выводимость, порождаемый грамматикой язык. Эквивалентность грамматик. }
  \item {Контекстно-свободные грамматики и деревья вывода, примеры. Теорема о соотношении вывода и дерева вывода, доказательство.}
  \item {Конечные автоматы, полные конечные автоматы, путь в конечном автомате, такт работы КА, распознавание слова КА, язык, распознаваемый КА, примеры. }
  \item {Эквивалентность КА, проверка на эквивалентность. Минимальность КА, алгоритм минимизации путем выделения классов эквивалентности, корректность алгоритма, сложность, пример использования. }
  \item {Недетерминированные КА, их соотношение с детерминированными. Распознавание слова НКА, алгоритм, проверяющий допустимость слова НКА. Детерминизация: алгоритм Томпсона. Эквивалентность ДКА и НКА.}
  \item {Произведение автоматов, пример. Нахождение пересечения, объединения, разности регулярных языков при помощи произведения автоматов. Замкнутость автоматных языков относительно теоретико-множественных операций. }
  \item {Регулярные множества (языки), примеры регулярных языков, академические регулярные выражения, примеры. Замкнутость регулярных языков относительно различных операций. Свойства регулярных выражений. }
  \item {Теорема Клини. НКА с $\varepsilon$-переходами, эквивалентность НКА без $\varepsilon$-переходов, $\varepsilon$-замыкание. Доказательство теоремы Клини (в обе стороны). Примеры построения НКА по регулярному выражению и регулярного выражения по НКА.}
  \item {Праволинейные/леволинейные грамматики, регулярные грамматики, эквивалентность регулярных грамматик и НКА. Лемма о накачке для регулярных языков, доказательство, применение. }
  \item {Контекстно-свободные грамматики. Вывод в КС-грамматике, пример. Теорема о существовании левостороннего вывода. Однозначность и неоднозначность грамматик. Неразрешимость проверки однозначности грамматики. }
  \item {Контекстно-свободные языки, существенная неоднозначность. Проверка пустоты порождаемого языка, доказательство. Удаление непродуктивных нетерминалов грамматики, приведение грамматики, удаление цепных правил. }
  \item {Нормальная форма Хомского, алгоритм приведения к НФХ, пример, важность порядка операций при приведении к НФХ, разрастание грамматики при нормализации. CYK-алгоритм, пример, сложность работы. }
  \item {Восходящий и нисходящий синтаксический анализ. Функции FIRST, FOLLOW. LL-грамматики, Фундаментальное свойство LL-грамматик. Пример LL(k) грамматики, простая LL(1) грамматика. LL(k)-грамматика: необходимое и достаточное условие. LL(1)-грамматика: необходимое и достаточное условие. LL-грамматики и левая рекурсия.}
  \item {Типы нисходящих синтаксических анализаторов. нисходящий синтаксический анализ с откатом, пример, сложность. Нисходящий синтаксический анализ без отката. Рекурсивный спуск, пример. }
  \item {LL(k) анализаторы. Избавление от левой рекурсии (явной, неявной, взаимной) в грамматиках. Левая факторизация грамматики, пример. Вычисление множеств FIRST и FOLLOW с примерами. LL(1) анализ, построение таблиц анализатора, сложность, пример, ограничения LL-анализаторов. }
  \item {Восходящий синтаксический анализ. Алгоритмы LR(0), SLR(1), CLR(1), построение таблиц, принцип работы, различия, сложность работы, ограничения алгоритмов, примеры.}
  \item {Магазинный автомат: неформальное понимание. Детерминированные и недетерминированные магазинные автоматы. Отношение переходов, семантика магазинного автомата, 2 варианта принятия слова: по достижении конечного состояния, по опустошению стека. Пример. Построение МА по КС грамматике. Лемма о накачке для КС языка, пример использования. }
\end{enumerate}


\newpage

\begin{center}
{\LARGE Расстрельный список определений}

{\emph{ Знание всех определений --- необходимое условие положительной оценки за экзамен }}
\end{center}

\begin{enumerate}
  \item {Множество, подмножество, множество всех подмножеств, операции над множествами.}
  \item {Алфавит, цепочка (строка), формальный язык}
  \item {Формальная грамматика, порождаемый грамматикой язык.}
  \item {Контекстно-свободные грамматики и деревья вывода.}
  \item {Конечные автоматы; язык, распознаваемый КА. }
  \item {Регулярные множества (языки). }
  \item {Теорема Клини.}
  \item {Регулярные грамматики.}
  \item {Контекстно-свободные грамматики, вывод в КС-грамматике.}
  \item {Контекстно-свободные языки.}
  \item {Нормальная форма Хомского.}
  \item {Магазинный автомат.}
\end{enumerate}

\end{document}

