\documentclass{beamer}
\usepackage{beamerthemesplit}
\usepackage{wrapfig}
\usetheme{SPbGU}
\usepackage{pdfpages}
\usepackage{amsmath}
\usepackage{mathtools}
\usepackage{cmap}
\usepackage[T2A]{fontenc}
\usepackage[utf8]{inputenc}
\usepackage[english,russian]{babel}
\usepackage{indentfirst}
\usepackage{amsmath}
\usepackage{tikz}
\usepackage{multirow}
\usepackage[noend]{algpseudocode}
\usepackage{algorithm}
\usepackage{algorithmicx}
\usepackage{ stmaryrd }
\usepackage{qtree}
\usetikzlibrary{shapes,arrows}
\usepackage{fancyvrb}

\usetikzlibrary{positioning}
\usetikzlibrary{decorations.pathmorphing}

\usetikzlibrary{shapes,arrows}
\usetikzlibrary{positioning,automata}
\tikzset{
  snake it/.style={decorate, decoration=snake},
  every state/.style={minimum size=0.2cm},
  initial text={}
}

\newtheorem{rutheorem}{Теорема}
\newtheorem{ruproof}{Доказательство}
\newtheorem{rudefinition}{Определение}
\newtheorem{rulemma}{Лемма}
\beamertemplatenavigationsymbolsempty

\newcommand{\derive}[0]{\xRightarrow[]{*}}
\newcommand{\derives}[0]{\xRightarrow[]{}}
\newcommand{\derivek}[1]{\xRightarrow[]{#1}}
\newcommand{\deriveg}[1]{\xRightarrow[#1]{*}}
\newcommand{\derivegone}[1]{\xRightarrow[#1]{}}

\setbeamertemplate{itemize item}[circle]
\setbeamertemplate{enumerate item}[circle]

  % \newcommand{\derives}[1][*]{\xRightarrow[]{#1}}
  % \newcommand{\deriveg}[1]{\xRightarrow[#1]{*}}
  % \newcommand{\derivegone}[1]{\xRightarrow[#1]{}}

\title[]{Теория автоматов и формальных языков}
\subtitle[]{Контекстно-свободные языки: нисходящий анализ}
\institute[]{
Санкт-Петербургский государственный электротехнический университет <<ЛЭТИ>>\\
}

\author[]{Екатерина Вербицкая}

\date{16 ноября 2021}

\definecolor{orange}{RGB}{179,36,31}

\begin{document}
{
  \begin{frame}
    \titlepage
  \end{frame}
}


\begin{frame}[fragile]
  \transwipe[direction=90]
  \frametitle{В предыдущей серии}
  \begin{itemize}
    \item Контекстно-свободные грамматики (все правила вида $A \to \alpha$)
    \item КС языки и разрешимость проверки пустоты
    \item Нормальная форма Хомского
    \item Алгоритм CYK
  \end{itemize}
\end{frame}

\begin{frame}[fragile]
  \transwipe[direction=90]
  \frametitle{В предыдущей серии: НФХ}
  КС грамматика находится в \textbf{нормальной форме Хомского}, если все ее правила имеют вид:
  \begin{itemize}
    \item $A \to B C$, где $A,B,C \in V_N$
    \item $A \to a$, где $A \in V_N, a \in V_T$
    \item $S \to \varepsilon$, если в языке есть пустое слово, где $S$ --- стартовый нетерминал
  \end{itemize}

  \begin{enumerate}
    \item Удалить стартовый нетерминал из правых частей правил
    \item Избавиться от неодиночных терминалов в правых частях
    \item Удалить длинные правила (длины больше 2)
    \item Удалить непродуктивные правила ($\varepsilon$-правила)
    \item Удалить цепные правила
   \end{enumerate}
\end{frame}


\begin{frame}[fragile]
  \transwipe[direction=90]
  \frametitle{В предыдущей серии: CYK}
  \begin{itemize}
      \item Алгоритм синтаксического анализа, работающий с грамматиками в НФХ
      \item Динамическое программирование
  \end{itemize}

  \begin{figure}
    \begin{tikzpicture}
    \draw (0,2.8) -- (-0.5, 1.7);
    \draw (0,2.8) -- (0.5, 1.7);
    \draw (-0.5, 1.3) -- (-2, 0.05);
    \draw (0.5, 1.3) -- (2, 0.05);
    \draw (-0.5, 1.3) -- (-0.2, 0.05);
    \draw (0.5, 1.3) -- (0.2, 0.05);
    \draw (0,2.8) -- (-2, 0.05);
    \draw (0,2.8) -- (2, 0.05);
    \node[draw=none,fill=none] (A) at (0, 3) {A};
    \node[draw=none,fill=none] (B) at (-0.5, 1.5) {B};
    \node[draw=none,fill=none] (C) at (0.5, 1.5) {C};
    \node[draw=none,fill=none] (w) at (-3.5, -0.2) {$\omega$:};
    \node[draw=none,fill=none] (o) at (-2.9, -0.6) {};
    \node[draw=none,fill=none, right=0.6 of o]  {i};
    \node[draw=none,fill=none, right=2.35 of o]  {k};
    \node[draw=none,fill=none] at (0.5, -0.615)  {k+1};
    \node[draw=none,fill=none, right=4.6 of o] {j};

    \draw[draw=black] (-3,-0.4) rectangle ++(6,0.4);
    \draw[draw=black] (-2.2,-0.4) rectangle ++(0.4,0.4);
    \draw[draw=black] (-0.4,-0.4) rectangle ++(0.4,0.4);
    \draw[draw=black] (0,-0.4) rectangle ++(0.4,0.4);
    \draw[draw=black] (1.8,-0.4) rectangle ++(0.4,0.4);

    \end{tikzpicture}
  \end{figure}
\end{frame}


\begin{frame}[fragile]
  \transwipe[direction=90]
  \frametitle{CYK}
  \begin{itemize}
    \item Дано: строка $\omega$ длины $n$, грамматика $G = \langle V_T, V_N, P, S\rangle$ в НФХ
    \item Используем трехмерный массив d булевых значений размером $|V_N| \times n \times n$, $d[A][i][j] = true \Leftrightarrow A \derives \omega[i \dots j]$
    \item Инициализация: $i = j$
    \begin{itemize}
      \item $d[A][i][i] = true$, если в грамматике есть правило $A \to \omega[i]$
      \item $d[A][i][i] = false$, иначе
    \end{itemize}
    \item Динамика. Предполагаем, d построен для всех нетерминалов и пар $\{(i', j') \mid j' - i' < m \}$
    \begin{itemize}
      \item $d[A][i][j] = \bigvee_{A\to BC}^{}{\bigvee_{k=i}^{j}{d[B][i][k] \wedge d[C][k+1][j]}}$
    \end{itemize}
    \item В конце работы алгоритма в $d[S][1][n]$ записан ответ, выводится ли $\omega$ в данной грамматике
\end{itemize}

\end{frame}

\begin{frame}[fragile]
  \transwipe[direction=90]
  \frametitle{CYK --- алгоритм восходящего анализа}

\begin{center}
    Восходящий анализ: начинаем с символов входной строки, строим дерево вывода до стартового нетерминала
\end{center}

\vfill

\begin{tabular}{p{2.5cm} p{1cm} p{1cm} | p{6cm}}

\Tree [.E [.E T ] + T ]
&
\begin{center}
+
\end{center}
&
\begin{center}
T
\end{center}
&

\Tree [.E [.E [.E T ] + T ] + T ]

\end{tabular}

\vfill


\begin{center}
    Восходящий анализ контринтуинтивен

    (особенно при диагностике ошибок)
\end{center}
\end{frame}

\begin{frame}[fragile]
  \transwipe[direction=90]
  \frametitle{Нисходящий синтаксический анализ}
  \begin{itemize}
    \item Top-down parsing
    \item Начинаем разбирать со стартового нетерминала, применяем правила грамматики, пока не получим строку
    \begin{itemize}
      \item С откатом ([full] backtracking)
      \item Без отката (without backtracking)
    \end{itemize}
  \end{itemize}
\end{frame}

\begin{frame}[fragile]
  \transwipe[direction=90]
  \frametitle{Нисходящий синтаксический анализ с откатом}
  \begin{itemize}
    \item Метод грубой силы, bruteforce
    \item Перебираем все возможные варианты разбора, если что-то пошло не так --- возвращаемся к началу и пробуем снова
  \end{itemize}
\end{frame}

\begin{frame}[fragile]
  \transwipe[direction=90]
  \frametitle{Bruteforce parsing: пример}
  \[
  \begin{array}{crcccl}
  &S& \to & a A d & | & a B \\
  &A& \to & b     & | & c \\
  &B& \to & c c d & | & d d c
  \end{array}
  \]

  \[\omega = a d d c\] \pause

  $S \pause \derives a A d \pause \derives a b d$ \pause \hfill не подходит, откатываемся \pause

  $S \derives a A d \pause \derives a c d$ \pause \hfill не подходит, откатываемся \pause

  $S \pause \derives a B \pause \derives a c c d$ \pause \hfill не подходит, откатываемся \pause

  $S \derives a B \pause \derives a d d c$ \pause \hfill ура!

  \vfill


  \begin{center}
    Проблема: ну очень уж долго работает: экспоненциальное время!
  \end{center}
\end{frame}

\begin{frame}[fragile]
  \transwipe[direction=90]
  \frametitle{Нисходящий синтаксический анализ без отката}
  \begin{itemize}
    \item Рекурсивный спуск (recursive descent parsing)
    \begin{itemize}
      \item Для каждого нетерминала написана функция
      \item Функции для нетерминалов рекурсивно вызывают друг друга
    \end{itemize}
  \end{itemize}

  \bigskip

  \begin{verbatim}
  parse_S(word) =
    if (isEmpty(word))
    then (true, word)
    else
      let (r, w') = parse_lbr(word) in
      if (r)
      then
        let (r, w'') = parse_S(w') in
        if (r)
        then parser_rbr(w'')
        else (false, w')
      else (false, word)

  \end{verbatim}
\end{frame}


\begin{frame}[fragile]
  \transwipe[direction=90]
  \frametitle{Нисходящий синтаксический анализ без отката: LL(1)}
  \begin{itemize}
    \item Идея: откат запрещен, но разрешен предпросмотр
    \item По следующему терминалу принять решение о том, какую продукцию использовать
    \item Как и предыдущие 2 подхода не может обрабатывать леворекурсивные правила грамматики
    \item Достаточно хорош для используемых на практике языков
  \end{itemize}
\end{frame}


\begin{frame}[fragile]
  \transwipe[direction=90]
  \frametitle{LL(1)-анализ}
  \begin{itemize}
   \item Нисходящий синтаксический анализ с предпросмотром одного символа
   \item Читает вход слева направо (L: left-to-right), строит левый вывод в грамматике (L: leftmost)
   \item Состоит из:
   \begin{itemize}
     \item Входного буфера (откуда читается входная строка)
     \item Стека (для промежуточных данных)
     \item \textbf{Таблицы} анализатора (управляет процессом разбора)
   \end{itemize}
   \item Работает за $O(n)$, где $n$ --- длина входной строки
  \end{itemize}
\end{frame}

\begin{frame}[fragile]
  \transwipe[direction=90]
  \frametitle{Таблица LL(1)-анализатора}

  \begin{center}
    Управляет процессом разбора: показывает, какую продукцию применять, если во время анализа рассматривается нетерминал $A$, а~следующий символ входа --- $t$
  \end{center}



\vfill

\begin{center}
\begin{tabular}{ r || c | c | c | c  }
      & $\dots$ & $t$          & $\dots$ & $\$ $ \\ \hline
      & $\dots$ & $\dots$      & $\dots$ & $\dots$ \\ \hline
  $A$ & $\dots$ & $A\to\alpha$ & $\dots$ & $\dots$ \\ \hline
      & $\dots$ & $\dots$      & $\dots$ & $\dots$
\end{tabular}
\end{center}

\vfill

\begin{center}
  Для заполнения таблицы надо научиться считать множества символов, которые можно встретить во время анализа
\end{center}
\end{frame}

\begin{frame}[fragile]
  \transwipe[direction=90]
  \frametitle{Множество FIRST}
\begin{center}
  Множество символов, которые могут появиться первыми во время вывода из данной сентенциальной формы
\end{center}

  \begin{itemize}
   \item $FIRST(a \alpha) = \{ a \}, $ если $\, a \in V_T, \alpha \in (V_T \cup V_N)^*$
   \item $FIRST(\varepsilon) = \{ \varepsilon \}$
   \item $FIRST(\alpha \beta) = FIRST(\alpha) \cup (FIRST(\beta), \text{ если } \varepsilon \in FIRST(\alpha))$
   \item $FIRST(S) = FIRST(\alpha) \cup FIRST(\beta),$ если есть правило $S \to \alpha \mid \beta $
  \end{itemize}
\end{frame}

\begin{frame}[fragile]
  \transwipe[direction=90]
  \frametitle{Множество FIRST: пример}
  \[
  \begin{array}{crcl}
  &S  & \to & a S' \\

  &S' & \to & A b B S' \mid \varepsilon \\

  &A  & \to & a A' \mid \varepsilon \\
  &A' & \to & b \mid a \\
  &B  & \to & c \mid \varepsilon
  \end{array}
  \] \pause

  \begin{itemize}
    \item $FIRST(S) = \{ a \}$ \pause
    \item $FIRST(A) = \{ a, \varepsilon \}$ \pause
    \item $FIRST(A') = \{ a, b \}$ \pause
    \item $FIRST(B) = \{ c, \varepsilon \}$ \pause
    \item $FIRST(S') = \{ a, b, \varepsilon \}$
  \end{itemize}
\end{frame}

\begin{frame}[fragile]
  \transwipe[direction=90]
  \frametitle{Множество FOLLOW}
\begin{center}
  Множество символов, которые могут появиться в некотором выводе сразу после данной сентенциальной формы
\end{center}

  \begin{itemize}

   \item Положим $FOLLOW(X) = \varnothing $
   \item Если $X$ --- стартовый нетерминал, $FOLLOW(X) = FOLLOW(X) \cup \{ \$ \}$ --- символ конца строки
   \item Для всех правил вида $A \to \alpha X \beta$, $FOLLOW(X) = FOLLOW(X) \cup (FIRST(\beta) \setminus \{ \varepsilon\})$
   \item Для всех правил вида $A \to \alpha X$ и $A \to \alpha X \beta$, где $\varepsilon \in FIRST(\beta)$, $FOLLOW(X) = FOLLOW(X) \cup FOLLOW(A)$
   \item Повторять последние 2 пункта, пока можно что-то добавлять
  \end{itemize}
\end{frame}

\begin{frame}[fragile]
  \transwipe[direction=90]
  \frametitle{Множество FOLLOW: пример}
  \[
  \begin{array}{crcl}
  &S  & \to & a S' \\

  &S' & \to & A b B S' \mid \varepsilon \\

  &A  & \to & a A' \mid \varepsilon \\
  &A' & \to & b \mid a \\
  &B  & \to & c \mid \varepsilon
  \end{array}
  \] \pause

  \begin{itemize}
    \item $FOLLOW(S) = \{ \$ \}$ \pause
    \item $FOLLOW(S') = \{ \$ \}$ \hfill $(S \to a S')$  \pause
    \item $FOLLOW(A) = \{ b \}$ \hfill $(S' \to A b B S')$ \pause
    \item $FOLLOW(A') = \{ b \}$ \hfill $(A \to a A')$ \pause
    \item $FOLLOW(B) = \{ a, b, \$ \}$ \hfill $(S' \to A b B S', \varepsilon \in FIRST(S'))$
  \end{itemize}
\end{frame}

\begin{frame}[fragile]
  \transwipe[direction=90]
  \frametitle{Таблица LL(1)-анализатора}
  \[
  S \to ( S ) \mid \varepsilon
  \]

\vfill

  \begin{center}
    Размещаем продукции в таблице

    (по горизонтали --- нетерминалы; по вертикали --- терминалы + $\$ $)
  \end{center}

  \begin{itemize}
    \item Продукции вида $A \to \alpha$ --- в ячейки $(A, a)$, где $a \in FIRST(\alpha), a \neq \varepsilon$
    \item Продукции вида $A \to \alpha$ --- в ячейки $(A, a)$, где $a \in FOLLOW(A)$, если $\varepsilon \in FIRST(\alpha)$
  \end{itemize}

  \vfill

\begin{center}
\begin{tabular}{ l || c | c || c | c | r }
  N & FIRST & FOLLOW & ( & ) & $\$ $ \\ \hline
  $S$ & \pause $\{ (, \varepsilon \}$ & $\{ ), \$ \}$ & \pause $S \to (S)$ & \pause $S \to \varepsilon$ & $S \to \varepsilon$
\end{tabular}
\end{center}
\end{frame}

\begin{frame}[fragile]
  \transwipe[direction=90]
  \frametitle{LL(1)-анализ}
\begin{itemize}
  \item Инициализация: указатель в строке на первый символ, в стек помещаем \$ и стартовый нетерминал
  \item Пока стек не пуст
  \begin{itemize}
    \item Если на вершине стека нетерминал $N$, указатель в строке на символе $t$, смотрим на содержимое ячейки $(N,t)$ управляющей таблицы
    \begin{itemize}
      \item Если ячейка пуста, сообщаем об ошибке анализа
      \item Если в ячейке продукция $N \to \varepsilon$, cнимаем со стека $N$
      \item Если в ячейке продукция $N \to \alpha$, снимаем со стека $N$, символы $\alpha$ кладем на стек в обратном порядке
    \end{itemize}
    \item Если на вершине стека терминал $t$
    \begin{itemize}
      \item Если указатель в строке на терминале $t$, снимаем со стека вершину, двигаем указатель на следующий символ
      \item Если указатель в строке на любом другом терминале, сообщаем об ошибке
    \end{itemize}
  \end{itemize}

  \item Если строка прочитана полностью, анализ завершен успешно.
  Иначе — полагается сообщить об ошибке
\end{itemize}
\end{frame}

\begin{frame}[fragile]
  \transwipe[direction=90]
  \frametitle{Пример (доска)}
  \[
  S \to ( S ) \mid \varepsilon
  \]

\begin{center}
\begin{tabular}{ l || c | c || c | c | r }
  N & FIRST & FOLLOW & ( & ) & $\$ $ \\ \hline
  $S$ & $\{ (, \varepsilon \}$ & $\{ ), \$ \}$ & $S \to (S)$ & $S \to \varepsilon$ & $S \to \varepsilon$
\end{tabular}
\end{center}

$\omega = (()) \$ $

Стек: $\$, S, ), S, (, ), S, ($
\end{frame}

\begin{frame}[fragile]
  \transwipe[direction=90]
  \frametitle{Когда LL-анализ не возможен}
\begin{itemize}
  \item Леворекурсивные правила
  \item Когда при построении таблицы в одну ячейку нужно записать
  больше одной записи
  \begin{itemize}
    \item FIRST-FIRST конфликт
    \begin{itemize}
      \item $A \to \alpha \mid \beta, FIRST(\alpha) \cap FIRST(\beta) \neq \varnothing$
      \item $E \to T + E \mid T * E$
    \end{itemize}
    \item FIRST-FOLLOW конфликт
    \begin{itemize}
      \item $FIRST(A) \cap FOLLOW (A) \neq \varnothing$
      \item $S \to Aab, A \to a \mid \varepsilon$
    \end{itemize}
  \end{itemize}
  \item Как с этим бороться?
  \begin{itemize}
    \item Избавиться от левой рекурсии
    \item Избавиться от недетерминизма
    \item Факторизовать грамматику
    \item Использовать аннотации (если есть)
    \item Переписать грамматику
    \item Использовать более одного символа предпросмотра
  \end{itemize}
\end{itemize}
\end{frame}
\end{document}



