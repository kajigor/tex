\documentclass[12pt,a4paper]{article}%
\usepackage{amsthm}
\usepackage{amsmath}%
\usepackage{amsfonts}%
\usepackage{amssymb}%
\usepackage{graphicx}
\usepackage[T2A]{fontenc}
\usepackage[utf8]{inputenc}
\usepackage[english,russian]{babel}
%-------------------------------------------
\setlength{\textwidth}{7.0in}
\setlength{\oddsidemargin}{-0.35in}
\setlength{\topmargin}{-0.5in}
\setlength{\textheight}{9.0in}
\setlength{\parindent}{0.3in}

\newtheorem{theorem}{Theorem}
\newtheorem{task}[theorem]{Задача}
\addto\captionsrussian{\renewcommand*{\proofname}{Решение}}


\newcommand{\abovemath}[2]{\ensuremath{\stackrel{\text{#1}}{#2}}}
\newcommand{\aboveeq}[1]{\abovemath{#1}{=}}
\newcommand\bydef{\aboveeq{def}}
\begin{document}


% \begin{flushright}
% \textbf{Екатерина Вербицкая \\
% \today}
% \end{flushright}

\begin{center}
\textbf{Формальные языки\\
HW03 \\
Дедлайн: 23:59 15 ноября 2021} \\
\end{center}

\task{
    Привести однозначную контекстно-свободную грамматику для языка арифметических выражений над положительными целыми \emph{числами} с операциями \verb!+!, \verb!-!, \verb!*!, \verb!/!, \verb!^!, \verb!==!,\verb!<>!, \verb!<!, \verb!<=!, \verb!>!, \verb!>=! со следующими приоритетами и ассоциативностью:

  \begin{center}
    \begin{tabular}{ c | c }
      Наибольший приоритет & Ассоциативность  \\ \hline \hline
      \verb!^! & Правоассоциативна \\
     \verb!*!,\verb!/! & Левоассоциативна \\
     \verb!+!,\verb!-! & Левоассоциативна \\
     \verb!==!,\verb!<>!, \verb!<!,\verb!<=!, \verb!>!,\verb!>=! & Неассоциативна \\ \hline \hline
     Наименьший приоритет & Ассоциативность
    \end{tabular}
    \end{center}


    Неассоциативные операции встречаются только один раз: \verb!1 == 2! -- корректная строка, \verb!1 == 2 == 3!, \verb!(1 == 2) == 3!, \verb!1 < 2 > 3! --- некорректные строки
  }

\task {Привести грамматику из 1 задания в нормальную форму Хомского.}

\task {Промоделировать работу алгоритма CYK на грамматике из 2 задания на трех корректных строках не короче 7 символов и на трех некорректных строках. (Привести таблицы и деревья вывода)}


\bigskip

\begin{center}
  \Large{Примеры оформления грамматик и таблиц}
\end{center}

\begin{center}
  \begin{align*}
    S' &\to A B \mid \varepsilon \\
    S  &\to A B \\
    A  &\to L S \mid ( \\
    B  &\to R S \mid \ ) \\
    L  &\to ( \\
    R  &\to \ )
  \end{align*}
\end{center}

\begin{center}
  \begin{tabular}{ c || c | c | c | c | c | c | }
      &  1  &  2  &  3  &  4  &  5  &  6   \\ \hline \hline
    1 & A L &     &  A  &     &  A  & S S' \\ \hline
    2 &     & A L &  S  &     &  S  &      \\ \hline
    3 &     &     & B R &     &  B  &      \\ \hline
    4 &     &     &     & A L &  S  &      \\ \hline
    5 &     &     &     &     & B R &      \\ \hline
    6 &     &     &     &     &     & B R  \\ \hline
  \end{tabular}
  \end{center}

\end{document}
