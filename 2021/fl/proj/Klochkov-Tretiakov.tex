\documentclass[12pt]{article}
\usepackage[left=1cm,right=1.5cm,top=2cm,bottom=2cm,bindingoffset=0cm]{geometry}
\usepackage{hyperref}
\usepackage{fontspec}
\usepackage{polyglossia}
\usepackage{amssymb}
\usepackage{cprotect}
\setdefaultlanguage{russian}
\pagestyle{empty}
\newfontfamily{\cyrillicfonttt}{Liberation Mono}
\setmainfont[Mapping=tex-text]{CMU Serif}

\begin{document}
\begin{center}
{\LARGE Формальные языки}

\bigskip

{\Large Антон Клочков и Константин Третьяков}
\end{center}

\bigskip

Основная цель: оценить целесообразность инкрементального синтаксического парсинга на современных машинах.

\begin{itemize}
  \item Антон реализует обычный, неинкрементальный, парсер для pascal в рамках курса про разработку IDE.
  \item Константин реализует инкрементальный парсер для того же языка.
  \item Можно использовать любые инструменты, писать вручную --- как кажется наиболее разумным.
  \item Сравнить задержки, которые возникают при синтаксическом анализе, вызванными изменениями в коде. Оценить, стоит ли вообще использовать инкрементальный парсер.
  \item Учесть, что некоторые изменения могут быть хоть и мелкими, но сильно нелокальными, как при удалении какой-нибудь скобки или раскомментировании строки. Имеет смысл изучить, какие изменения приводят к наибольшим задержкам.
  \item Подготовить отчет.
\end{itemize}

\end{document}