\section{Specialization Efforts for \texttt{miniKanren}}

Specialization, or partial evaluation, is an optimization technique that precomputes parts of program execution based on information known about a program before execution. 
For example, consider a function \texttt{exp n x = if n == 0 then 1 else x * (exp (n - 1) x)} and imagine that we know from some context that it is always being called with the argument \texttt{n} equal to  \texttt{4}. 
In this case, we can partially evaluate the function to \texttt{exp\_4 x = x * x * x * x * 1} that is more efficient than the original function called with \texttt{n = 4}. 
Note, that a smart enough specializer can also be able to generate a function of form \texttt{exp\_4 x = let sqr = x * x in sqr * sqr} that makes even less multiplications. 

This pattern can be expressed in a way that if there is a function with some of its arguments statically known \texttt{f x$_{static}$ y$_{dynamic}$}, it can be transformed into a more efficient function \texttt{f\_x$_{static}$} with its parts dependent on the static arguments precomputed.
The resulting program must be equivalent to the original one, meaning that given the same dynamic arguments, it will return the same results: \texttt{f x$_{static}$ y$_{dynamic}$ == f\_x$_{static}$ y$_{dynamic}$}.

In the field of logic programming, specialization is generally known as partial deduction. 
Besides the values of static arguments, a partial deducer can also consider the information about a direction of a program or the interaction between logic variables in a conjunction of calls. 
In addition to specialization, a relation with a given direction can be converted into a function in which expensive logic operations are replaced with streamlined functional counterparts. 
In this project, we combine the two methods for the verifier-to-solver approach.

\subsection{Conservative Partial Deduction}


Conjunctive Partial Deduction was first developed by Michael Leuschel for \texttt{Prolog} in the ECCE system~\cite{de1999conjunctive}. 
It makes use of the interaction between conjuncts for specialization, getting rid of some repeating traversals of data structures as a result. 
We implemented this algorithm as a proof-of-concept for \texttt{miniKanren}, and found out that some of the specialization results were subpar.
In some cases, the specialized programs had worse performance than the original ones. 

Then, we formulated a different specialization method, which we called Conservative Partial Deduction. 
The difference with CPD lies in the way conjunctions are treated. 
They are split more often and thus generate smaller programs. 
The method was able to achieve a 1.5-2 times performance increase on a propositional evaluator program and almost a 40 times performance increase on a type checker. 

\subsection{Functional Conversion}

Even after the partial deduction has finished, there are some sources of inefficiency that can be addressed. 
The base of relational programming is unification, which is an expensive operation, especially if it runs occurs-check. 
When a direction of a relation is known, mode analysis~\cite{somogyi1987system,smaus2000mode} can be run to determine data flow from the statically known variables. 
As a result, most unifications can be converted into assignments, pattern matching or equality checks. 
Besides that, we can reorder the calls within a conjunction to restrict the search space early. 

In some cases, when the direction is deterministic, the relation can be transformed into a pure function with no overhead of the relational language. 
However, if the direction results in multiple possible answers, we still need to express non-determinism. 
This is possible in a functional programming language by using the \texttt{Stream} monad that is at the base of the relational programming. 
We implemented a functional conversion method that gave the propositional evaluation program a 2.5-fold increase in performance.
For some programs dealing with arithmetic, it improved performance by up to two orders of magnitude.

