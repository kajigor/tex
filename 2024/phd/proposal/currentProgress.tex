\section{Current Progress}

To date, several steps have been taken towards achiving the research objectives. 

Firstly, we formalized the verifier-to-solver approach in the paper \emph{``Relational Interpreters for Search Problems.''} by Petr Lozov, Ekaterina Verbitskaia, and Dmitry Boulytchev, published at the  Relational Programming Workshop, 2019. 
\href{https://dash.harvard.edu/bitstream/handle/1/41307116/tr-02-19.pdf?sequence=1&isAllowed=y#page=47}{The full text is available here}. 
This paper also highlights the pitfalls of the approach and motivates the need for program specialization. 

Secondly, we noticed that Conjunctive Partial Deduction generates vastly different results depending on the order of the conjuncts within a definition. 
In an effort to combat this drawback, we proposed a novel partial evaluation for \mk in the paper \emph{``An Empirical Study of Partial Deduction for \textsc{miniKanren}.''} by Ekaterina Verbitskaia, Daniil Berezun, and Dmitry Boulytchev, published at  Verification and Program Transformation Workshop, 2021.
\href{https://cgi.cse.unsw.edu.au/~eptcs/paper.cgi?VPT2021.5.pdf}{The full text is available here}. 
The difference with CPD lies in the way conjunctions are treated. 
They are split more often and thus generate smaller programs. 
The method was able to achieve a 1.5-2 times performance increase on a propositional evaluator program and almost a 40 times performance increase on a type checker. 

Finally, we described functional conversion that makes it possible to generate a functional implementation of a solver from a functional implementation of a verifier in the paper  \emph{``A Case Study in Functional Conversion and Mode Inference in \textsc{miniKanren}.''} by Ekaterina Verbitskaia, Igor Engel, and Daniil Berezun, published at  the Partial Evaluation and Program Manipulation Workshop, 2024. 
\href{https://web.archive.org/web/20240111175807id_/https://dl.acm.org/doi/pdf/10.1145/3635800.3636966}{The full text is available here}.
We were able to get rid of some innefficiences innate to the relational programming itself. 
Among these are expensive unification with occurs-check that can often be replaced with much cheaper functional couterparts such as pattern matching and syntactic equality check based on the program's mode. 
To achived this goal, we adapted mode analysis~\cite{somogyi1987system,smaus2000mode} to \mk. 
This analisis determines data flow based on what variables are known to be ground at runtime and also reorders conjuncts based on a heuristic aimed at reducing the search space.
The implemented functional conversion gave the propositional evaluation program a 2.5-fold increase in performance.
For some programs dealing with arithmetic, it improved performance by up to two orders of magnitude.



Significant progress has already been made in all the main declared areas, so to complete the work, it remains to collect all the results and describe their joint contribution to bringing the verifier-to-solver approach to life. 


