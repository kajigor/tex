% This is samplepaper.tex, a sample chapter demonstrating the
% LLNCS macro package for Springer Computer Science proceedings;
% Version 2.21 of 2022/01/12
%
\documentclass[runningheads]{llncs}
\usepackage{hyperref}
\usepackage{geometry}
\geometry{
  a4paper,         % or letterpaper
  textwidth=15cm,  % llncs has 12.2cm
  textheight=24cm, % llncs has 19.3cm
  heightrounded,   % integer number of lines
  hratio=1:1,      % horizontally centered
  vratio=2:3,      % not vertically centered
}

%
\usepackage[T1]{fontenc}
% T1 fonts will be used to generate the final print and online PDFs,
% so please use T1 fonts in your manuscript whenever possible.
% Other font encondings may result in incorrect characters.
%
\usepackage{graphicx}
% Used for displaying a sample figure. If possible, figure files should
% be included in EPS format.
%
% If you use the hyperref package, please uncomment the following two lines
% to display URLs in blue roman font according to Springer's eBook style:
%\usepackage{color}
%\renewcommand\UrlFont{\color{blue}\rmfamily}
%\urlstyle{rm}
%
\begin{document}
%
\title{Towards Efficient Search: Leveraging Relational Interpreters and Partial Deduction Techniques}

% \subtitle{PhD Thesis Proposal}
%
%\titlerunning{Abbreviated paper title}
% If the paper title is too long for the running head, you can set
% an abbreviated paper title here
%
\author{Ekaterina Verbitskaia\inst{1,2}\orcidID{0000-0002-6828-3698}}
%
\authorrunning{Ekaterina Verbitskaia}
% First names are abbreviated in the running head.
% If there are more than two authors, 'et al.' is used.
%
\institute{Constructor University, Bremen, Germany \and
JetBrains Research, Amsterdam, the Netherlands
\\ \email{kajigor@gmail.com}}
%
\maketitle              % typeset the header of the contribution
%
% \begin{abstract}
% Many programs which solve complicated problems can be seen as inversions of other, much simpler, programs. One particular example is transforming verifiers into solvers, which can be achieved with low effort by implementing the verifier in a relational language and then executing it in the backward direction. Unfortunately, as it is common with inverse computations, interpretation overhead may lead to subpar performance compared to direct program inversion. In this paper we discuss functional conversion aimed at improving relational miniKanren specifications with respect to a known fixed direction. Our preliminary evaluation demonstrates a significant performance increase for some programs which exemplify the approach.

% \keywords{Relational programming \and Partial evaluation \and Program inversion.}
% \end{abstract}
%
%
%
\section{Introduction}

Here is going to be an introduction.
\section{Specialization Efforts for \texttt{miniKanren}}

Specialization, or partial evaluation, is an optimization technique that precomputes parts of program execution based on information known about a program before execution. 
For example, consider a function \texttt{exp n x = if n == 0 then 1 else x * (exp (n - 1) x)} and imagine that we know from some context that it is always being called with the argument \texttt{n} equal to  \texttt{4}. 
In this case, we can partially evaluate the function to \texttt{exp\_4 x = x * x * x * x * 1} that is more efficient than the original function called with \texttt{n = 4}. 
Note, that a smart enough specializer can also be able to generate a function of form \texttt{exp\_4 x = let sqr = x * x in sqr * sqr} that makes even less multiplications. 

This pattern can be expressed in a way that if there is a function with some of its arguments statically known \texttt{f x$_{static}$ y$_{dynamic}$}, it can be transformed into a more efficient function \texttt{f\_x$_{static}$} with its parts dependent on the static arguments precomputed.
The resulting program must be equivalent to the original one, meaning that given the same dynamic arguments, it will return the same results: \texttt{f x$_{static}$ y$_{dynamic}$ == f\_x$_{static}$ y$_{dynamic}$}.

In the field of logic programming, specialization is generally known as partial deduction. 
Besides the values of static arguments, a partial deducer can also consider the information about a direction of a program or the interaction between logic variables in a conjunction of calls. 
In addition to specialization, a relation with a given direction can be converted into a function in which expensive logic operations are replaced with streamlined functional counterparts. 
In this project, we combine the two methods for the verifier-to-solver approach.

\subsection{Conservative Partial Deduction}


Conjunctive Partial Deduction was first developed by Michael Leuschel for \texttt{Prolog} in the ECCE system~\cite{de1999conjunctive}. 
It makes use of the interaction between conjuncts for specialization, getting rid of some repeating traversals of data structures as a result. 
We implemented this algorithm as a proof-of-concept for \texttt{miniKanren}, and found out that some of the specialization results were subpar.
In some cases, the specialized programs had worse performance than the original ones. 

Then, we formulated a different specialization method, which we called Conservative Partial Deduction. 
The difference with CPD lies in the way conjunctions are treated. 
They are split more often and thus generate smaller programs. 
The method was able to achieve a 1.5-2 times performance increase on a propositional evaluator program and almost a 40 times performance increase on a type checker. 

\subsection{Functional Conversion}

Even after the partial deduction has finished, there are some sources of inefficiency that can be addressed. 
The base of relational programming is unification, which is an expensive operation, especially if it runs occurs-check. 
When a direction of a relation is known, mode analysis~\cite{somogyi1987system,smaus2000mode} can be run to determine data flow from the statically known variables. 
As a result, most unifications can be converted into assignments, pattern matching or equality checks. 
Besides that, we can reorder the calls within a conjunction to restrict the search space early. 

In some cases, when the direction is deterministic, the relation can be transformed into a pure function with no overhead of the relational language. 
However, if the direction results in multiple possible answers, we still need to express non-determinism. 
This is possible in a functional programming language by using the \texttt{Stream} monad that is at the base of the relational programming. 
We implemented a functional conversion method that gave the propositional evaluation program a 2.5-fold increase in performance.
For some programs dealing with arithmetic, it improved performance by up to two orders of magnitude.


\section{Published Papers}

\begin{itemize}
    \item Petr Lozov, Ekaterina Verbitskaia, and Dmitry Boulytchev. \emph{``Relational interpreters for search problems.''} Relational Programming Workshop. 2019. \href{https://dash.harvard.edu/bitstream/handle/1/41307116/tr-02-19.pdf?sequence=1&isAllowed=y#page=47}{The full text is available here}.
    \begin{itemize}
        \item This paper formalizes the verifier-to-solver approach and highlights its pitfalls. 
    \end{itemize}
    \item Ekaterina Verbitskaia, Daniil Berezun, Dmitry Boulytchev. \emph{``An Empirical Study of Partial Deduction for \textsc{miniKanren}.''} Verification and Program Transformation Workshop. 2021. \href{https://cgi.cse.unsw.edu.au/~eptcs/paper.cgi?VPT2021.5.pdf}{The full text is available here}. 
    \begin{itemize}
        \item This paper describes a novel partial evaluation approach developed for \textsc{miniKanren}. 
    \end{itemize}
    \item Ekaterina Verbitskaia, Igor Engel, Daniil Berezun. \emph{``A Case Study in Functional Conversion and Mode Inference in \textsc{miniKanren}.''}  Partial Evaluation and Program Manipulation Workshop. 2024. \href{https://web.archive.org/web/20240111175807id_/https://dl.acm.org/doi/pdf/10.1145/3635800.3636966}{The full text is available here}.
    \begin{itemize}
        \item This paper describes functional conversion that makes it possible to generate a functional implementation of a solver from the functional implementation of a verifier, thus getting rid of the inefficiencies innate to the relational programming itself.  
    \end{itemize}

% http://minikanren.org/workshop/2020/minikanren-2020-paper2.pdf
% 
\end{itemize}




\bibliographystyle{plain}
\bibliography{main}

\end{document}
