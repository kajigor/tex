% This is samplepaper.tex, a sample chapter demonstrating the
% LLNCS macro package for Springer Computer Science proceedings;
% Version 2.21 of 2022/01/12
%
\documentclass[runningheads]{llncs}
\usepackage{hyperref}
\usepackage{geometry}
\geometry{
  a4paper,         % or letterpaper
  textwidth=15cm,  % llncs has 12.2cm
  textheight=24cm, % llncs has 19.3cm
  heightrounded,   % integer number of lines
  hratio=1:1,      % horizontally centered
  vratio=2:3,      % not vertically centered
}

%
\usepackage[T1]{fontenc}
% T1 fonts will be used to generate the final print and online PDFs,
% so please use T1 fonts in your manuscript whenever possible.
% Other font encondings may result in incorrect characters.
%
\usepackage{graphicx}
% Used for displaying a sample figure. If possible, figure files should
% be included in EPS format.
%
% If you use the hyperref package, please uncomment the following two lines
% to display URLs in blue roman font according to Springer's eBook style:
%\usepackage{color}
%\renewcommand\UrlFont{\color{blue}\rmfamily}
%\urlstyle{rm}
%
\begin{document}
%
\title{Towards Efficient Search: Leveraging Relational Interpreters and Partial Deduction Techniques}

% \subtitle{PhD Thesis Proposal}
%
%\titlerunning{Abbreviated paper title}
% If the paper title is too long for the running head, you can set
% an abbreviated paper title here
%
\author{Ekaterina Verbitskaia\inst{1,2}\orcidID{0000-0002-6828-3698}}
%
\authorrunning{Ekaterina Verbitskaia}
% First names are abbreviated in the running head.
% If there are more than two authors, 'et al.' is used.
%
\institute{Constructor University, Bremen, Germany \and
JetBrains Research, Amsterdam, the Netherlands
\\ \email{kajigor@gmail.com}}
%
\maketitle              % typeset the header of the contribution
%
% \begin{abstract}
% Many programs which solve complicated problems can be seen as inversions of other, much simpler, programs. One particular example is transforming verifiers into solvers, which can be achieved with low effort by implementing the verifier in a relational language and then executing it in the backward direction. Unfortunately, as it is common with inverse computations, interpretation overhead may lead to subpar performance compared to direct program inversion. In this paper we discuss functional conversion aimed at improving relational miniKanren specifications with respect to a known fixed direction. Our preliminary evaluation demonstrates a significant performance increase for some programs which exemplify the approach.

% \keywords{Relational programming \and Partial evaluation \and Program inversion.}
% \end{abstract}
%
%
%
\section{Introduction}

Implementing a program is often significantly easier than its inversion.
For example, integer multiplication is much simpler than factoring, while program evaluation is easier than program generation.
Although inversion is undecidable, there are approaches capable of inversing a computation in some cases, notably, universal resolving algorithm \todo{cite Gluck}, logic and relational programming.
Inversion comes with a lot of overhead which may be reduced in some circumstances.
% The focus of this paper is to reduce overhead when implementing inversions of  programs written in \mk by translating them into functional counterparts.

One source of overhead in relational programming comes from \emph{unification} --- the basic operation which is at the core of \mk.
Unification involves traversing terms being unified along with a list of substitutions and doing occurs-check all of which may be redundant when there is a specific execution \emph{direction} in mind.
Directions fix at compile-time which arguments of a relation are always going to be known and ground at runtime.
Having this information, it is possible to specialize a relation for the direction \todo{cite Verbitskaia} and get rid of some of the overhead.
In this case, unifications may prove to be redundant and be replaced with much simpler pattern-matching and equality checks.

In this paper we present a scheme of translation of \mk programs into a host functional programming language as a sequence of examples.
Examples start from the simplest translations and evolve to introduce different features of \mk which influence translation.
Currently translation is not automated: everything is done manually.
We believe the translation can be semi-automated, leaving some decisions up to a programmer.
Although this project is at the early state, evaluation demonstrates its usefulness by significantly speeding up such programs as computing a topological sorting of a graph and generating logic formulas which evaluate to the given value.





\section{Published Papers}

\begin{itemize}
    \item Petr Lozov, Ekaterina Verbitskaia, and Dmitry Boulytchev. \emph{``Relational interpreters for search problems.''} Relational Programming Workshop. 2019. \href{https://dash.harvard.edu/bitstream/handle/1/41307116/tr-02-19.pdf?sequence=1&isAllowed=y#page=47}{The full text is available here}.
    \begin{itemize}
        \item This paper formalizes the verifier-to-solver approach and highlights its pitfalls. 
    \end{itemize}
    \item Ekaterina Verbitskaia, Daniil Berezun, Dmitry Boulytchev. \emph{``An Empirical Study of Partial Deduction for \textsc{miniKanren}.''} Verification and Program Transformation Workshop. 2021. \href{https://cgi.cse.unsw.edu.au/~eptcs/paper.cgi?VPT2021.5.pdf}{The full text is available here}. 
    \begin{itemize}
        \item This paper describes a novel partial evaluation approach developed for \textsc{miniKanren}. 
    \end{itemize}
    \item Ekaterina Verbitskaia, Igor Engel, Daniil Berezun. \emph{``A Case Study in Functional Conversion and Mode Inference in \textsc{miniKanren}.''}  Partial Evaluation and Program Manipulation Workshop. 2024. \href{https://web.archive.org/web/20240111175807id_/https://dl.acm.org/doi/pdf/10.1145/3635800.3636966}{The full text is available here}.
    \begin{itemize}
        \item This paper describes functional conversion that makes it possible to generate a functional implementation of a solver from the functional implementation of a verifier, thus getting rid of the inefficiencies innate to the relational programming itself.  
    \end{itemize}

% http://minikanren.org/workshop/2020/minikanren-2020-paper2.pdf
% 
\end{itemize}




\bibliographystyle{plain}
\bibliography{main}

\end{document}
