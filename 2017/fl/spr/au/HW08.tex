\documentclass[12pt]{article}
\usepackage[left=2cm,right=2cm,top=2cm,bottom=2cm,bindingoffset=0cm]{geometry}
\usepackage{hyperref}
\usepackage{fontspec}
\usepackage{polyglossia}
\setdefaultlanguage{russian}
\setmainfont[Mapping=tex-text]{CMU Serif}

\begin{document}
%% Весь этот текст можно удалить
%% ====== от сих =====
\centering {\LARGE Формальные языки}

{\Large домашнее задание до 23:59 20.04}
\bigskip

\begin{enumerate}
  \item Прошлое домашнее задание.
  \item Добавить построение дерева при синтаксическом анализе при помощи алгоритма CYK (3 балла). 
  \begin{itemize}
      \item Используйте реализацию алгоритма из прошлого задания, изменения должны быть небольшими.
      \item Результатом является:
      \begin{itemize}
        \item ответ на вопрос, принадлежит ли цепочка языку, задаваемому грамматикой;
        \item таблица, которая строится алгоритмом (выводить в человекочитаемом виде, таблицу можно в \verb!.csv! сохранить);
        \item дерево разбора (в человекочитаемом виде, можно в \verb!.dot!).
      \item Входная грамматика может быть не в НФХ, тогда нужно запустить сначала процедуру нормализации из второй задачи. 
  \end{itemize}
  
\end{enumerate}

%% ===== и до сих =====
\end{document}
