 \documentclass[12pt]{article}
\usepackage[left=2cm,right=2cm,top=2cm,bottom=2cm,bindingoffset=0cm]{geometry}
\usepackage{hyperref}
\usepackage{fontspec}
\usepackage{polyglossia}
\setdefaultlanguage{russian}
\setmainfont[Mapping=tex-text]{CMU Serif}

\begin{document}
%% Весь этот текст можно удалить
%% ====== от сих =====
\centering {\LARGE Формальные языки}

{\Large домашнее задание до 23:59 30.03}
\bigskip

\begin{enumerate}
  \item Расширить лексер возможностью обработки многострочных комментариев. Комментарии могут содержать произвольный текст, весь комментарий --- одна лексема. Добавить тесты. 
  \begin{itemize}
      \item Формат многострочного комментария \verb!(*comment*)! (1 балл)
      \item Изучить вопрос поддержки вложенных многострочных комментариев тулом, которым вы пользуетесь. Выбрать стратегию, которой будет придерживаться ваша реализация. (2 балла)
      \begin{itemize}
          \item Распознавать концом комментария первую встретившуюся закрывающуюся скобку (\verb!(*(*comment*)*) -> MLComment ("(*comment"), Op ("*"), R_Bracket!)
          \item Распознавать концом комментария самую крайнюю закрывающуюся скобку (\verb!(*(*comment*)*) -> MLComment ("(*comment*)")!)
          \item Распознавать честно вложенные комментарии. Это умеет OCaml, можно поизучать, как он себя ведет. 
      \end{itemize}
  \end{itemize}
  \item Добавить в язык L операцию возведения в степень \verb!**!, не забыть про тесты (1 балл)
  \begin{itemize}
      \item Обратите внимание на разные значения символа \verb!*! в нашем языке
  \end{itemize}
  \item Добавить в язык L присвоение \verb!:=!, не забыть про тесты (1 балл)
    
  \item Осуществить фильтрацию комментариев из потока лексем, выдаваемых лексером языка L (2 балла)
  \begin{itemize}
      \item Консольное приложение для лексера должно иметь возможность быть запущенным в двух режимах: с фильтрацией комментариев и без фильтрации (добавить ключ -filter)
  \end{itemize}
  
\end{enumerate}

%% ===== и до сих =====
\end{document}

 
 
 
 
 
 
 
 
