\documentclass[12pt]{article}
\usepackage[left=2cm,right=2cm,top=2cm,bottom=2cm,bindingoffset=0cm]{geometry}
\usepackage{hyperref}
\usepackage{fontspec}
\usepackage{polyglossia}
\setdefaultlanguage{russian}
\setmainfont[Mapping=tex-text]{CMU Serif}

\begin{document}
%% Весь этот текст можно удалить
%% ====== от сих =====
\centering {\LARGE Формальные языки}

{\Large домашнее задание до 23:59 13.04}
\bigskip

\begin{enumerate}
  \item Прошлое домашнее задание.
  \item Реализовать преобразование КС-грамматики в нормальную форму Хомского. Требования к оформлению задачи обычные: консольное приложение, принимающее как аргумент путь к файлу (с описанием грамматики), код и инструкция по запуску на гитхабе, скрипты сборки и запуска тестов.
  \begin{itemize}
      \item Каждый шаг преобразования оформлять в отдельную функцию (по 1 баллу за каждый шаг).
      \item Тесты должны включать три грамматики:
      \begin{itemize}
          \item однозначная грамматика правильных скобочных последовательностей;
          \item неоднозначная грамматика правильных скобочных последовательностей;
          \item однозначная грамматика арифметических выражений с $+$ и $*$ над цифрами (просто однозначные числа $0 \;|\; 1 \;|\; \dots \;|\; 9$).
      \end{itemize} 
      \item Формат записи грамматики (на вход подаются грамматики в этом формате; выводить тоже в ней): 
      \begin{itemize}
          \item Пустые строки обозначаются последовательностью \verb!eps!.
          \item Каждый нетерминал на новой строке, для разделения альтернатив используется $|$, конкатенация символов в правой части правила обозначается пробелом.
          \item Терминалы в одинарных кавычках.
          \item Нетерминалы: все, что не терминалы.
          \item Имя стартового нетерминала записывается на первой строке.
          \item Пример входного файла:
      \end{itemize}
  \end{itemize}

\begin{verbatim}
E
E = E '+' T | T
T = T '*' F | F
F = '(' E ')' | '0' | '1' | '2' | '3' | '4' | '5' | '6' | '7' | '8' | '9'
\end{verbatim}

  \item Реализовать алгоритм синтаксического анализа CYK. Требования к оформлению задачи обычные: консольное приложение, принимающее как аргумент пути к двум файлам (первый --- с описанием грамматики и второй --- со входной последовательностью), код и инструкция по запуску на гитхабе, скрипты сборки и запуска тестов. (6 баллов)
  \begin{itemize}
      \item Результатом является ответ на вопрос, принадлежит ли цепочка языку, задаваемому грамматикой, а также таблица, которая строится алгоритмом (выводить в человекочитаемом виде, таблицу можно в \verb!.csv! сохранить).
      \item Входная грамматика может быть не в НФХ, тогда нужно запустить сначала процедуру нормализации из второй задачи. 
  \end{itemize}
  
\end{enumerate}

%% ===== и до сих =====
\end{document}
