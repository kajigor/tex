\documentclass[12pt]{article}
\usepackage[left=1.5cm,right=1.5cm,top=2cm,bottom=2cm,bindingoffset=0cm]{geometry}
\usepackage{hyperref}
\usepackage{fontspec}
\usepackage{polyglossia}
\usepackage{amssymb}
\setdefaultlanguage{russian}
\setmainfont[Mapping=tex-text]{CMU Serif}

\begin{document}
%% Весь этот текст можно удалить
%% ====== от сих =====
\centering {\LARGE Формальные языки}

{\Large домашнее задание до 23:59 19.10}
\bigskip

\begin{enumerate}
  \item 
  {
    Как построить контекстно-свободную грамматику для произвольного конечного языка? Сколько нетерминалов для этого необходимо? Сколько необходимо нетерминалов для регулярной грамматики для конечного языка?
  }
  \item 
  {  
    Построить контекстно-свободную грамматику для языка $\{a^n b c^m \mid n, m \in \mathbb{N}\}$. Привести деревья вывода для двух нетривиальных цепочек.
  }
  \item 
  { 
    Построить контекстно-свободную грамматику для if-then-else выражений, в которых ветка else опциональна. Обозначайте условия за C, элементарный фрагмент кода в ветке --- за S. if-выражения могут быть вложенными. Постройте деревья для двух нетривиальных цепочек языка, как минимум одна из которых не из нижеперечисленных. Ниже корректные цепочки языка.
    \begin{itemize}
        \item if C then S else S 
        \item if C then S 
        \item if C then S else if C then S else if C then S else S
    \end{itemize} 
  }
  \item { Построить контекстно-свободную грамматику для языка $\{ (ac)^n (cb)^n \mid n \in \mathbb{N}\}$. Привести деревья разбора для двух нетривиальных цепочек. }
  
\end{enumerate}

%% ===== и до сих =====
\end{document}
