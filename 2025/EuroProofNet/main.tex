% This is samplepaper.tex, a sample chapter demonstrating the
% LLNCS macro package for Springer Computer Science proceedings;
% Version 2.21 of 2022/01/12
%
\documentclass[runningheads]{llncs}
%
\usepackage[T1]{fontenc}
% T1 fonts will be used to generate the final print and online PDFs,
% so please use T1 fonts in your manuscript whenever possible.
% Other font encondings may result in incorrect characters.
%
\usepackage{graphicx}
% Used for displaying a sample figure. If possible, figure files should
% be included in EPS format.
%
% If you use the hyperref package, please uncomment the following two lines
% to display URLs in blue roman font according to Springer's eBook style:
%\usepackage{color}
%\renewcommand\UrlFont{\color{blue}\rmfamily}
%\urlstyle{rm}
%
\usepackage{xcolor}
\usepackage{listings}
\usepackage{hyperref}

\begin{document}
%
\title{Can LLMs Enable Verification in Mainstream Programming?}
%
%\titlerunning{Abbreviated paper title}
% If the paper title is too long for the running head, you can set
% an abbreviated paper title here
%
\author{Aleksandr Shefer\inst{1,2}\orcidID{0000-1111-2222-3333} \and
Igor Engel\inst{1,2}\orcidID{1111-2222-3333-4444} \and
Stanislav Alekseev\inst{1,3}\orcidID{2222--3333-4444-5555} \and
Daniil Berezun\inst{1}\orcidID{3333-4444-5555-6666} \and
Ekaterina Verbitskaia\inst{1,2}\orcidID{0000-0002-6828-3698}}
%
\authorrunning{A. Shefer et al.}
% First names are abbreviated in the running head.
% If there are more than two authors, 'et al.' is used.
%
\institute{JetBrains Research, Amsterdam, the Netherlands \and 
Constructor University, Bremen, Germany \and 
Neapolis University, Pafos, Cyprus }
% \institute{Princeton University, Princeton NJ 08544, USA \and
% Springer Heidelberg, Tiergartenstr. 17, 69121 Heidelberg, Germany
% \email{lncs@springer.com}\\
% \url{http://www.springer.com/gp/computer-science/lncs} \and
% ABC Institute, Rupert-Karls-University Heidelberg, Heidelberg, Germany\\
% \email{\{abc,lncs\}@uni-heidelberg.de}}
%
\maketitle              % typeset the header of the contribution
%

\newcommand{\todo}[1]{{\color{red} #1}}

\begin{abstract}
Many programs which solve complicated problems can be seen as inversions of other, much simpler, programs.
One particular example is transforming verifiers into solvers which can be achieved with low effort by implementing the verifier in a relational language and then executing it in the backward direction.
Unfortunately, as it is common with inverse computations, interpretation overhead may lead to subpar performance compared to direct program inversion.
In this paper we discuss functional conversion aimed at improving relational \mk specifications with respect to a known fixed direction.
Our preliminary evaluation demonstrates a significant performance increase for some programs which exemplify the approach.
\end{abstract}

Software is notoriously difficult to get right. 
Off-by-one errors, null dereferencing, and infinite loops are among the most common avoidable mistakes developers tend to make. 
Formal methods aim to prevent these and more complicated errors by providing a programmer with means to reason about a program and prove its correctness.  
Such tools are especially valued in critical domains such as healthcare and finance, where software failures can have severe consequences. 
However, adopting formal verification requires significant additional effort and expertise, which limits their use beyond high-stakes applications.

SMT-powered software verification systems such as Dafny and F* partially automate proof search but remain standalone tools. 
This implies that to introduce verification into an existing project, one needs to make a tough decision of adopting a new language, which often comes with worse developer tools and a higher entrance barrier for the engineers. 
One way to overcome this drawback is to use an intermediate verification language such as Viper\footnote{Viper system: \url{https://www.pm.inf.ethz.ch/research/viper.html}}, with frontends in mainstream languages.
The last hurdle to clear is to make it easy for developers to specify properties of their programs, as well as to prove that they hold. 

\todo{Prior research!!!}
In this project, we explore whether large language models are capable of generating verified code in mainstream languages from a text description and a partial specification. 
We focus on Nagini\footnote{Nagini, an automatic verifier for Python programs: \url{https://github.com/marcoeilers/nagini}} and Verus\footnote{Verus, a tool for verifying code correctness in Rust: \url{https://github.com/verus-lang/verus}} -- the extensions of the popular programming languages Python and Rust. 


In addition to a function signature and a body, verified code contains a specification of its behavior. 
It includes preconditions that describe assumptions held before the evaluation of the function begins and guarantees ensured after execution, called postconditions.
Sometimes these are enough to establish correctness, bun in the majority of non-trivial cases, additional statements should be proven, such as loop invariants or lemmas.

This provides a spectrum of data which can be exposed to a model in code synthesis tasks ranging from a textual description of the problem at hands to everything but loop invariants. 
In this extended abstract, we only focus on a scenario where the user describes the problem in a natural language and supplies a function signature with pre- and postconditions, as we view it as the most precise way to express the user intent. 
An example of a query is ``Checks if given string is a palindrome'' along with the following code. 

\begin{lstlisting}[language=Python,basicstyle=\footnotesize]
def is_palindrome(text : List[int]) -> bool:
  Requires(Acc(list_pred(text))) # precondition
  Ensures(Acc(list_pred(text)))  # postconditions 
  Ensures(Result() == Forall(int, lambda i:
    (i >= 0 and i < len(text)) 
    implies (text[i] == text[len(text) - i - 1])))
\end{lstlisting}

The model is them prompted to generate the function body as well as any necessary additional conditions necessary to finish the proof.
We use few-shot The complete prompt can be found in the Appendix\todo{(ref)}.
If the produced code verifies, it is accepted and passed to the user. 
Otherwise, the verifier feedback is sent to the model for further revision of the suggestion; this process is repeated up to \todo{5} times. 

In order to evaluate the abilities of the model, we created a benchmark based on HumanEval\cite{chen2021evaluating}.
We manually implemented a subset of the problems in Nagini\footnote{HumanEval dataset in Nagini: \url{https://github.com/JetBrains-Research/HumanEval-Nagini/}} and contributed in a collaborative effort to create it for Verus\footnote{HumanEval dataset in Verus: \url{https://github.com/secure-foundations/human-eval-verus}}. 
Not every problem in the original dataset suits well for verification. 
For some of them, specification duplicates the implementation \todo{(numbers)}, while unsupported language features are needed for others \todo{(numbers)}. 
In total, our benchmarks contain 106 problems for Nagini and \todo{this many} for Verus. 

We ran the described experiment on Claude-3.5-Sonnet, which successfully produced verified code for 67 problems (63\%) in Nagini and \todo{this many} in Verus. 
We conclude that using large language models can enable verification in mainstream programming languages. 



% \section{Introduction}

Implementing a program is often significantly easier than its inversion.
For example, integer multiplication is much simpler than factoring, while program evaluation is easier than program generation.
Although inversion is undecidable, there are approaches capable of inversing a computation in some cases, notably, universal resolving algorithm \todo{cite Gluck}, logic and relational programming.
Inversion comes with a lot of overhead which may be reduced in some circumstances.
% The focus of this paper is to reduce overhead when implementing inversions of  programs written in \mk by translating them into functional counterparts.

One source of overhead in relational programming comes from \emph{unification} --- the basic operation which is at the core of \mk.
Unification involves traversing terms being unified along with a list of substitutions and doing occurs-check all of which may be redundant when there is a specific execution \emph{direction} in mind.
Directions fix at compile-time which arguments of a relation are always going to be known and ground at runtime.
Having this information, it is possible to specialize a relation for the direction \todo{cite Verbitskaia} and get rid of some of the overhead.
In this case, unifications may prove to be redundant and be replaced with much simpler pattern-matching and equality checks.

In this paper we present a scheme of translation of \mk programs into a host functional programming language as a sequence of examples.
Examples start from the simplest translations and evolve to introduce different features of \mk which influence translation.
Currently translation is not automated: everything is done manually.
We believe the translation can be semi-automated, leaving some decisions up to a programmer.
Although this project is at the early state, evaluation demonstrates its usefulness by significantly speeding up such programs as computing a topological sorting of a graph and generating logic formulas which evaluate to the given value.



% % \section{Method}

% Describe a general scenario

In addition to function signature and body, verified code contains a specification of its behavior. 
It includes preconditions that describe assumptions correct before evaluation of the function begins and guarantees ensured after execution called postconditions.
Sometimes these are enough to establish correctness, bun in the majority of non-trivial cases, additional statements should be proven, such as loop invariants or lemmas.

% Explain modes 1 and 4 in detail. 

This provides a spectrum of components which can be exposed to a model in code synthesis tasks ranging from a textual description of the problem at hands to everything but loop invariants. 
In this extended abstract, we only focus on a scenario where the user describes the problem in a natural language and supplies a function signature and pre- and postconditions, as we view this as the best way to precisely express the user intent. 
An example of a query is ``Checks if given string is a palindrome'' along with the following code. 

\begin{lstlisting}[language=Python]
def is_palindrome(text : List[int]) -> bool:
  Requires(Acc(list_pred(text))) # precondition
  Ensures(Acc(list_pred(text)))  # postconditions 
  Ensures(Result() == Forall(int, lambda i:
      (i >= 0 and i < len(text)) 
      implies (text[i] == text[len(text) - i - 1]))))
\end{lstlisting}

The model then generates the function body as well as any necessary additional conditions needed to finish the proof. 
If the produced code verifies, it gets accepted and passed to the user. 
Otherwise, the verifier feedback is sent to the model for further revision of the suggestion; this process is repeated up to \todo{5} times. 

% Describe the benchmark, add links. 

In order to evaluate the abilities of the model, we created a benchmark based on \todo{HumanEval}.
We manually implemented a subset of the problems in Nagini\footnote{HumanEval data set in Nagini: \url{https://github.com/JetBrains-Research/HumanEval-Nagini/}} and contributed in a collaborative effort to create it for Viper\todo{ (add footnote)}. 
Not every problem in the original dataset suits well for verification. 
For some of them, specification duplicates the implementation \todo{(numbers)}, while unsupported language features are needed for others \todo{(numbers)}. 
In total, our benchmarks contain 106 problems for Nagini and \todo{this many} for Verus. 

% Showcase the numbers. 

We run the described experiment on \todo{Claude} and got the results presented in \todo{table}. 
With about 

\begin{table}[]
  \begin{tabular}{l|l|l}
  System & Number of programs & Verified  \\ \hline
  Nagini & 106                & 67 (63\%) \\
  Verus  &                    &          
  \end{tabular}
  \end{table}

We conclude that using large language models can enable verification in mainstream programming languages. 


% Make conclusion. 



%
%
%

% \subsection{A Subsection Sample}
% Please note that the first paragraph of a section or subsection is
% not indented. The first paragraph that follows a table, figure,
% equation etc. does not need an indent, either.

% Subsequent paragraphs, however, are indented.

% \subsubsection{Sample Heading (Third Level)} Only two levels of
% headings should be numbered. Lower level headings remain unnumbered;
% they are formatted as run-in headings.

% \paragraph{Sample Heading (Fourth Level)}
% The contribution should contain no more than four levels of
% headings. Table~\ref{tab1} gives a summary of all heading levels.

% \begin{table}
% \caption{Table captions should be placed above the
% tables.}\label{tab1}
% \begin{tabular}{|l|l|l|}
% \hline
% Heading level &  Example & Font size and style\\
% \hline
% Title (centered) &  {\Large\bfseries Lecture Notes} & 14 point, bold\\
% 1st-level heading &  {\large\bfseries 1 Introduction} & 12 point, bold\\
% 2nd-level heading & {\bfseries 2.1 Printing Area} & 10 point, bold\\
% 3rd-level heading & {\bfseries Run-in Heading in Bold.} Text follows & 10 point, bold\\
% 4th-level heading & {\itshape Lowest Level Heading.} Text follows & 10 point, italic\\
% \hline
% \end{tabular}
% \end{table}


% \noindent Displayed equations are centered and set on a separate
% line.
% \begin{equation}
% x + y = z
% \end{equation}
% Please try to avoid rasterized images for line-art diagrams and
% schemas. Whenever possible, use vector graphics instead (see
% Fig.~\ref{fig1}).

% \begin{figure}
% \includegraphics[width=\textwidth]{fig1.eps}
% \caption{A figure caption is always placed below the illustration.
% Please note that short captions are centered, while long ones are
% justified by the macro package automatically.} \label{fig1}
% \end{figure}

% \begin{theorem}
% This is a sample theorem. The run-in heading is set in bold, while
% the following text appears in italics. Definitions, lemmas,
% propositions, and corollaries are styled the same way.
% \end{theorem}
% %
% % the environments 'definition', 'lemma', 'proposition', 'corollary',
% % 'remark', and 'example' are defined in the LLNCS documentclass as well.
% %
% \begin{proof}
% Proofs, examples, and remarks have the initial word in italics,
% while the following text appears in normal font.
% \end{proof}
% For citations of references, we prefer the use of square brackets
% and consecutive numbers. Citations using labels or the author/year
% convention are also acceptable. The following bibliography provides
% a sample reference list with entries for journal
% articles~\cite{ref_article1}, an LNCS chapter~\cite{ref_lncs1}, a
% book~\cite{ref_book1}, proceedings without editors~\cite{ref_proc1},
% and a homepage~\cite{ref_url1}. Multiple citations are grouped
% \cite{ref_article1,ref_lncs1,ref_book1},
% \cite{ref_article1,ref_book1,ref_proc1,ref_url1}.

% \begin{credits}
% \subsubsection{\ackname} A bold run-in heading in small font size at the end of the paper is
% used for general acknowledgments, for example: This study was funded
% by X (grant number Y).

% \subsubsection{\discintname}
% It is now necessary to declare any competing interests or to specifically
% state that the authors have no competing interests. Please place the
% statement with a bold run-in heading in small font size beneath the
% (optional) acknowledgments\footnote{If EquinOCS, our proceedings submission
% system, is used, then the disclaimer can be provided directly in the system.},
% for example: The authors have no competing interests to declare that are
% relevant to the content of this article. Or: Author A has received research
% grants from Company W. Author B has received a speaker honorarium from
% Company X and owns stock in Company Y. Author C is a member of committee Z.
% \end{credits}
% %

\bibliographystyle{plain}
\bibliography{main}
\end{document}
