\documentclass[12pt,a4paper]{article}%
\usepackage{amsthm}
\usepackage{amsmath}%
\usepackage{amsfonts}%
\usepackage{amssymb}%
\usepackage{graphicx}
\usepackage[T2A]{fontenc}
\usepackage[utf8]{inputenc}
\usepackage[english,russian]{babel}
%-------------------------------------------
\setlength{\textwidth}{7.0in}
\setlength{\oddsidemargin}{-0.35in}
\setlength{\topmargin}{-0.5in}
\setlength{\textheight}{9.0in}
\setlength{\parindent}{0.3in}

\newtheorem{theorem}{Theorem}
\newtheorem{task}[theorem]{Задача}
\addto\captionsrussian{\renewcommand*{\proofname}{Решение}}


\newcommand{\abovemath}[2]{\ensuremath{\stackrel{\text{#1}}{#2}}}
\newcommand{\aboveeq}[1]{\abovemath{#1}{=}}
\newcommand\bydef{\aboveeq{def}}
\begin{document}


\begin{flushright}
\textbf{Екатерина Вербицкая \\
\today}
\end{flushright}

\begin{center}
\textbf{Формальные языки\\
HW01} \\
\end{center}

\task{Привести 3 примера строк из языка $\{\alpha \cdot a b b a b \cdot \beta \mid \alpha, \beta \in \{a, b\}^*\}$}

\begin{proof} \hfill % hfill нужен, чтобы список начинался со следующей страницы
\begin{itemize}
  \item $abbab$
  \item $aabbab$
  \item $abbababbab$
\end{itemize}
\end{proof}

\task{Доказать, что для любого языка $L$ верно $(L^r)^r = L$}

\begin{proof}{}
  Раскроем левую часть равенства по определению и воспользуемся тем, что обращение строк --- инволюция.

  \[
    (L^r)^r \bydef \{ m^r \mid m \in L^r\} \bydef \{ (l^r)^r \mid l \in L\} \aboveeq{invol} \{ l \mid l \in L\} \bydef L
  \]
\end{proof}

\task{Доказать, что P = NP}
% Задача, которая осталась без решения

\end{document}
