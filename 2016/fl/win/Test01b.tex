\documentclass[12pt]{article}
\usepackage[left=2cm,right=2cm,top=2cm,bottom=2cm,bindingoffset=0cm]{geometry}
\usepackage{hyperref}
\usepackage{fontspec}
\usepackage{polyglossia}
\setdefaultlanguage{russian}
\pagestyle{empty}
\setmainfont[Mapping=tex-text]{CMU Serif}

\begin{document}
%% Весь этот текст можно удалить
%% ====== от сих =====
\centering {\LARGE Формальные языки}

{\Large Контрольная работа 1, вариант 2}
\bigskip

\begin{enumerate}
  \item Построить минимальный детерминированный конечный автомат для языка \\ $L = \{ \omega a b \, | \, \omega \in \{ 0, 1, \dots, 9 \}^*; a, b \in \{ 0, 1, \dots, 9 \}; a + b = 5 \} $
  \item { 
         Построить автомат, распознающий разность языков, распознаваемых следующими автоматами
         
         \includegraphics[width=250pt]{../05.03/2_0.png}
         
~\\~
         \includegraphics[width=150pt]{../05.03/2_1.png} 
        }
    \item Построить регулярную грамматику, задающую язык \\ $L = \{ a \omega b \, | \, \omega \in \{ 0, 1 \}^*; a, b \in \{ 0, 1 \}; a \, xor \, b = 0 \} $
    \item Проверить регулярность языка арифметических выражений (операции +, *) в постфиксной записи. Если язык регулярный --- привести грамматику, автомат или регулярное выражение, его задающее. Если нет --- привести доказательство
    \item {Построить эпсилон-замыкание автомата
           
           \includegraphics[width=0.9\textwidth]{test1b_ec.PNG}
    }
\end{enumerate}
%% ===== и до сих =====
\end{document}
