\documentclass[12pt]{article}
\usepackage[left=2cm,right=2cm,top=2cm,bottom=2cm,bindingoffset=0cm]{geometry}
\usepackage{fontspec}
\usepackage{polyglossia}
\setdefaultlanguage{russian}
\setmainfont[Mapping=tex-text]{CMU Serif}

\begin{document}
\centering {\LARGE Формальные языки}

{\Large домашнее задание до 23:59 12.09}
\bigskip

\enumerate
{
  \item
  {
    Перечислить слова языка $L_1 \cap L_2,$ где $L_1 = \{ (ab)^n \mid n \geq 0 \}$ и $L_2 = \{ a^m b^m \mid m \geq 0 \}$. Доказать, что других цепочек в пересечении нет. 
  }
  \item
  {
    Пусть $V_T = \{a,b,c\}$. Равны ли языки $L_1 = \{ (abc)^n a \mid n \geq 2 \}$ и $L_2 = \{ ab (cab)^n ca \mid n \geq 1 \}$? Привести аргументацию точки зрения. 
  }
  \item
  {
  Описать язык (как множество), порождаемый грамматикой

$$
\begin{array}{crcl}
 &F & {\to} & a F H  \\
 &F &  {\to} & a b c  \\
 &b H &  {\to} & b b c  \\
 &c H &  {\to} & H c
 \end{array}
$$
  }
  \item 
  {
    Записать КС-грамматику языка знаковых чисел с плавающей точкой. Привести деревья вывода для трех нетривиальных цепочек.
    \begin{itemize} 
          \item \verb;0, +0, -0, 10, 0.0, 0.1e0, 0.10010e+1, 10e-123, .1, 2.; --- числа
          \item \verb;01, +.2, e, e., .,; $\varepsilon$ --- не числа
        \end{itemize}
  }
  \item 
  {
    Доказать (или опровергнуть), что любой язык можно описать с помощью некоторого конечного описания.
  }
}
\end{document}
