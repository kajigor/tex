\documentclass{article}

\usepackage[left=2cm,right=2cm,top=2cm,bottom=2cm,bindingoffset=0cm]{geometry}
\usepackage{listings}
\usepackage{indentfirst}
\usepackage{verbatim}
\usepackage{amsmath, amsthm, amssymb}
\usepackage{stmaryrd}
\usepackage{graphicx}
\usepackage{euscript}

\usepackage[utf8]{inputenc}
\usepackage[english,russian]{babel}
\usepackage[T2A]{fontenc}

\lstdefinelanguage{llang}{
keywords={skip, do, while, read, write, if, then, else},
sensitive=true,
%%basicstyle=\small,
commentstyle=\scriptsize\rmfamily,
keywordstyle=\ttfamily\underbar,
identifierstyle=\ttfamily,
basewidth={0.5em,0.5em},
columns=fixed,
fontadjust=true,
literate={->}{{$\to$}}1
}

\lstset{
language=llang
}

\newcommand{\aset}[1]{\left\{{#1}\right\}}
\newcommand{\term}[1]{\mbox{\texttt{\bf{#1}}}}
\newcommand{\cd}[1]{\mbox{\texttt{#1}}}
\newcommand{\sembr}[1]{\llbracket{#1}\rrbracket}
\newcommand{\conf}[1]{\left<{#1}\right>}
\newcommand{\fancy}[1]{{\cal{#1}}}

\newcommand{\trule}[2]{\frac{#1}{#2}}
\newcommand{\crule}[3]{\frac{#1}{#2},\;{#3}}
\newcommand{\withenv}[2]{{#1}\vdash{#2}}
\newcommand{\trans}[3]{{#1}\xrightarrow{#2}{#3}}
\newcommand{\ctrans}[4]{{#1}\xrightarrow{#2}{#3},\;{#4}}
\newcommand{\llang}[1]{\mbox{\lstinline[mathescape]|#1|}}
\newcommand{\pair}[2]{\inbr{{#1}\mid{#2}}}
\newcommand{\inbr}[1]{\left<{#1}\right>}
\newcommand{\highlight}[1]{\color{red}{#1}}
\newcommand{\ruleno}[1]{\eqno[\textsc{#1}]}
\newcommand{\inmath}[1]{\mbox{$#1$}}
\newcommand{\lfp}[1]{fix_{#1}}
\newcommand{\gfp}[1]{Fix_{#1}}

\newcommand{\NN}{\mathbb N}
\newcommand{\ZZ}{\mathbb Z}

\begin{document}

%% Весь этот текст можно удалить
%% ====== от сих =====
\centering {\LARGE Формальные языки}

{\Large домашнее задание до 23:59 03.10}
\bigskip

\begin{enumerate}
  \item {Задание 5 из предыдущего домашнего задания} (2 балла за полностью выполненное задание)
  \item Написать лексер для языка L (описание языка ниже), используя любимый генератор лексеров из семейства Lex. (8 баллов за полностью выполненное задание)
    \begin{itemize}
        \item Структуры данных для лексем должны однозначно их идентифицировать, а также содержать привязку к коду (в какой строке исходного кода и с какого по какой символ располагается лексема). 
        \item Составить набор тестов, демонстрирующий правильность работы полученного лексера (качество тестового покрытия важно!).
        \item Сделать консольное  приложение, принимающее на вход путь к файлу, содержащему программу на языке L, производящее лексический анализ и печатающее полученный поток лексем.
        \item Код должен быть размещен на гитхабе, собираться одним скриптом, содержать инструкцию по сборке и запуску собранного приложения, собираться на чистой Ubuntu 15.04 или Windows 8. Все зависимости, в случае их отсутствия в системе, должны либо доставляться скриптом, либо перечисляться в явном виде. 
     \end{itemize}
\end{enumerate}

\bigskip

\centering {\Large Абстрактный синтаксис языка L }
$$
X \mbox{ --- счетно-бесконечное множество переменных}
$$
$$
\otimes=\{\llang{+}, \llang{-}, \llang{*}, \llang{/}, \llang{\%}, \llang{==}, \llang{!=}, 
\llang{>}, \llang{>=}, \llang{<}, \llang{<=}, \llang{\&\&}, \llang{||}\}
$$

\begin{itemize}
\item Выражения: $\fancy{E}=X\cup\NN\cup(\fancy{E}\otimes\fancy{E})$. В выражениях могут использоваться круглые скобки.
\item Операторы: 

$$
\begin{array}{rll}
  \fancy{S}=&\llang{skip}&\cup\\
            &X\;\llang{:=}\;\fancy{E}&\cup\\
            &\fancy{S}\;\llang{;}\;\fancy{S}&\cup\\
            &\llang{write}\;\fancy{E}&\cup\\
            &\llang{read}\;X&\cup\\
            &\llang{while}\;\fancy{E}\;\llang{do}\;\fancy{S}&\cup\\
            &\llang{if}\;\fancy{E}\;\llang{then}\;\fancy{S}\;\llang{else}\;\fancy{S}
\end{array}
$$
\item Программы: $\fancy{P}=\fancy{S}$
\end{itemize}

Пример программы: 
$$
\llang{read}\; x \llang{;} \; \llang{if} \; y + 1 == x  \; \llang{then} \; \llang{write} \; y \; \llang{else} \; \llang{skip} 
$$

Результат лексического анализа должен быть в духе: 

\begin{verbatim}
KW_Read(0, 0, 3); Var("x", 0, 5, 5); Colon(0, 6, 6); KW_If(0, 8, 9); Var("y", 0, 11, 11);
Op(Plus, 0, 13, 13); Num(1, 0, 15, 15); Op(Eq, 0, 17, 18); Var("x", 0, 20, 20); KW_Then(0, 22, 25); 
KW_Write(0, 27, 31); Var("y", 0, 33, 33); KW_Else(0, 35, 38); KW_Skip(0, 40, 43)
\end{verbatim}

\end{document}

