\documentclass{article}

\usepackage[left=2cm,right=2cm,top=2cm,bottom=2cm,bindingoffset=0cm]{geometry}
\usepackage{listings}
\usepackage{indentfirst}
\usepackage{verbatim}
\usepackage{amsmath, amsthm, amssymb}
\usepackage{stmaryrd}
\usepackage{graphicx}
\usepackage{euscript}

\usepackage[utf8]{inputenc}
\usepackage[english,russian]{babel}
\usepackage[T2A]{fontenc}

\lstdefinelanguage{llang}{
keywords={skip, do, while, read, write, if, then, else},
sensitive=true,
%%basicstyle=\small,
commentstyle=\scriptsize\rmfamily,
keywordstyle=\ttfamily\underbar,
identifierstyle=\ttfamily,
basewidth={0.5em,0.5em},
columns=fixed,
fontadjust=true,
literate={->}{{$\to$}}1
}

\lstset{
language=llang
}

\newcommand{\aset}[1]{\left\{{#1}\right\}}
\newcommand{\term}[1]{\mbox{\texttt{\bf{#1}}}}
\newcommand{\cd}[1]{\mbox{\texttt{#1}}}
\newcommand{\sembr}[1]{\llbracket{#1}\rrbracket}
\newcommand{\conf}[1]{\left<{#1}\right>}
\newcommand{\fancy}[1]{{\cal{#1}}}

\newcommand{\trule}[2]{\frac{#1}{#2}}
\newcommand{\crule}[3]{\frac{#1}{#2},\;{#3}}
\newcommand{\withenv}[2]{{#1}\vdash{#2}}
\newcommand{\trans}[3]{{#1}\xrightarrow{#2}{#3}}
\newcommand{\ctrans}[4]{{#1}\xrightarrow{#2}{#3},\;{#4}}
\newcommand{\llang}[1]{\mbox{\lstinline[mathescape]|#1|}}
\newcommand{\pair}[2]{\inbr{{#1}\mid{#2}}}
\newcommand{\inbr}[1]{\left<{#1}\right>}
\newcommand{\highlight}[1]{\color{red}{#1}}
\newcommand{\ruleno}[1]{\eqno[\textsc{#1}]}
\newcommand{\inmath}[1]{\mbox{$#1$}}
\newcommand{\lfp}[1]{fix_{#1}}
\newcommand{\gfp}[1]{Fix_{#1}}

\newcommand{\NN}{\mathbb N}
\newcommand{\ZZ}{\mathbb Z}

\begin{document}

%% Весь этот текст можно удалить
%% ====== от сих =====
\centering {\LARGE Формальные языки}

{\Large домашнее задание до 23:59 07.11}
\bigskip

\begin{enumerate}
  \item Осуществить фильтрацию комментариев из потока лексем, выдаваемых лексером языка L (4 балла)
  \begin{itemize}
      \item Результатом работы \underline{лексера} должен быть поток лексем --- не просто печать в консоль
      \item Консольное приложение для лексера должно выводить лексемы на печать, но печатать должна отдельная процедура
      \item Консольное приложение для лексера должно иметь возможность быть запущенным в двух режимах: с фильтрацией комментариев и без фильтрации (добавить ключ -filter)
      \item Консольное приложение \textbf{должно} принимать имя входного файла как параметр: если ваше приложение только \underline{после} начала работы спрашивает имя входного файла, поправьте, пожалуйста
  \end{itemize}
  \item Написать парсер для языка L (описание языка ниже), используя любимый способ писать парсеры. Не забыть про тесты (8 баллов)
    \begin{itemize}
        \item Можно использовать генераторы синтаксических анализаторов (yacc, bison, antlr, любой другой)
        \item Можно использовать написанный вами CYK
        \item Ваш синтаксический анализатор должен принимать на вход то, что выдает ваш лексер --- те самые токены с позициями во входной строке; если для этого нужно править лексер --- правьте
        \item Создайте консольное приложение для запуска синтаксического анализа
        \begin{itemize}
            \item Консольное приложение обязательно должно принимать адрес файла со входной программой
            \item Программа может быть многострочной
            \item Консольное приложение должно запускать сначала лексер, потом парсер
            \item Результатом синтаксического анализа должно быть дерево вывода --- напечатанное в человекочитаемом формате; можно в файл, но название файла должно быть связано с названием входного файла
            \item Лексемы в дереве вывода должны отображаться со всей необходимой информацией
            \item Код должен быть размещен на гитхабе, собираться одним скриптом, содержать инструкцию по сборке и запуску собранного приложения, собираться на чистой Ubuntu 15.04 или Windows 8. Все зависимости, в случае их отсутствия в системе, должны либо доставляться скриптом, либо перечисляться в явном виде.
        \end{itemize}
     \end{itemize}
\end{enumerate}

\bigskip

\centering {\Large Абстрактный синтаксис языка L }
$$
X \mbox{ --- счетно-бесконечное множество переменных}
$$
$$
\otimes=\{\llang{+}, \llang{-}, \llang{*}, \llang{/}, \llang{\%}, \llang{==}, \llang{!=}, 
\llang{>}, \llang{>=}, \llang{<}, \llang{<=}, \llang{\&\&}, \llang{||}\}
$$

\begin{itemize}
\item Выражения: $\fancy{E}=X\cup\NN\cup(\fancy{E}\otimes\fancy{E})$. В выражениях могут использоваться круглые скобки.
\item Операторы: 

$$
\begin{array}{rll}
  \fancy{S}=&\llang{skip}&\cup\\
            &X\;\llang{:=}\;\fancy{E}&\cup\\
            &\fancy{S}\;\llang{;}\;\fancy{S}&\cup\\
            &\llang{write}\;\fancy{E}&\cup\\
            &\llang{read}\;\fancy{X}&\cup\\
            &\llang{while}\;\fancy{E}\;\llang{do}\;\fancy{S}&\cup\\
            &\llang{if}\;\fancy{E}\;\llang{then}\;\fancy{S}\;\llang{else}\;\fancy{S}
\end{array}
$$
\item Программы: $\fancy{P}=\fancy{S}$
\end{itemize}

Пример программы: 
$$
\llang{read}\; x \llang{;} \; \llang{if} \; y + 1 == x  \; \llang{then} \; \llang{write} \; y \; \llang{else} \; (* nothing \, to \, do \, here *) \, \llang{skip} 
$$


\end{document}

