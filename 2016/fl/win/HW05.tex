\documentclass[12pt]{article}
\usepackage[left=2cm,right=2cm,top=2cm,bottom=2cm,bindingoffset=0cm]{geometry}
\usepackage{hyperref}
\usepackage{fontspec}
\usepackage{polyglossia}
\setdefaultlanguage{russian}
\setmainfont[Mapping=tex-text]{CMU Serif}

\begin{document}
%% Весь этот текст можно удалить
%% ====== от сих =====
\centering {\LARGE Формальные языки}

{\Large домашнее задание до 23:59 10.10}
\bigskip

\begin{enumerate}
  \item Задача 2 из предыдущего задания (4 балла)  
  \item Расширить лексер возможностью обработки комментариев. Комментарии могут содержать произвольный текст, весь комментарий --- одна лексема. Добавить тесты. 
  \begin{itemize}
      \item Однострочные в формате \verb!//comment! (1 балл)
      \item Многострочные в формате \verb!(*comment*)! (1 балл)
  \end{itemize}
  \item Добавить в язык L операцию возведения в степень \verb!**!, не забыть про тесты (1 балл)
  \begin{itemize}
      \item Обратите внимание на разные значения символа \verb!*! в нашем языке
  \end{itemize}
    
  \item Реализовать преобразование КС-грамматики в нормальную форму Хомского. Требования к оформлению задачи такие же, как для лексера из 4 домашнего задания: консольное приложение, принимающее как аргумент путь к файлу (с описанием грамматики), код и инструкция по запуску на гитхабе, скрипты сборки и запуска тестов.
  \begin{itemize}
      \item Каждый шаг преобразования оформлять в отдельную функцию (по 1 баллу за каждый шаг)
      \item Результат выводите в человекочитаемом виде
      \item Тесты должны включать три грамматики, которые нормализовывались на паре: однозначная и неоднозначная грамматика правильных скобочных последовательностей, однозначная грамматика арифметических выражений (1 балл)
  \end{itemize}
  \item Реализовать алгоритм синтаксического анализа CYK. Требования к оформлению задачи такие же, как для лексера из 4 домашнего задания: консольное приложение, принимающее как аргумент путь к файлу (с описанием грамматики и входной последовательностью), код и инструкция по запуску на гитхабе, скрипты сборки и запуска тестов. (6 баллов)
  \begin{itemize}
      \item Результатом является ответ на вопрос, принадлежит ли цепочка языку, задаваемому грамматикой, а также таблица, которая строится алгоритмом (выводить в человекочитаемом виде)
      \item Алгоритм принимает на вход грамматику в НФХ, нормализацию в этой задаче проводить не нужно. Однако если вы подключите модуль нормализации из задачи 2, вам будет 1 балл бонуса. 
      \item Обязательно хорошее тестовое покрытие. 
  \end{itemize}
  
\end{enumerate}

%% ===== и до сих =====
\end{document}
