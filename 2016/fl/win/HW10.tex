\documentclass[12pt]{article}
\usepackage[left=2cm,right=2cm,top=2cm,bottom=2cm,bindingoffset=0cm]{geometry}
\usepackage{fontspec}
\usepackage{polyglossia}
\setdefaultlanguage{russian}
\setmainfont[Mapping=tex-text]{CMU Serif}

\begin{document}
\centering {\LARGE Формальные языки}

{\Large домашнее задание до 23:59 14.11}
\bigskip

\enumerate
{
  \item
  
    Для грамматики скобочных последовательностей $ S \rightarrow  S ( S ) \, | \, \varepsilon$:
    
    \begin{enumerate}
        \item Построить LR(0) автомат и LR(0) таблицу.
        \item Если не удалось, построить SLR(1) таблицу для той же грамматики.
        \item Если не удалось, построить LALR(1) автомат и таблицу для той же грамматики.
        \item Если не удалось, подумать и написать, почему так вышло. 
        \item Если какую-нибудь таблицу построить все-таки удалось, промоделировать с ней разбор строк "()(())" и "(()": предоставить историю изменения стека и дерево разбора.
    \end{enumerate} 
   \item Для грамматики скобочных последовательностей $ S \rightarrow  ( S ) \, | \, S S  \, | \, \varepsilon$:
    
    \begin{enumerate}
        \item Построить LR(0) автомат и LR(0) таблицу.
        \item Если не удалось, построить SLR(1) таблицу для той же грамматики.
        \item Если не удалось, построить LALR(1) автомат и таблицу для той же грамматики.
        \item Если не удалось, подумать и написать, почему так вышло. 
        \item Если какую-нибудь таблицу построить все-таки удалось, промоделировать с ней разбор строк "()(())" и "(()": предоставить историю изменения стека и дерево разбора.
    \end{enumerate} 
}

  Автомат можно не рисовать в виде графа: достаточно указать, из каких LR-item-ов состоят состояния, и предоставить таблицу переходов между состояниями. 
\end{document}