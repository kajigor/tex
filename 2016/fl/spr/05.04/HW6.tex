\documentclass[12pt]{article}
\usepackage[left=2cm,right=2cm,top=2cm,bottom=2cm,bindingoffset=0cm]{geometry}
\usepackage{hyperref}
\usepackage{fontspec}
\usepackage{polyglossia}
\setdefaultlanguage{russian}
\setmainfont[Mapping=tex-text]{CMU Serif}

\begin{document}
%% Весь этот текст можно удалить
%% ====== от сих =====
\centering {\LARGE Формальные языки}

{\Large домашнее задание до 23:59 11.04}
\bigskip

\begin{enumerate}
  \item 
  {  Написать синтаксический анализатор для языка L, используя любимую библиотеку парсер-комбинаторов (\url{https://en.wikipedia.org/wiki/Parser_combinator} --- в разделе External Links есть примеры библиотек; можно взять любую другую хорошую). (8 баллов за полностью выполненное задание) 
  \begin{itemize}
    \item Вход --- обычная строка, функции лексического анализа должны быть зашиты в парсер. Количество пробелов между лексемами не должно быть существенным. 
    \item Результат работы парсера --- дерево разбора строки. 
    \item Предоставить набор тестов, демонстрирующий работоспособность парсера. Деревья должны визуализироваться; чем проще их прочитать, тем лучше.
    \item Как и в предыдущих заданиях, все это должно быть приложением, ожидающим на вход строку.
  \end{itemize}
  }
  \item
  {
    Сделать pretty printer: функцию, которая, получая на вход дерево разбора, возвращает (красиво отформатированную) строчку с программой. (2 балла)
  }
  \item 
  {
    Провести оптимизацию арифметических и логических выражений: по построенному парсером дереву разбора построить новое, в котором нет лишних операций. Лишние операции это, например, домножения на 1, сложения с 0, домножения на 0, конъюнкция, где первый элемент --- ложь и т.д. Чем больше оптимизаций будет реализовано, тем лучше. 
    \begin{itemize}
      \item Количество баллов зависит от количества оптимизаций, поэтому желательно явно написать, какие оптимизации производятся.
      \item Результат оптимизаций --- дерево разбора. 
      \item Составить набор юнит-тестов, демонстрирующих, что каждая оптимизация работает, как задумано. 
      \item Добавить к вашему приложению возможность запуска анализатора в оптимизирующем режиме. Программа должна принимать на вход программу, визуализировать оптимизированное дерево и печатать оптимизированную программу (с помощью принтера из задания 2).
    \end{itemize}
  }
\end{enumerate}

%% ===== и до сих =====
\end{document}
