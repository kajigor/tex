\documentclass[final,acmlarge,draft=false,review=false,nonacm=false]{acmart}
\usepackage[
    type={CC},           % your choice
    modifier={by-sa},    % your choice
    version={4.0},       % your choice
]{doclicense}            % your choice, see \doclicenseThis below

\settopmatter{printacmref=false,printfolios=false}
\fancyfoot{}

\makeatletter
\def\@formatdoi#1{}
\def\@permissionCodeOne{miniKanren.org/workshop}
\def\@copyrightpermission{\doclicenseThis} % your choice of text
\def\@copyrightowner{Copyright held by the author(s).} % your choice
\makeatother

\copyrightyear{2020}
\setcopyright{rightsretained}

% \acmConference[miniKanren 2020]{The Second miniKanren and Relational Programming Workshop}{August 27 2020}{Online}

\acmMonth{8}
\acmArticle{9} % your number in the order of presentations (between 1 and 11)

\usepackage{xspace}
\usepackage{listings}
\usepackage{subcaption}
\usepackage{multirow}
\usepackage{multicol}

\newcommand{\todo}[1]{{\color{red}{#1}}}

\newcommand{\mk}{\textsc{miniKanren}\xspace}
\newcommand{\micro}{\textsc{microKanren}\xspace}
\newcommand{\merc}{\textsc{Mercury}\xspace}
\newcommand{\curry}{\textsc{curry}\xspace}
\newcommand{\haskell}{\textsc{Haskell}\xspace}
\newcommand{\ocaml}{\textsc{OCaml}\xspace}
\newcommand{\ocanren}{\textsc{OCanren}\xspace}
\newcommand{\prolog}{\textsc{Prolog}\xspace}
\newcommand{\inm}{\emph{in}\xspace}
\newcommand{\outm}{\emph{out}\xspace}

\sloppy

\newcommand{\db}[1]{{\color{red}#1}}

\graphicspath{{figures/}}

\begin{document}

\title{An Empirical Study of Partial Deduction for \mk{}}

\titlenote{The reported study was funded by RFBR, project number 18-01-00380}

%%
%% The "author" command and its associated commands are used to define
%% the authors and their affiliations.
%% Of note is the shared affiliation of the first two authors, and the
%% "authornote" and "authornotemark" commands
%% used to denote shared contribution to the research.
\author{Ekaterina Verbitskaia}
\email{kajigor@gmail.com}

\author{Daniil Berezun}
\email{daniil.berezun@jetbrains.com}

\author{Dmitry Boulytchev}
\email{dboulytchev@math.spbu.ru}


\affiliation{
  \institution{Saint Petersburg State University}
  \country{Russia}
}
\affiliation{
  \institution{JetBrains Research}
  \country{Russia}
}


%%
%% By default, the full list of authors will be used in the page
%% headers. Often, this list is too long, and will overlap
%% other information printed in the page headers. This command allows
%% the author to define a more concise list
%% of authors' names for this purpose.
\renewcommand{\shortauthors}{Verbitskaia, Berezun and Boulytchev.}

%%
%% The abstract is a short summary of the work to be presented in the
%% article.
\begin{abstract}
  Relational programming is known for its capability to provide a short and concise executable specifications
  for a wide range of interesting problems. Specifically, the nature of relational programming makes it possible
  to consider a single specification as a whole family of concrete programs. Individual programs of this family
  can be taken and run by placing free variables inside a top-level goal arguments. In particular, relational
  programming provides a very generic way to implement \emph{program inversion}, which opens a way for
  program synthesis via converting \emph{verifiers} into \emph{solvers}. However, acquired in such a way
  solvers often come with an overhead, originating from the very nature of relational computations
  with substitutions, unifications, interleaving, etc. In this paper we study a conversion of relational programs
  into functional form taking into account a concrete \emph{direction} of evaluation. The project is at an early stage,
  but the resulats so far are promising: converted functions run much faster than the original relations.
\end{abstract}


%%
%% The code below is generated by the tool at http://dl.acm.org/ccs.cfm.
%% Please copy and paste the code instead of the example below.
%%
\begin{CCSXML}
  <ccs2012>
  <concept>
  <concept_id>10011007.10011006.10011008.10011009.10011015</concept_id>
  <concept_desc>Software and its engineering~Constraint and logic languages</concept_desc>
  <concept_significance>500</concept_significance>
  </concept>
  <concept>
  <concept_id>10011007.10011006.10011041.10011047</concept_id>
  <concept_desc>Software and its engineering~Source code generation</concept_desc>
  <concept_significance>500</concept_significance>
  </concept>
  </ccs2012>
\end{CCSXML}

\ccsdesc[500]{Software and its engineering~Constraint and logic languages}
\ccsdesc[500]{Software and its engineering~Source code generation}
%%
%% Keywords. The author(s) should pick words that accurately describe
%% the work being presented. Separate the keywords with commas.
\keywords{relational programming, partial deduction, specialization}

\maketitle
\thispagestyle{empty}

\section{Introduction}

Here is going to be an introduction.
\section{Related Work}

A mode generalizes the concept of a direction; this terminology is commonly used in the conventional logic programming community.
In its most primitive form, a mode specifies which arguments of a relation will be known at runtime (input) and which are expected to be computed (output).
Several logic programming languages have mode systems used for optimizations, with \merc\footnote{Website of the \merc programming language: \url{https://mercurylang.org/}} standing out among them.
\merc is a modern functional-logic programming language with a complicated mode system capable not only of describing directions, but also specifying if a relation in the given mode is deterministic, among other things~\cite{overton2002constraint}.

Functional logic programming languages such as \merc translate the logic subset of the language into a functional programming language much like we do.
\merc utilizes a strong system of modes and types to guide the compilation.
The first version of mode analysis was based on the abstract interpretation~\cite{somogyi1987system}.
It is still the one used in the implementation.
They also researched the use of constraint systems for mode inference~\cite{overton2002constraint}.
While being more precise, this system proved to be too slow to be used in the compiler.

Mercury's mode system is tight into a type system.
Many \mk languages are embedded into host languages which are not typed and thus we cannot rely on type information in our conversion.

It is hard to compare our implementation with \merc because of the difference in semantics.
\merc uses a system of prescriptive modes which means that the semantics is defined by the mode assigned to a program.
We use modes in a prescriptive fashion, and the semantics of a \mk program is the same set of answers regardless of the order of subgoals.
This allows us to compare runtime of differently modded relational programs.

Even though using abstract interpretation is a standard for doing many kinds of program analysis~\cite{seminal paper}, we decided to abstain from using it for mode analysis.
This is done in order to simplify the implementation of the functional conversion in the spirit of \mk language family.



\label{sec:mode}
\subsection{Modes}


Given an annotation for a relation, mode inference determines modes of each variable of the relation.
For some modes, conjunctions in the body of a relation may need reordering to ensure that consumers of computed values come after the producers of said values so that a variable is never used before it is bound to some value.
In this project, we employed the least complicated mode system, in which variables may only have an \inm or \outm mode.
A mode maps variables of a relation to a pair of the initial and final instantiations.
The mode \inm stands for $g \rightarrow g$, while \outm stands for $f \rightarrow g$.
The instantiation $f$ represents an unbound, or \emph{free}, variable, when no information about its possible values is available.
When the variable is known to be \emph{ground}, its instantiation is $g$.

In this paper, we call a pair of instantiations a mode of a variable.
figure~\ref{fig:double_modded} shows examples of the normalized \mk relations with modes inferred for the forward and backward directions.
We use superscript annotation for variables to represent their modes visually.
Notice the different order of conjuncts in the bodies of the \lstinline{add$^o$} relation in different modes.

\begin{figure}[h]
  \centering
  \begin{subfigure}[b]{0.45\textwidth}
    \begin{lstlisting}[frame=tb]
  let double$^o$ x$^{g \rightarrow g}$ r$^{f \rightarrow g}$ = 
    addo$^o$ x1$^{g \rightarrow g}$ x2$^{g \rightarrow g}$ r$^{f \rightarrow g}$ /\ 
    x1$^{g \rightarrow g}$ ===    x2$^{g \rightarrow g}$

  let rec add$^o$ x$^{g \rightarrow g}$ y$^{g \rightarrow g}$ z$^{f \rightarrow g}$ =
    (x$^{g \rightarrow g}$ ===    O /\ y$^{g \rightarrow g}$ ===    z$^{f \rightarrow g}$) \/
    (x$^{g \rightarrow g}$ ===    S x1$^{f \rightarrow g}$ /\
     add$^o$ x1$^{g \rightarrow g}$ y$^{g \rightarrow g}$ z1$^{f \rightarrow g}$ /\
     z$^{f \rightarrow g}$ ===    S z1$^{g \rightarrow g}$) ]
    \end{lstlisting}
   \caption{Forward direction}
    \label{fig:double_fwd}
  \end{subfigure}
  \hfill
  \begin{subfigure}[b]{0.45\textwidth}
    \begin{lstlisting}[frame=tb]
  let double$^o$ x$^{f \rightarrow g}$ r$^{g \rightarrow g}$ = 
    addo$^o$ x1$^{f \rightarrow g}$ x2$^{f \rightarrow g}$ r$^{g \rightarrow g}$ /\ 
    x1$^{g \rightarrow g}$ ===    x2$^{g \rightarrow g}$

  let rec add$^o$ x$^{f \rightarrow g}$ y$^{f \rightarrow g}$ z$^{g \rightarrow g}$ =
    (x$^{f \rightarrow g}$ ===    O /\ y$^{f \rightarrow g}$ ===    z$^{g \rightarrow g}$) \/
    (z$^{f \rightarrow g}$ ===    S z1$^{g \rightarrow g}$ /\
     add$^o$ x1$^{f \rightarrow g}$ y$^{f \rightarrow g}$ z1$^{g \rightarrow g}$ /\
     x$^{f \rightarrow g}$ ===    S x1$^{g \rightarrow g}$) ]
    \end{lstlisting}
    \caption{Backward direction}
    \label{fig:double_bw}
  \end{subfigure}
  \caption{Normalized doubling and addition relations with mode annotations}
  \label{fig:double_modded}
\end{figure}




\input{conspd.tex}
\section{Evaluation}

To evaluate our functional conversion scheme, we implemented the proposed algorithm in \haskell.
We compared execution time of several \ocanren relations in different directions against their functional counterparts in the \ocaml language.
Here we showcase two relational programs and their conversions.
The implementation of the functional conversion\footnote{The repository of the functional conversion project \url{https://github.com/kajigor/uKanren_transformations}} as well as the execution code\footnote{Evaluation code \url{https://github.com/kajigor/miniKanren-func}} can be found on Github.



\begin{figure}[!t]
  \centering
  \begin{minipage}{0.49\textwidth}
    \begin{lstlisting}
    
            eval$^o$ st fm u = 
              fresh (x y v w z) 
                (fm === Conj x y /\
                 eval$^o$ st x v /\
                 eval$^o$ st y w /\
                 and$^o$ v w u) \/
                 ... 
    \end{lstlisting}
  \end{minipage}
\end{figure}

% and$^o$ x y b = conde [
%   (x === True  /\ y === True  /\ b === True );
%   (x === False /\ y === True  /\ b === False);
%   (x === True  /\ y === False /\ b === False);
%   (x === False /\ y === False /\ b === False)]

% or$^o$ x y b = conde [
%   (x === True  /\ y === True  /\ b === True );
%   (x === False /\ y === True  /\ b === True );
%   (x === True  /\ y === False /\ b === True );
%   (x === False /\ y === False /\ b === False)]

% not$^o$ x b = [(x === True   /\ b === False);
%             (x === False /\ b === True )]

% elem$^o$ i st v =
%   fresh (h t i') conde [
%     (i === O /\ st === (v : t));
%     (i === S i' /\ st === (h : t) /\ elem$^o$ i' t v)]

\subsection{Evaluator of Propositional Formulas}

In this example, we converted a relational evaluator of propositional formulas: see figure~\ref{fig:prop}.
It evaluates a propositional formula \lstinline{fm} in the environment \lstinline{st} to get the result \lstinline{u}.
A formula is either a boolean literal, a numbered variable, a negation of another formula, a conjunction or a disjunction of two formulas.
Converting it in the direction when everything but the formula is \inm (see figure~\ref{fig:prop_modded}), allows one to synthesize formulas which can be evaluated to the given value.
The conversion of this relation does not involve any generators and is presented in figure~\ref{fig:prop_hsk}.

We ran an experiment to compare the execution time of the relational interpreter vs. its functional conversion.
In the experiment, we generated from $1000$ to $10000$ formulas which evaluate to true and contain up to $3$ variables with known values.
The results are presented in figure~\ref{fig:prop_time}.
The functional conversion improved execution time of the query about $2.5$ times from $724ms$ to $291ms$ for retrieving $10000$ formulas.

\begin{figure}[h]
  \centering
\includegraphics[width=0.49\textwidth]{fig/propIOI.pdf}
  \caption{Execution time of the evaluators of propositional formulas, \lstinline{eval [true; false; true] q true}}
  \label{fig:prop_time}
\end{figure}


\begin{figure*}[h]
  \centering
  \begin{subfigure}[b]{0.49\textwidth}
    \includegraphics[width=\textwidth]{fig/muloIIO.pdf}
    \caption{Multiplication: \lstinline{mulo n 10 q}}
    \label{fig:mulo_IIO}
  \end{subfigure}
  \hfill
  \begin{subfigure}[b]{0.49\textwidth}
    \includegraphics[width=1\textwidth]{fig/muloIOI.pdf}
    \caption{Division: \lstinline{mulo (n/10) q n}}
    \label{fig:mulo_IOI}
  \end{subfigure}

  \hfill

  \begin{subfigure}[b]{0.49\textwidth}
    \includegraphics[width=\textwidth]{fig/muloIOO.pdf}
    \caption{Generation: \lstinline{take n (mulo 10 q r)}}
    \label{fig:mulo_IOO}
  \end{subfigure}
  \hfill
  \begin{subfigure}[b]{0.45\textwidth}
    \begin{subfigure}[b]{\textwidth}
      \begin{lstlisting}[frame=tb]
let rec mul$^o$ x$^{g \rightarrow g}$ y$^{f \rightarrow g}$ z$^{f \rightarrow g}$ =
  (x$^{g \rightarrow g}$ ===    O /\ z$^{f \rightarrow g}$ ===    O) \/
  (x$^{g \rightarrow g}$ ===    S x$_1^{f \rightarrow g}$ /\
   add$^o$ y$^{f \rightarrow g}$ z$_1^{f \rightarrow g}$ z$^{f \rightarrow g}$ /\
   mul$^o$ x$_1^{g \rightarrow g}$ y$^{g \rightarrow g}$ z$_1^{g \rightarrow g}$ )
    \end{lstlisting}
      \caption{Inefficient mode}
      \label{fig:mult_modded_bad}
    \end{subfigure}
    \hfill

    \begin{subfigure}[b]{\textwidth}
      \begin{lstlisting}[frame=tb]
let rec mul$^o$ x$^{g \rightarrow g}$ y$^{f \rightarrow g}$ z$^{f \rightarrow g}$ =
  (x$^{g \rightarrow g}$ ===    O /\ z$^{f \rightarrow g}$ ===    O) \/
  (x$^{g \rightarrow g}$ ===    S x$_1^{f \rightarrow g}$) /\
   mul$^o$ x$_1^{g \rightarrow g}$ y$^{f \rightarrow g}$ z$_1^{f \rightarrow g}$ /\
   add$^o$ y$^{g \rightarrow g}$ z$_1^{g \rightarrow g}$ z$^{f \rightarrow g}$ )
    \end{lstlisting}
      \caption{Efficient mode}
      \label{fig:mult_modded_good}
    \end{subfigure}
  \end{subfigure}

  \caption{Execution times of the multiplication relation}
  \label{fig:mulo_time}
\end{figure*}


\subsection{Multiplication}

In this example, we converted the multiplication relation in several directions and compared them to the relational counterparts: see figure~\ref{fig:mulo_time}.
Functional conversion significantly reduced execution time in most directions.

In the forward direction, we run the query \lstinline{mul$^o$ n 10 q} with \lstinline{n} in the range from $100$ to $1000$, and the functional conversion was $2$ orders of magnitude faster: $927ms$ vs $9.4ms$ for the largest \lstinline{n}, see figure~\ref{fig:mulo_IIO}.
In the direction which serves as division, we run the query \lstinline{mul$^o$ (n/10) q n} with \lstinline{n} ranging from $100$ to $1000$.
Here, performance improved 3 orders of magnitude: from $24s$ to $0.17s$ for the largest \lstinline{n}, see figure~\ref{fig:mulo_IOI}.
Even more impressive was the backward direction \lstinline{mul$^o$ x$^{f \to g}$ y$^{f \to g}$ z$^{g \to g}$}.
Querying for all $16$ pairs of divisors of $1000$ (\lstinline{mul$^o$ q r 1000}) took \ocanren about $32.9s$, while the functional conversion succeeded in $1.1s$.

What was surprising was the mode \lstinline{mul$^o$ x$^{g \to g}$ y$^{f \to g}$ z$^{f \to g}$}.
In this case, the functional conversion was not only worse than its relational counterpart, its performance degraded exponentially with the number of answers asked.
It took almost $1450ms$ to find the first $7$ pairs of numbers \lstinline{q} and \lstinline{r} such that \lstinline{10 * q = r}, while \ocanren was able to execute the query in $0.74ms$ (see figure~\ref{fig:mulo_IOO}).
The source of this terrible behavior was the suboptimal order of the calls in the second disjunct of the \lstinline{mul$^o$} relation in the corresponding mode (see figure~\ref{fig:mult_modded_bad}).
In this case, the call \lstinline{add$^o$ y$^{f \to g}$ z$_1^{f \to g}$ z$^{f \to g}$} is put first, which generates all possible triples in the addition relation before filtering them by the call \lstinline{mul$^o$ x$_1^{g \to g}$ y$^{g \to g}$ z$_1^{g \to g}$}.
The other order of calls is much better (see figure~\ref{fig:mult_modded_good}): it is an order of magnitude faster than its relational source.
To achieve the better of these two orders automatically, we  delay picking any call with all arguments free.
A call of this kind always works as a generator of every tuple of values which are in relation.
It is a reasonable heuristics to postpone their execution until its arguments become more instantiated.

\subsection{Relational Sorting}

This is a truly relational program.
It is derived from the definition of a sorted list.
It can be used for both sorting and generating permutations.
There is no way to implement the program in such a way that it works in both directions with decent performance.
Mode analysis is capable of reordering the conjuncts in each of the directions and thus make it runnable.


\subsection{Deterministic Directions}

Running in some directions, relations produce deterministic results.
For example, any forward direction of a relation created by the relational conversion produces a single result, since it mimics the original function.
The guard directions are semi-deterministic: they may fail, but if they succeed, they produce a single unit value.
If the addition relation is run with one of the first two arguments \outm, it acts as subtraction and is also deterministic.

For such directions, there is no need to model nondeterminism with the Stream monad.
Semi-determinism can be expressed with a Maybe monad, while deterministic directions can be converted into simple functions.
Our implementation of functional conversion only restricts the computations to be monadic, it does not specify which monad to use.
By picking other monads, we can achieve performance improvement.
For example, using Maybe for division reduces its execution time $30$ times in addition to the 2 orders of magnitude improvement from the functional conversion itself: see figure~\ref{fig:maybe}

\begin{center}
  \begin{tikzpicture}[
    every node/.style = {
      shape=rectangle,
      rounded corners=0.05cm,
      draw,
      align=center,
      minimum size=5mm},
    node distance=1.3em,
    anchor=center
  ]
    \node[inner sep=10pt,draw=none] (code) at (0,0) {
      \begin{minipage}{\columnwidth}
      \begin{lstlisting}[]
 ~\texttt{muloOII $::$ Nat $\to$ Nat $\to$ Maybe Nat}~
 ~\lststrikethrough{muloOII $::$ Nat $\to$ Nat $\to$ Stream Nat}~
 muloOII x1 x2 =
     zero <$\mid$> positive
   where
     zero = do
       guard (x2 == O)
       return O
     positive = do
       x4 <- addoIOI x1 x2
       S <$\$$> muloOII x1 x4
       \end{lstlisting}
      \end{minipage}
    };
    \node [draw=none, above=of code.south, xshift=2.5cm, yshift=2.5cm] {\goodBadge{10x \linebreak faster}};
  \end{tikzpicture}
  \end{center}
  
\section{Conclusion and Future Work}

In this paper we described a semi-automatic functional conversion from a \mk relation with a fixed direction into a functional language.

As part of the future work, we plan to employ a determinism check in the mode analysis.
We also plan to integrate the functional conversion with specialization techniques such as partial deduction.

\bibliographystyle{ACM-Reference-Format.bst}
\bibliography{bibl.bib}

\end{document}
\endinput
