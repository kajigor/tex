\documentclass{beamer}
\usepackage{beamerthemesplit}
\usepackage{wrapfig}
\usetheme{SPbGU}
\usepackage{pdfpages}
\usepackage{amsmath}
\usepackage{mathtools}
\usepackage{cmap}
\usepackage[T2A]{fontenc}
\usepackage[utf8]{inputenc}
\usepackage[english,russian]{babel}
\usepackage{indentfirst}
\usepackage{amsmath}
\usepackage{tikz}
\usepackage{multirow}
\usepackage[noend]{algpseudocode}
\usepackage{algorithm}
\usepackage{algorithmicx}
\usepackage{ stmaryrd }
\usepackage{qtree}
\usetikzlibrary{shapes,arrows}
\usepackage{fancyvrb}
\usepackage{minted}
\usepackage{forest}
\usepackage{tikz-qtree}
\newtheorem{rutheorem}{Теорема}
\newtheorem{ruproof}{Доказательство}
\newtheorem{rudefinition}{Определение}
\newtheorem{rulemma}{Лемма}
\beamertemplatenavigationsymbolsempty

\setbeamertemplate{itemize item}[circle]
\setbeamertemplate{enumerate item}[circle]

\newcommand{\derive}[0]{\xRightarrow[]{*}}
\newcommand{\derives}[0]{\xRightarrow[]{}}
\newcommand{\derivek}[1]{\xRightarrow[]{#1}}
\newcommand{\deriveg}[1]{\xRightarrow[#1]{*}}
\newcommand{\derivegone}[1]{\xRightarrow[#1]{}}

\title[]{Теория автоматов и формальных языков}
\subtitle[]{Синтаксический анализ}
\institute[]{
Санкт-Петербургский государственный электротехнический университет <<ЛЭТИ>>\\
}

\author[]{Екатерина Вербицкая}

\date{4 декабря 2020}

\definecolor{orange}{RGB}{179,36,31}

\begin{document}
{
  \begin{frame}
    \titlepage
  \end{frame}
}

\begin{frame}[fragile]
  \frametitle{Синтаксический анализ}
\begin{center}
  Задачи синтаксического анализа: по последовательности символов
\end{center}

  \begin{itemize}
    \item Определить, корректна ли она с точки зрения грамматики языка
    \begin{itemize}
      \item Если нет, сообщить об ошибке
    \end{itemize}
    \item Построить для нее некоторое структурное представление
    \begin{itemize}
      \item Дерево вывода
      \item Абстрактное синтаксическое дерево
      \item Вычислить семантику (реже)
    \end{itemize}
  \end{itemize}
\end{frame}

\begin{frame}[fragile]
  \frametitle{Безлексерный синтаксический анализ}
\begin{center}
  Терминалами грамматики являются символы строки
\end{center}

\begin{align*}
  E &\to E + T \mid T  \\
  T &\to T * F \mid F  \\
  F &\to (E) \mid N \mid Id \\
  N &\to 0 \mid 1 \mid \dots \mid 9 \mid 0N \mid 1N \mid \dots \mid 9N \\
  Id &\to a \mid b \mid \dots \mid z \mid a \, Id \mid \dots \mid z \, Id
\end{align*}
\end{frame}

\begin{frame}[fragile]
  \frametitle{Недостатки безлексерного синтаксического анализа: грамматики очень быстро перестают быть читаемыми}

  \begin{align*}
    E &\to Sp \, E \, Sp + Sp \, T \, Sp \mid Sp \, T \, Sp \\
    T &\to Sp \, T \, Sp * Sp \, F \, Sp \mid Sp \, F \, Sp  \\
    F &\to Sp \, ( Sp \, E \, Sp) \, Sp \mid Sp \, N \, Sp \, \mid Sp \, Id \, Sp \\
    N &\to 0 \mid 1 \mid \dots \mid 9 \mid 1N \mid 2N \mid \dots \mid 9N \\
    Id &\to a \mid b \mid \dots \mid z \mid a \, Id \mid \dots \mid z \, Id \\
    Sp &\to \varepsilon \mid \text{\textvisiblespace} \, Sp
  \end{align*}
\end{frame}

\begin{frame}[fragile]
  \frametitle{Недостатки безлексерного синтаксического анализа: работает медленно}

  Большую часть времени синтаксический анализатор тратит на разбор регулярных фрагментов языка, что чревато накладными расходами

  \begin{align*}
    N &\to 0 \mid 1 \mid \dots \mid 9 \mid 1N \mid 2N \mid \dots \mid 9N \\
    Id &\to a \mid b \mid \dots \mid z \mid a \, Id \mid \dots \mid z \, Id
  \end{align*}
\end{frame}

\begin{frame}[fragile]
  \frametitle{Лексический анализ}
\begin{center}
  Преобразование последовательности символов в

  последовательность \emph{лексем} (или \emph{токенов})
\end{center}

\begin{center}
  Лексема --- минимальная смысловая единица языка
\end{center}

\bigskip

\verb!"(123 + 45) * 6"! $\to$ \verb!LBR, NUM 123, PLUS, NUM 45, RBR, MULT, NUM 6!

\end{frame}

\begin{frame}[fragile]
  \frametitle{Типы лексем}
  \begin{itemize}
    \item Ключевое слово
    \item Число
    \begin{itemize}
      \item Целое
      \item Целое в двоичной системе
      \item С плавающей точкой
      \item $\dots$
    \end{itemize}
    \item Идентификатор
    \item Оператор
  \end{itemize}
\end{frame}

\begin{frame}[fragile]
  \frametitle{Лексический анализ и регулярные языки}

  Часто языки разных типов лексем являются регулярными языками

  \bigskip

  Лексеры часто реализуются при помощи регулярных выражений

  \bigskip

  \verb!NUM :: 0 | [1-9][0-9]*!

  \verb!IDENT :: [a-z_][a-z_0-9]*!

  \bigskip

  Лексеры также часто пропускают пробельные символы и выделяют комментарии
\end{frame}


\begin{frame}[fragile]
  \frametitle{Синтаксический анализ после лексического}

  \begin{center}
    Терминалы грамматики --- лексемы
  \end{center}

  \begin{align*}
    E &\to E \, PLUS \, T \mid T  \\
    T &\to T \, MULT \, F \mid F  \\
    F &\to LBR \, E \, RBR \mid NUM \mid IDENT \\
  \end{align*}

\end{frame}

\begin{frame}[fragile]
  \frametitle{Конкретный синтаксис vs абстрактный синтаксис}
\begin{columns}
  \begin{column}{0.2\textwidth}
    \begin{center}
      \begin{tikzpicture}
        \tikzset{edge from parent/.style={draw,edge from parent path={(\tikzparentnode.south)-- +(0,-1pt)-- (\tikzchildnode)}, node distance=2pt}, scale=0.7, every node/.style={scale=0.7}}
        \Tree [.E [.T
            [.T [.F
            [.( ]
            [.E
              [.E [.T [.F [.N [.123 ] ] ] ] ]
              [.+ ]
              [.T [.F [.N [.45 ] ] ] ]
            ]
            [.) ]
          ]]
            [.* ]
            [.F [.N [.6 ] ] ]
          ]
        ]
        \end{tikzpicture}
    \end{center}
  \end{column}
  \begin{column}{0.4\textwidth}
    \begin{align*}
      E &\to E \, PLUS \, T \mid T  \\
      T &\to T \, MULT \, F \mid F  \\
      F &\to LBR \, E \, RBR \mid N \\
    \end{align*}
  \end{column}
  \begin{column}{0.4\textwidth}
    \begin{align*}
      e :: e + e \mid e * e \mid num
    \end{align*}

    \begin{center}
      \begin{tikzpicture}
        \tikzset{edge from parent/.style={draw,edge from parent path={(\tikzparentnode.south)-- +(0,-1pt)-- (\tikzchildnode)}, node distance=2pt}, scale=0.9, every node/.style={scale=0.9}}
        \Tree [.* [.+ [.123 ] [.45 ] ] [.6 ] ]
        \end{tikzpicture}
    \end{center}
  \end{column}
\end{columns}
\end{frame}

\end{document}
