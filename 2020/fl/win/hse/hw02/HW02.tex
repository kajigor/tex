\documentclass[12pt]{article}
\usepackage[left=2cm,right=2cm,top=2cm,bottom=2cm,bindingoffset=0cm]{geometry}
\usepackage[utf8x]{inputenc}
\usepackage[english,russian]{babel}
\usepackage{cmap}
\usepackage{amssymb}
\usepackage{amsmath}
\usepackage{url}
\usepackage{pifont}
\usepackage{tikz}
\usepackage{verbatim}
\usepackage{hyperref}

\usetikzlibrary{shapes,arrows}
\usetikzlibrary{positioning,automata}
\tikzset{every state/.style={minimum size=0.2cm},
initial text={}
}

\begin{document}
\begin{center} {\LARGE Формальные языки} \end{center}

\begin{center} \Large домашнее задание до 23:59 17.09 \end{center}
\bigskip

\begin{enumerate}
  \item
  {
    Построить полные минимальные автоматы для языков $L_1, L_2, L_1 \cup L_2, L_1 \cap L_2, L_1 \setminus L_2$, где $L_1 = \{ \alpha 1 \beta \mid \alpha \in \{0\}^*, \beta \in \{0, 1\}^* \}$, $L_2 = \{ \alpha 0 \beta 1 \gamma \mid \alpha \in \{1\}^*, \beta \in \{0\}^*, \gamma \in \{0, 1\}^* \}$. Обосновать минимальность.
  }
  \item
  {
    Реализовать лексер для файлов \verb!.sig! и \verb!.mod!. Файлы в \href{https://github.com/kajigor/fl_2020_hse_win/tree/f8315a3bdfb721945ab26648b71e7cadad097833/prolog}{репозитории} должны успешно анализироваться.
    \begin{itemize}
      \item Программировать можно на любом языке, можно использовать любой инструмент для построения лексеров (например, что угодно из семейства lex).
      \item Результатом должно быть консольное приложение, которое принимает на вход программу и печатает результат лексического анализа в файл с таким же названием и дополнительным расширением \verb!.out!.
      \item Формат вывода: один токен на одной строке, сначала тип токена, потом его значение, номер строки и порядковый номер символа. Примеры лексем: \verb!NUM, 13, 1, 0!, \verb!LITERAL, "lit", 2, 1!.
      \item Значением лексемы-числа должно быть число.
      \item Слова \verb!module!, \verb!sig!, \verb!type! являются зарезервированными.
      \item Операторы языка: \verb!->!, \verb!:-!.
      \item Разделители: \verb!,!, \verb!.!, \verb![!, \verb!]!, \verb!|!.
      \item Идентификаторы могут содержать латинские буквы в верхнем и нижнем регистре, цифры и символ нижнего подчеркивания.
    \end{itemize}
  }
\end{enumerate}


\end{document}
