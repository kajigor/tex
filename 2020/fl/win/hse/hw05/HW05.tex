\documentclass[12pt]{article}
\usepackage[left=2cm,right=2cm,top=2cm,bottom=2cm,bindingoffset=0cm]{geometry}
\usepackage[utf8x]{inputenc}
\usepackage[english,russian]{babel}
\usepackage{cmap}
\usepackage{amssymb}
\usepackage{amsmath}
\usepackage{url}
\usepackage{pifont}
\usepackage{tikz}
\usepackage{verbatim}
\usepackage{hyperref}

\usetikzlibrary{shapes,arrows}
\usetikzlibrary{positioning,automata}
\tikzset{every state/.style={minimum size=0.2cm},
initial text={}
}

\begin{document}
\begin{center} {\LARGE Формальные языки} \end{center}

\begin{center} \Large домашнее задание до 23:59 08.10 \end{center}
\bigskip

\begin{enumerate}
  \item
  {
    Релизовать синтаксический анализатор с использованием генератора парсеров из семейства yacc для упрощенного подмножества пролога.
    \begin{itemize}
      \item Описание синтаксиса
      \begin{itemize}
        \item Программа на прологе --- последовательность \emph{определений отношений}.
        \item Определение отношений состоит из \emph{головы} и \emph{тела}, разделенных штопором (\verb!:-!), в конце стоит точка (\verb!.!).
        \item Голова --- атом.
        \item Тело --- выражение с правоассоциативными бинарными операторами конъюнкции (\verb!,!) и дизъюнкции (\verb!;!) над атомами. Конъюнкция имеет более высокий приоритет, чем дизъюнкция. Возможно использование скобок (\verb!(!, \verb!)!) для управления порядком вычислений.
        \item Атом --- идентификатор, за которым идет последовательность атомов, разделенных пробельными символами. Если в каком-то из атомов есть пробельный символ, то он должен быть взят в скобки (\verb!(!, \verb!)!).
        \item Тело и штопор могут отсутствовать.
        \item Где угодно могут встречаться пробельные символы: советую использовать лексер.
      \end{itemize}
      \item Примеры корректных определений отношений.
      \begin{itemize}
        \item \verb!f.!
        \item \verb!f :- g.!
        \item \verb!f :- g, h; t.!
        \item \verb!f :- g, (h; t).!
        \item \verb!f a :- g, h (t c d)!
        \item \verb!f (cons h t) :- g h, f t!
      \end{itemize}
      \item Примеры некорректных определений отношений.
      \begin{itemize}
        \item \verb!f! --- нет точки.
        \item \verb!:- f.! --- нет головы.
        \item \verb!f :- .! --- нет тела.
        \item \verb!f :- g; h, .! --- нет правого подвыражения у конъюнкции.
        \item \verb!f :- (g; (f).! --- несбалансированные скобки.
        \item \verb!f ().! --- пустые скобки
      \end{itemize}
      \item Результатом должно быть консольное приложение, которое принимает на вход программу и печатает результат синтаксического анализа в файл с таким же названием и дополнительным расширением \verb!.out!.
      \item Результатом синтаксического анализа является абстрактное синтаксическое дерево в случае успешного разбора и сообщение об ошибке иначе.
      \item Код должен быть сопровожден инструкцией по сборке и запуску. Желательно выложить его на гитхаб и сопроводить тестами.
      \item Пример парсера с пары выложен на гитхаб: \url{https://github.com/kajigor/fl-2020-hse-win/tree/3d6cae8eb08b862679de5e199e6ba4153d295d7e}
    \end{itemize}
  }
\end{enumerate}


\end{document}
