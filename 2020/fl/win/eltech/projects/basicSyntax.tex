\begin{center}
  {\Large Описание базового синтаксиса}
\end{center}
\begin{itemize}
  \item Лексический синтаксис
  \begin{itemize}
    \item Идентификатор --- непустая последовательность букв латинского алфавита в любом регистре, цифр и символа нижнего подчеркивания (\verb!_!), начинающаяся на букву латинского алфавита в нижнем регистре, не являющаяся ключевым словом.
    \begin{itemize}
      \item Корректные идентификаторы: \verb!x!, \verb!list!, \verb!listNat_123!.
      \item Некорректные идентификаторы: \verb!Abc!, \verb!123!, \verb!_List!.
    \end{itemize}
    \item Число: натуральное или ноль в десятичной системе счисления, не может содержать лидирующие нули.
    \begin{itemize}
      \item Корректные числа: \verb!123!, \verb!0!.
      \item Некорректные числа: \verb!-1!, \verb!007!, \verb!89A!.
    \end{itemize}
    \item Ключевые слова не могут быть идентификаторами. Конкретные ключевые слова вы выбираете сами.
    \item Операторы языка:
      \begin{itemize}
        \item сложение \verb!+!,
        \item умножение \verb!*!,
        \item деление \verb!/!,
        \item вычитание \verb!-!,
        \item возведение в степень \verb!**!,
        \item конъюнкция \verb!&&!,
        \item дизъюнкция \verb!||!,
        \item логическое отрицание \verb!--!,
        \item операторы сравнения: \verb!<!, \verb!<=!, \verb!==!, \verb!/=!, \verb!>!, \verb!>=!,
      \end{itemize}
    \item Пробелы не являются значимыми, но не могут встречаться внутри одной лексемы.
  \end{itemize}
  \item Базовый абстрактный синтаксис
  \begin{itemize}
    \item Программа --- непустая последовательность определений функций.
    \item Определение функции содержит ее сигнатуру и тело. Сигнатура функции содержит ее название (идентификатор) и список аргументов (может быть пустым). Тело~--- последовательность инструкций (может быть пустой).
    \item Инструкции:
    \begin{itemize}
      \item Присвоение значения арифметического выражения переменной. Переменная может быть произвольным идентификатором.
      \item Возвращение значения из функции.
      \item Условное выражение с обязательной веткой \verb!else!. Условием является арифметическое выражение. В ветках --- произвольные последовательности инструкций (могут быть пустыми).
      \item Цикл с предусловием. Условием является арифметическое выражение. Тело цикла --- произвольная последовательность инструкций (может быть пустой).
    \end{itemize}
    \item Арифметические выражения заданы над числами и идентификаторами, операторы перечислены в таблице ниже с указанием их приоритетов, арности и ассоциативности.

    \begin{center}
      \begin{tabular}{ c | c | c }
        Наибольший приоритет & Арность & Ассоциативность  \\ \hline \hline
        \verb!-! & Унарная & \\
        \verb!**! & Бинарная &Правоассоциативная \\
       \verb!*!,\verb!/! & Бинарная & Левоассоциативная \\
       \verb!+!,\verb!-! & Бинарная & Левоассоциативная \\
       \verb!==!,\verb!/=!, \verb!<!,\verb!<=!, \verb!>!,\verb!>=! & Бинарная & Неассоциативная \\
       \verb!--! & Унарная &  \\

       \verb!&&! & Бинарная & Правоассоциативная \\
       \verb!||! & Бинарная & Правоассоциативная \\
       \hline \hline
       Наименьший приоритет & Арность & Ассоциативность
      \end{tabular}
      \end{center}
  \end{itemize}
\end{itemize}