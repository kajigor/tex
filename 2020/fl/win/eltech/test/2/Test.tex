\documentclass[12pt]{article}
\usepackage[left=2cm,right=2cm,top=1cm,bottom=1cm,bindingoffset=0cm]{geometry}
\usepackage[utf8x]{inputenc}
\usepackage[english,russian]{babel}
\usepackage{cmap}
\usepackage{amssymb}
\usepackage{amsmath}
\usepackage{pifont}
\usepackage{tikz}
\usepackage{verbatim}
\usepackage{enumitem}
\usepackage{hyperref}

\pagenumbering{gobble}

\begin{document}

\begin{center}
{\LARGE Формальные языки}

{\Large Контрольная работа 2}

{\large 20.11.2020}
\end{center}

\bigskip

\begin{center}
  \LARGE Порядок проведения контрольной работы
\end{center}

\begin{itemize}
  \item Контрольная работа рассчитана на три часа: с 9 утра до 12 дня по Санкт-Петербургу.
  \item Результаты этой и следующей контрольной работы будут учитываться при выставлении оценки за курс. Хорошо написанные контрольные будут означать автомат по курсу. Если все контрольные вами написаны плохо, вы сможете в конце семестра сдать экзамен, повышающий оценку. Переписываний контрольной не будет.
  \item Контрольная работа должна выполняться каждым индивидуально. Если будет обнаружено списывание хотя бы одной задачи, вся контрольная работа будет не зачтена всем заподозренным в списывании вне зависимости от того, кто у кого списывал.
  \item Контрольную работу можно писать ручкой на листе бумаги. Если есть возможность отсканировать выполненную работу --- отсканируйте, иначе достаточно качественной фотографии. Нечитаемые работы проверяться не будут. Если есть планшет, можно использовать его. Если есть навык верстки в техе --- верстайте, но учитывайте ограничения по времени.
  \item Перед решением каждого задания обязательно укажите номер задачи. Обязательно убедитесь, что решаете положенный вам вариант, иначе контрольная не будет зачтена, даже если решена правильно. Вариант будет один на все задачи контрольной.
  \item Контрольная работа должна быть прислана на мою электронную почту не позднее \textbf{12:10 20.11.2020}. Присланные после этого момента контрольные проверяться не будут. Можно присылать по одной задаче, присланные задачи можно исправлять в новом письме, но не позднее  \textbf{12:10 20.11.2020}.
  \item Каждая присланная страница должна быть подписана вашими ФИО и номером группы.
  \item Обязательно присылать контрольную в письме с темой \textbf{[FL\_ElTech] Test 2}. Письма с любой другой темой будут игнорироваться.
  \item Любые соображения, которые привели вас к решению, целесообразно написать. Иногда студенты опечатываются в самом ответе, хотя все предыдущие шаги были выполнены правильно. Приведенные шаги помогут мне поверить, что это действительно опечатка, а не ошибка.
  \item Проверьте, что у грамматик явно указан стартовый нетерминал (один). Убедитесь, что если вас просят построить дерево вывода, вы строите дерево, и оно является деревом вывода. Убедитесь, что построенный вами левосторонний вывод является левосторонним, и что он является выводом.
  \item Прочитайте, что такое язык Дика (\url{https://bit.ly/2UH0hus}) и префиксная/постфиксная запись арифметических выражений (\url{https://bit.ly/3feP5ie}, \url{https://bit.ly/2UT9gcl}).
  \item Читайте задания предельно внимательно.
\end{itemize}


\end{document}
