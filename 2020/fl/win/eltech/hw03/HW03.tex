\documentclass[12pt]{article}
\usepackage[left=2cm,right=2cm,top=2cm,bottom=2cm,bindingoffset=0cm]{geometry}
\usepackage[utf8x]{inputenc}
\usepackage[english,russian]{babel}
\usepackage{cmap}
\usepackage{amssymb}
\usepackage{amsmath}
\usepackage{url}
\usepackage{pifont}
\usepackage{tikz}
\usepackage{verbatim}

\usetikzlibrary{shapes,arrows}
\usetikzlibrary{positioning,automata}
\tikzset{every state/.style={minimum size=0.2cm},
initial text={}
}

\begin{document}
\begin{center} {\LARGE Формальные языки} \end{center}

\begin{center} \Large домашнее задание до 23:59 05.11 \end{center}
\bigskip

\begin{enumerate}
  \item
  {
    Построить LL(1) таблицу по грамматике списков (грамматика ниже).
    \begin{itemize}
      \item Не забываем про то, что нужно избавляться от левой рекурсии и делать левую факторизацию грамматики. Если в одной ячейке таблицы оказалось больше одной ячейки, значит, вы не преобразовали грамматику к подходящей форме.
      \item Добавьте в отчет таблицу LL(1), а также посчитанные для нетерминалов множества FIRST и FOLLOW.
    \end{itemize}
  }
  \item
  {
    Осуществить синтаксический анализ алгоритмом LL(1) для 2 списков, содержащих не меньше 7 терминалов: один список должно быть корректным, другой --- нет. Для корректного списка приведите дерево вывода. Используйте грамматику, полученную в прошлом задании.
    \begin{itemize}
      \item Добавьте в отчет ``историю'' стека --- то, в каком порядке в стек помещались символы грамматики (как мы делали на паре).
    \end{itemize}
  }
\end{enumerate}

\bigskip

\begin{center} Грамматика списков \end{center}

    $$
    \begin{array}{cccccccccc}
       &S & \rightarrow & S ; L & \mid & L \\
       &L & \rightarrow & a & \mid & [S]
    \end{array}
    $$

    \bigskip

    \begin{center} Заготовка для LL(1) таблицы

    (чтобы с версткой не мучиться)
    \end{center}
    \begin{center}
    \begin{tabular}{ l || c | c || c | c | c }
      N & FIRST & FOLLOW &  &  & $ \$ $ \\ \hline
        &       &        &  &  & \\
        &       &        &  &  &

    \end{tabular}
    \end{center}

\end{document}
