\documentclass[12pt]{article}
\usepackage[left=2cm,right=2cm,top=2cm,bottom=2cm,bindingoffset=0cm]{geometry}
\usepackage[utf8x]{inputenc}
\usepackage[english,russian]{babel}
\usepackage{cmap}
\usepackage{amssymb}
\usepackage{amsmath}
\usepackage{url}
\usepackage{pifont}
\usepackage{tikz}
\usepackage{verbatim}
\usepackage{xcolor}

\usetikzlibrary{shapes,arrows}
\usetikzlibrary{positioning,automata}
\tikzset{every state/.style={minimum size=0.2cm},
initial text={}
}


\newcommand{\state}[1]{{\color{red}{#1}}}
\newcommand{\stack}[1]{{\color{blue}{#1}}}
\newcommand{\symbl}[1]{{\color{teal}{#1}}}
\newcommand{\conf}[3]{( \state{#1},\symbl{#2},\stack{#3} )}
\newcommand{\trans}[3]{\delta ( \state{#1},\symbl{#2},\stack{#3} )}

\newenvironment{myauto}[1][3]
{
  \begin{center}
    \begin{tikzpicture}[> = stealth,node distance=#1cm, on grid, very thick]
}
{
    \end{tikzpicture}
  \end{center}
}

\begin{document}
\begin{center} {\LARGE Формальные языки} \end{center}

\begin{center} \Large домашнее задание до 23:59 03.12 \end{center}
\bigskip

\begin{enumerate}
  \item
  {
    Построить магазинный автомат, распознающий язык $\{\omega \omega^R \mid \omega \in \{0,1\}^*\}$
  }
  \item
  {
    Построить контекстно-свободную грамматику по магазинному автомату для языка \[\{0^n 1^m \mid 1 \leq m \leq n \}\]

    \begin{myauto}
      \node[state,initial]   (p)              {$\state{p}$};
      \node[state]           (q) [right=of p] {$\state{q}$};

      \path[->] (p) edge [loop above]    node [left] {$\symbl{0}; \stack{\bot / A\bot}$} ()
                    edge [loop below]    node [left] {$\symbl{0}; \stack{A / AA}$} ()
                    edge node [below] {$\symbl{1}; \stack{A / \varepsilon}$} (q)
                (q) edge [loop above]    node [above] {$\symbl{1}; \stack{A / \varepsilon}$} ()
                    edge [loop below]    node [below] {$\symbl{\varepsilon}; \stack{\bot / \varepsilon}$} ()
                    edge [loop right]    node [right] {$\symbl{\varepsilon}; \stack{A / \varepsilon}$} ()
                ;
    \end{myauto}
  }
  \item
  {
    Построить магазинный автомат для языка арифметических выражений в префиксной нотации с операциями $+$ и $*$ над числами $0$ и $1$.
  }

  \item
  {
    Определить, являются ли следующие языки контекстно свободными. Если являются, привести КС грамматику, иначе --- доказать.
    \begin{itemize}
      \item $\{\omega \omega \omega \mid \omega \in \{0, 1\}^*\}$
      \item $\{a^k b^m b^n a^m \mid n \geq k \geq 0, m \geq 0\}$
      \item $\{a^{n^2} \mid n > 0 \}$
    \end{itemize}
  }
\end{enumerate}
\end{document}
