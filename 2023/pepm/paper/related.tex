\section{Related Work}
\label{sec:related}

A mode generalizes the concept of a direction; this terminology is commonly used in the conventional logic programming community.
In its most primitive form, a mode specifies which arguments of a relation will be known at runtime (input) and which are expected to be computed (output).
Several logic programming languages have mode systems used for optimizations~\cite{warren1977implementing, van1992high, thom1986nu}, with \merc\footnote{Website of the \merc programming language: \url{https://mercurylang.org/}} standing out among them.
\merc is a modern functional-logic programming language with a complicated mode system capable not only of describing directions, but also specifying if a relation in the given mode is deterministic, among other things~\cite{somogyi1987system,overton2002constraint}.

The mode system of \merc is \emph{prescriptive} which means that the mode dictates the data flow.
\merc translates the logic subset of the language into a functional programming language according to the mode assigned to the relation.
The semantics of a \merc program exists only when the mode is assigned.
This is not the case for a \mk program whose semantics is the bag of answers it produces~\cite{rozplokhas2020certified} regardless of the direction, data flow or the order of subgoals within the definition.
In our paper we aim to create a \emph{descriptive} mode system for \mk which does not impose constraints on its execution.
As another consequence, we are free to compare the execution time of programs with and without any optimizations, which \merc papers do not usually do.

There are multiple papers describing automatic mode inference of logic programs~\cite{debray1988automatic, ridoux1999typed, smaus2000mode}.
The most common way to implement mode inference is by abstract interpretation as introduced in~\cite{janssens1992deriving}.
\merc utilizes this approach~\cite{somogyi1987system} in its implementation to guide the compilation.
This mode system proved to be not expressive enough in the context of mode polymorphism, so they researched the use of constraint systems for mode inference~\cite{overton2002constraint}.
While being more precise, this system proved to be too slow to be used in the compiler.

Moreover, the compiler of \merc is highly complicated and demands many annotations from the end-user.
They include type, mode, uniqueness, and determinism specifications.
Many \mk languages are embedded into host languages which are not typed and thus we cannot rely on type information in our conversion.
It is also impossible to do what \merc compiler does as a light-weight embedded DSL which is one of the design principles of the \mk family.
Thus, our goal is to develop the simplest approach to mode analysis which is capable of improving the performance of the verifier-to-solver approach with the least amount of annotations needed from the user --- ideally, only the top-level relation call should be annotated.

\begin{figure}[h]
  \centering
  \begin{subfigure}[b]{0.45\textwidth}
    \begin{lstlisting}[frame=tb]
  let double$^o$ x$^{g \rightarrow g}$ r$^{f \rightarrow g}$ = 
    addo$^o$ x1$^{g \rightarrow g}$ x2$^{g \rightarrow g}$ r$^{f \rightarrow g}$ /\ 
    x1$^{g \rightarrow g}$ ===    x2$^{g \rightarrow g}$

  let rec add$^o$ x$^{g \rightarrow g}$ y$^{g \rightarrow g}$ z$^{f \rightarrow g}$ =
    (x$^{g \rightarrow g}$ ===    O /\ y$^{g \rightarrow g}$ ===    z$^{f \rightarrow g}$) \/
    (x$^{g \rightarrow g}$ ===    S x1$^{f \rightarrow g}$ /\
     add$^o$ x1$^{g \rightarrow g}$ y$^{g \rightarrow g}$ z1$^{f \rightarrow g}$ /\
     z$^{f \rightarrow g}$ ===    S z1$^{g \rightarrow g}$) ]
    \end{lstlisting}
   \caption{Forward direction}
    \label{fig:double_fwd}
  \end{subfigure}
  \hfill
  \begin{subfigure}[b]{0.45\textwidth}
    \begin{lstlisting}[frame=tb]
  let double$^o$ x$^{f \rightarrow g}$ r$^{g \rightarrow g}$ = 
    addo$^o$ x1$^{f \rightarrow g}$ x2$^{f \rightarrow g}$ r$^{g \rightarrow g}$ /\ 
    x1$^{g \rightarrow g}$ ===    x2$^{g \rightarrow g}$

  let rec add$^o$ x$^{f \rightarrow g}$ y$^{f \rightarrow g}$ z$^{g \rightarrow g}$ =
    (x$^{f \rightarrow g}$ ===    O /\ y$^{f \rightarrow g}$ ===    z$^{g \rightarrow g}$) \/
    (z$^{f \rightarrow g}$ ===    S z1$^{g \rightarrow g}$ /\
     add$^o$ x1$^{f \rightarrow g}$ y$^{f \rightarrow g}$ z1$^{g \rightarrow g}$ /\
     x$^{f \rightarrow g}$ ===    S x1$^{g \rightarrow g}$) ]
    \end{lstlisting}
    \caption{Backward direction}
    \label{fig:double_bw}
  \end{subfigure}
  \caption{Normalized doubling and addition relations with mode annotations}
  \label{fig:double_modded}
\end{figure}



