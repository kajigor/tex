\section{Introduction}


There is a well-known observation\cite{lozov2019relational,SemanticsModifiers1} that programs solving certain problems can be acquired by inverting programs solving some other, much simpler, problems.
Sometimes the difference in the ``simplicity'' can be characterized in precise complexity-theoretic terms: for example, type checking for simple typed lambda calculus (STLC) is known to be linear-time (and rather straightforward to implement), while type inference (its inversion) is PTIME-complete~\cite{mairson2004linear}, and type inhabitation problem (its another inversion) is PSPACE-complete~\cite{urzyczyn1997inhabitation}.

In the scope of this paper we will be interested in a more concrete scenario of this generic idea, namely, of turning \emph{verifiers} into \emph{solvers}.
A verifier is a procedure which, given an instance of the problem and some \emph{sample}, verifies if this sample is a solution.
A solver takes an instance of the problem and returns such a sample which makes the verifier to succeed.
For the variety of search problems the implementation of a verifier is straightforward; on the other hand its inversion is a solver, which as a rule is much harder to implement in an explicit manner.
There are a few approaches to program inversion~\cite{RevURA,aman2020foundations}, and the properties of the solver produced by inversion greatly depend on the approach utilized.
We focus on the application of \emph{relational programming}~\cite{TheReasonedSchemer} as a way to run programs in the reverse direction.


\begin{figure*}[t!]
    \begin{tabular}{llll}
        $\Def_{V}^N$ & $:$    & $R_{n}\left( x_1, \ldots, x_{n} \right) = \Disj_{V}, x_{i}\in V$                        & normalized relation definition \\
        $\Disj_{V}$  & $:$    & $\bigvee\left( c_1, \ldots, c_{n} \right), c_{i}\in \Conj_{V}$                          & normal form                    \\
        $\Conj_{V}$  & $:$    & $\bigwedge\left( g_1, \ldots, g_n \right), g_{i}\in \Base_{V}$                          & normal conjunction             \\
        $\Base_{V}$  & $:$    & $V \equiv \FlatTerm_{V}$                                                                & flat unification               \\
                     & $\mid$ & $R_{n}\left( x_1, \ldots, x_{n} \right), x_{i}\in V, i \neq j \Rightarrow x_i \neq x_j$ & flat call                      \\

        % $\Delay$ & $:$ &  $\text{Delay} \mid \text{NoDelay} $ &
    \end{tabular}
    \caption{Abstract syntax of \micro in the normal form}
    \label{fig:miniKanren}
\end{figure*}


Relational programming can be considered as a subfield of conventional logic programming focused on study of implementation techniques and applications of \emph{purely relational} specifications.
In a narrow sense, relational programming amounts to writing programs in \mk\footnote{Website of the \mk programming language: \url{http://minikanren.org}}~--- an embedded DSL initially developed for \textsc{Scheme} and later ported for dozens of other host languages.
Based on the same theory of first-order Horn clauses as, for example, \prolog, \mk employs a complete \emph{interleaving search}\cite{kiselyov2005backtracking, rozplokhas2020certified} and discourages the use of extra-logical features such as knowledge of concrete search order, ``cuts'', side-effects, efficient, but non-relational arithmetic, etc.
In conventional logic programming the specification provided by the end-user usually encodes a certain concrete way to solve a problem.
Contrary to this, \mk shifts the focus onto the specification of the problem itself with no certain hints on how to solve its various instances.
% Thus, contrary to conventional logic programming, where the specification provided by an end-user as a rule encodes a certain concrete way to solve a problem, in \mk the focus is shifted even more into the specification of the problem itself with no certain hints of how to solve its various instances.
This makes the specifications written in \mk short, elegant and expressive.

It is possible to directly employ the verifier-to-solver approach~\cite{type-inferencer,kosarev2020relational} with \mk.
It has been successfully applied in a few non-trivial projects~\cite{kosarev2022declarative,lozov2023relational}.
On the other hand, many useful optimization techniques cannot be applied for \mk programs directly since these programs lack an important part of information~--- the \emph{direction} under which relational verifier turns into a solver.
By taking this information into account, it is possible to make the approach more universally practically applicable.

In this paper we present the results of our exploration in the area of \emph{mode inference} and \emph{functional conversion} for \mk.
Mode analysis and inference is a relevant technique for conventional logic programming~\cite{debray1988automatic,somogyi1987system,overton2002constraint}.
A mode can be considered as an implicit specification of a direction in which a relation is intended to be evaluated.
Given a user-defined description of \emph{modes} for (some) relations, mode analysis propagates the mode information through the rest of the logic program thus defining more concrete evaluation strategy for the rest of its relations.

Various notions and concrete approaches are employed for mode analysis in various settings, and we give a survey in Section~\ref{sec:mode}.
In our setting we consider user-defined mode specification for the top-level goal as the prescription of the direction in which relational specification has to be evaluated to provide a solver for the problem in question.
However, such a prescription cannot be directly employed in \mk as it contradicts the very nature of relational programming.
Instead, we accompany mode inference with \emph{functional conversion}~--- a transformation which, given a relational specification, a top-level goal, user-defined modes for this goal and the results of mode inference provides a regular functional program which delivers exactly the same answers as the top-level goal being evaluated in the direction prescribed by the modes.
In addition, functional conversion can sometimes eliminate the interpretation overhead introduced by \mk implementation as a shallow DSL: it is capable of replacing unification with pattern-matching, make use of deterministic order of evaluation if such an order is discovered by mode inference, etc.

The contribution of this paper is as follows:

\begin{itemize}
\item We reiterate on mode inference for \mk, specifying concrete requirements specific for both \mk and our ultimate goal of putting verifier-to-solver idea to work.
\item We describe a concrete approach to mode inference which takes the aforementioned requirements into account. As generally mode inference is known to be undecidable we   develop a number of heuristics specific to our case.
\item We implement both mode inference and functional conversion for a reference \mk implementation.
\item We evaluate our implementation on several benchmarks, to investigate the advantages, drawbacks, and potential areas for improvement in our approach.
\end{itemize}

The rest of the paper is structures as follows. ...




\begin{comment}
\emph{Inverse computation} is the technique in which a program can be automatically inverted to solve problems different from its original purpose.
For example, by inverting a sorting function, one can generate permutations; by inverting multiplication, one achieves division.
In the context of software development, inverse computations open the door to verifier-to-solver approach.
In it, a verifier, whose primary purpose is to check whether a candidate solution is indeed a solution to a problem, is inverted to become a solver.
This way, an interpreter for a programming language can be used for program synthesis, a type checker---to solve type inhabitation problem and so on~\cite{Untagged, lozov2019relational}.

There are multiple approaches to inverse computations such as universal resolving algorithm~\cite{RevURA} and semantics modifiers~\cite{SemanticsModifiers1}.
The one, which we are considering in this paper, employs \emph{relational programming}.
It is a powerful paradigm which, akin to logic programming, is based on Horn clauses.
Contrary to such languages as \prolog, the relational paradigm discourages the use of extra-logic features such as cut and instead employs interleaving search strategy to guarantee the completeness of search.
Thanks to the latter, it is also not expected that a program is deterministic, i.e. produces at most one result, which plays an important role in enabling verifier-to-solver approach.

In a narrow sense, relational programming means programming in a \mk language\todo{footnote}.
It is a family of small yet powerful embedded DSLs whose aim is to bring the benefits of logic programming into a general-purpose programming language.
A relational program exist as a part of a host program, which utilizes query results in some way.
The host languages are not expected to be able to process logic variables, nondeterminism and other aspects of relational computations.
The host program usually only deals with a finite subset of answers, which have been reified into a ground representation, meaning the answer do not contain any logic variables.

When a relation is expected to produce ground answers, and the direction in which it is intended to be run is known, then it becomes possible to convert it into a function which may execute significantly faster than its relational counterpart.
Performance improvement comes from reducing inherent interpretation overhead as well as replacing expensive unifications with considerably faster equality checks, assignments and pattern matches of the host language.
An informal functional conversion scheme was introduced in the paper~\cite{verbitskaia2022direction}and a semi-automatic functional conversion algorithm and implementation for a minimal core relational programming language \micro was described in~\footnote{cite mk 2023}.

\todo{What's the contribution?}

This paper focuses on converting to the target languages of \haskell and \ocaml, although other languages can also be considered as potential target languages.
Our evaluation showed performance improvement of $2.5$ times for propositional formulas synthesis and up to $3$ orders of magnitude improvement for relations over Peano numbers.
\end{comment}

