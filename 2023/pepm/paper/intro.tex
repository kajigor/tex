\section{Introduction}


There is a well-known observation~\cite{lozov2019relational,SemanticsModifiers1} that programs solving certain problems can be acquired by inverting programs solving some other, much simpler, problems.
Sometimes the difference in the ``simplicity'' can be characterized in precise complexity-theoretic terms: for example, type checking for simple typed lambda calculus (STLC) is known to be linear-time (and rather straightforward to implement), while type inference (its inversion) is PTIME-complete~\cite{mairson2004linear}, and type inhabitation problem (another potential inversion) is PSPACE-complete~\cite{urzyczyn1997inhabitation}.

In the scope of this paper we will be interested in a more concrete scenario of this generic idea, namely, in turning \emph{verifiers} into \emph{solvers}.
A verifier is a procedure that, given an instance of the problem and some \emph{sample}, verifies if this sample is a solution.
A solver takes an instance of the problem and returns such a sample which makes the verifier to succeed.
For the variety of search problems, the implementation of a verifier is straightforward; on the other hand its inversion is a solver, which as a rule is much harder to implement in an explicit manner.
There are a few approaches to program inversion~\cite{RevURA,aman2020foundations}, and the properties of the solver produced by inversion greatly depend on the approach utilized.
We focus on the application of \emph{relational programming}~\cite{TheReasonedSchemer} as a way to run programs in the reverse direction.

Relational programming can be considered as a subfield of conventional logic programming focused on the study of implementation techniques and applications of \emph{purely relational} specifications.
In a narrow sense, relational programming amounts to writing programs in \mk\footnote{Website of the \mk programming language: \url{http://minikanren.org}}~--- an embedded DSL initially developed for \textsc{Scheme} and later ported to dozens of other host languages.
Based on the same theory of first-order Horn clauses as, for example, \prolog, \mk employs a complete \emph{interleaving search}~\cite{kiselyov2005backtracking, rozplokhas2020certified} and discourages the use of extra-logical features such as knowledge of concrete search order, ``cuts'', side effects, efficient, but non-relational arithmetic, etc.
In conventional logic programming the specification provided by the end-user usually encodes a certain concrete way to solve a problem.
Contrary to this, \mk shifts the focus onto the specification of the problem itself, with no certain hints on how to solve its various instances.
% Thus, contrary to conventional logic programming, where the specification provided by an end-user as a rule encodes a certain concrete way to solve a problem, in \mk the focus is shifted even more into the specification of the problem itself with no certain hints of how to solve its various instances.
This makes the specifications written in \mk short, elegant and expressive.

It is possible to directly employ the verifier-to-solver approach~\cite{byrd2017unified,kosarev2020relational} with \mk.
It has been successfully applied in a few non-trivial projects~\cite{kosarev2022declarative,lozov2023relational}.
On the other hand, many useful optimization techniques cannot be applied to \mk programs directly since these programs lack an important part of information~--- the \emph{direction} under which relational verifier turns into a solver.
By taking this information into account, it is possible to make the approach more universally practical.

In this paper we present the results of our exploration in the area of \emph{mode inference} and \emph{functional conversion} for \mk.
Mode analysis and inference is a relevant technique for conventional logic programming~\cite{debray1988automatic,somogyi1987system,overton2002constraint}.
A mode can be considered as an implicit specification of a direction in which a relation is intended to be evaluated.
Given a user-defined description of \emph{modes} for (some) relations, mode analysis propagates the mode information through the rest of the logic program, thus defining more concrete evaluation strategy for the rest of its relations.

Various notions and concrete approaches are employed for mode analysis in different settings, and we give a survey in Section~\ref{sec:related}.
In our setting we consider user-defined mode specification for the top-level goal as the prescription of the direction in which relational specification has to be evaluated to provide a solver for the problem in question.
However, such a prescription cannot be directly employed in \mk as it contradicts the very nature of relational programming.
Instead, we accompany mode inference with \emph{functional conversion}~--- a transformation which, given a relational specification, a top-level goal, user-defined modes for this goal and the results of mode inference provides a regular functional program which delivers exactly the same answers as the top-level goal being evaluated in the direction prescribed by the modes.
In addition, functional conversion can sometimes eliminate the interpretation overhead introduced by \mk implementation as a shallow DSL: it is capable of replacing unification with pattern-matching, making use of deterministic order of evaluation if such an order is discovered by mode inference, etc.


The contribution of this paper is as follows:

\begin{itemize}
\item We elaborate on mode inference for \mk, specifying concrete requirements specific for \mk and our ultimate goal of putting the verifier-to-solver idea to work.
\item We describe a concrete approach to mode inference which takes the aforementioned requirements into account. As mode inference in general is known to be undecidable, we develop a number of heuristics specific to our case.
\item We implement both mode inference and functional conversion for a reference \mk implementation.
\item We evaluate our implementation on several benchmarks to investigate the advantages, drawbacks, and potential areas for improvement in our approach.
\end{itemize}

The rest of the paper is structured as follows.
Section~\ref{sec:mk} describes the main ideas behind relational programming, as well as the object language used in this paper.
Section~\ref{sec:conversion} describes the scheme of functional conversion.
The conversion is illustrated by examples in Section~\ref{sec:examples}.
The evaluation of the approach is presented in Section~\ref{sec:evaluation}, followed by the discussion in Section~\ref{sec:discussion}.
Related work including inverse computations and mode analysis is discussed in Section~\ref{sec:related}.
We conclude and sketch the directions for future work in Section~\ref{sec:conclusion}.
