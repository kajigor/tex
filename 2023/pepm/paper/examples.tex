

\begin{columns}[t]
    \begin{column}{0.49\textwidth}
        \begin{figure}[h!]
        \centering
          \begin{subfigure}[b]{0.45\textwidth}
            \begin{lstlisting}[frame=tb]
          let rec add$^o$ x$^{g \to g}$ y$^{f \to g}$ z$^{f \to g}$ =
            (x$^{g \to g}$ ===    O /\ y$^{f \to g}$ ===    z$^{f \to g}$) \/
            (x$^{g \to g}$ ===    S x$_1^{f \to g}$ /\
             add$^o$ x$_1^{g \to g}$ y$^{f \to g}$ z$_1^{f \to g}$ /\
             z$^{f \to g}$ ===    S z$_1^{g \to g}$)
            \end{lstlisting}
           \caption{Mode inference result}
            \label{fig:addo_modded}
          \end{subfigure}
          \hfill
          \begin{subfigure}[b]{0.45\textwidth}
            \begin{lstlisting}[frame=tb]
          genNat = msum
            [ return O
            , do { x <- genNat
                 ; return (S x) } ]
          runAddoIIO x = addoIIO x genNat
            \end{lstlisting}
           \caption{Generator of Peano numbers}
          \end{subfigure}
        \end{figure}
    
    \end{column}
    \begin{column}{0.49\textwidth}
        
    \end{column}
\end{columns}

\begin{figure}[h!]
  \centering
  \begin{subfigure}[b]{0.45\textwidth}
    \begin{lstlisting}[frame=tb]
  let rec add$^o$ x$^{g \to g}$ y$^{f \to g}$ z$^{f \to g}$ =
    (x$^{g \to g}$ ===    O /\ y$^{f \to g}$ ===    z$^{f \to g}$) \/
    (x$^{g \to g}$ ===    S x$_1^{f \to g}$ /\
     add$^o$ x$_1^{g \to g}$ y$^{f \to g}$ z$_1^{f \to g}$ /\
     z$^{f \to g}$ ===    S z$_1^{g \to g}$)
    \end{lstlisting}
   \caption{Mode inference result}
    \label{fig:addo_modded}
  \end{subfigure}
  \hfill
  \begin{subfigure}[b]{0.45\textwidth}
    \begin{lstlisting}[frame=tb]
  genNat = msum
    [ return O
    , do { x <- genNat
         ; return (S x) } ]
  runAddoIIO x = addoIIO x genNat
    \end{lstlisting}
   \caption{Generator of Peano numbers}
    \label{fig:addo_gen}
  \end{subfigure}

  \hfill

  \begin{subfigure}[b]{0.91\textwidth}
    \begin{lstlisting}[frame=tb]
  addoIOO x0 gen_addoIOO_x2 = msum
    [ do { guard (x0 == O)
         ; (x1, x2) <- do { x2 <- gen_addoIOO_x2 ; return (x2, x2) }
         ; return (x1, x2) }
    , do { x3 <- case x0 of { S y3 -> return y3 ; _ -> mzero }
         ; (x1, x4) <- addoIOO x3 gen_addoIOO_x2
         ; let {x2 = S x4} ; return (x1, x2) } ]
    \end{lstlisting}
    \caption{Functional conversion}
    \label{fig:addo_hsk}
  \end{subfigure}
  \caption{Addition relation when only the first argument is \inm}
  \label{fig:addo}
\end{figure}


\section{Examples}
\label{sec:examples}

In this section, we provide some examples which demonstrate mode analysis and conversion results.


\subsection{Multiplication Relation}


Figure~\ref{fig:mult} shows the implementation of the multiplication relation \lstinline{mul$^o$}, the mode analysis result for mode \lstinline[breaklines=true]{mul$^o$ x$^{f \to g}$ y$^{g\to g}$ z$^{g\to g}$} , and the results of functional conversion into \haskell and \ocaml.

Note that the unification comes last in the second disjunct.
This is because before the two relation calls are done, both variables in the unification are free.
Our version of mode inference puts the relation calls before the unification, but the order of the calls depends on their order in the original relation.
There is nothing else our mode inference uses to prefer the order presented in the figure over the opposite: \lstinline{mul$^o$ x$_1^{f \to g}$ y$^{g \to g}$ z$_1^{f \to g}$ /\        add$^o$ y$^{g \to g}$ z$_1^{g \to g}$ z$^{g \to g}$}.
    However, it is possible to derive this optimal order, if determinism analysis is employed: \lstinline{add$^o$ y$^{g \to g}$ z$_1^{f \to g}$ z$^{g \to g}$} is deterministic while \lstinline{mul$^o$ x$_1^{f \to g}$ y$^{g \to g}$ z$_1^{f \to g}$} is not.
    Putting nondeterministic computations first makes the search space larger, and thus should be avoided if another order is possible.

    Functional conversions in both languages are similar, modulo the syntax.
    The \haskell version employs do-notation, while we use let-syntax in the \ocaml code.
    Both are syntactic sugar for monadic computations over streams.
    We use the following convention to name the functions: we add a suffix to the relation's name whose length is the same as the number of the relation's arguments.
    The suffix consists of the letters \lstinline{I} and \lstinline{O} which denote whether the argument in the corresponding position is \inm or \outm.
    The function \lstinline{msum} uses the interleaving function \lstinline{mplus} to concatenate the list of streams constructed from disjuncts.
    To check conditions, we use the function \lstinline{guard} which fails the monadic computation if the condition does not hold.
    Note that even though patterns for the variable \lstinline{x0} in the function \lstinline{addoIOI} are disjunct in two branches, we do not express them as a single pattern match.
    Doing so would improve readability, but it does not make a difference when it comes to the performance, according to our evaluation.


    \subsection{The Mode of Addition Relation which Needs a Generator}

    Consider the example of the addition relation in mode \lstinline[breaklines=true]{add$^o$ x$^{g \to g}$ y$^{f \to g}$ z$^{f \to g}$} presented in figure~\ref{fig:addo}.
The unification in the first disjunct of this relation involves two free variables.
We use a generator \lstinline{gen_addoIIO_x2} to generate a stream of ground values for the variable \lstinline{z} which is passed into the function \lstinline{addIIO} as an argument.
It is up to the user to provide a suitable generator.
One of the possible generators which produces all Peano numbers in order and an example of its usage are presented in figure~\ref{fig:addo_gen}.

The generators which produce an infinite stream should be inverse eta-delayed in \ocaml and other non-lazy languages.
Otherwise, the function would not terminate trying to eagerly produce all possible ground values before using any of them.

It is possible to automatically produce generators from the data type of a variable, but it is currently not implemented, as we work with an untyped version of \micro.
