\section{Normal Form Abstract Syntax of \mk}

In this paper we use core \mk usually referred to as \micro.
In its syntax, a relation is a goal comprised of disjunctions or conjunctions of other goals.
A base goal can be either an explicit unification of two terms or a call to a relation.
An example program in \mk is shown in figure~\ref{fig:addo_mk}.
It relates triples of Peano natural numbers \lstinline{x}, \lstinline{y}, \lstinline{z} such that \lstinline{x + y = z}.
We use identifiers which start with the upper-case letters as constructors, and identifiers which start with the lower-case letters as variable names.
A superscript \lstinline{$^o$} denotes a relation name while the keyword \lstinline{fresh} introduces fresh variables into the scope.
To execute a relation, one should provide a query to run, for example the query \lstinline{run q (add$^o$ q q (S (S O)))} finds a number which, doubled, is 2 in Peano representation.


\begin{figure}[t]
  \centering
    \begin{lstlisting}[frame=tb]
  let rec add$^o$ x y z =
    (x === O /\ y === z) \/
    (fresh x$_1$, z$_1$ in
       x === S x$_1$ /\
       add$^o$ x$_1$ y z$_1$ /\
       z === S z$_1$)
    \end{lstlisting}
  \caption{Addition relation in \mk}
  \label{fig:addo_mk}
\end{figure}



To simplify the functional conversion scheme, we consider \mk relations to be in the superhomogeneous normal form used in the \merc programming language~\cite{somogyi1996execution}.
Converting an arbitrary \mk relation into the normal form is a simple syntactic transformation, which we omit.

In the normal form, a term is either a variable or a constructor application which is flat and linear.
Linearity means that arguments of a constructor are distinct variables.
To be flat, a term should not contain any nested constructors.
Each constructor has a fixed arity $n$.
Below is the abstract syntax of the term language over the set of variables $V$.
\[  \FlatTerm_{V} = V \cup \{\Cons_{n}\left( x_1, \ldots, x_{n} \right) \mid x_{i}\in V; i \neq j \Rightarrow x_i \neq x_j\} \]


Whenever a term which does not adhere to this form is encountered in a unification or as an argument of a call, it is transformed into a conjunction of several unifications, as illustrated by the following examples:
\[
    \begin{array}{rll}
        C\left( x_1, x_2 \right)                      & \equiv C\left( C\left( y_1, y_2 \right), y_3 \right)                                          & \Rightarrow \\
                                                      & \multicolumn{2}{l}{\quad\Rightarrow x_1 \equiv C\left( y_1, y_2 \right) \land x_2 \equiv y_3}               \\
        C\left( C\left( x_1, x_2 \right), x_3 \right) & \equiv C\left( C\left( y_1, y_2 \right), y_3 \right)                                          & \Rightarrow \\
                                                      & \multicolumn{2}{l}{\quad\Rightarrow x_1 \equiv y_1 \land x_2 \equiv y_2 \land x_3 \equiv y_3}               \\
        x \equiv C \left(y, y \right)                 & \multicolumn{2}{l}{\quad\Rightarrow x \equiv C \left(y_1, y_2\right) \land y_1 \equiv y_2}                  \\
        add^o\left( x, x, z \right)                   & \multicolumn{2}{l}{\quad\Rightarrow add^o\left( x_1, x_2, z \right) \land x_1 \equiv x_2}
    \end{array}
\]

Unification in the normal form is restricted to always unify a variable with a term.
We also prohibit using disjunctions inside conjunctions.
The normalization procedure declares a new relation whenever this is encountered.
This is done to limit the number of possible permutations one has to consider when doing the mode inference.

The complete abstract syntax of the \mk language used in this paper is presented in figure~\ref{fig:miniKanren}.