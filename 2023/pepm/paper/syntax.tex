\section{Relational Programming and \mk}
\label{sec:mk}

\begin{figure}[t]
  \centering
    \begin{lstlisting}[frame=tb]
  let rec add$^o$ x y z =
    (x === O /\ y === z) \/
    (fresh x$_1$, z$_1$ in
       x === S x$_1$ /\
       add$^o$ x$_1$ y z$_1$ /\
       z === S z$_1$)
    \end{lstlisting}
  \caption{Addition relation in \mk}
  \label{fig:addo_mk}
\end{figure}


\begin{figure*}[t!]
    \begin{tabular}{llll}
        $\Def_{V}^N$ & $:$    & $R_{n}\left( x_1, \ldots, x_{n} \right) = \Disj_{V}, x_{i}\in V$                        & normalized relation definition \\
        $\Disj_{V}$  & $:$    & $\bigvee\left( c_1, \ldots, c_{n} \right), c_{i}\in \Conj_{V}$                          & normal form                    \\
        $\Conj_{V}$  & $:$    & $\bigwedge\left( g_1, \ldots, g_n \right), g_{i}\in \Base_{V}$                          & normal conjunction             \\
        $\Base_{V}$  & $:$    & $V \equiv \FlatTerm_{V}$                                                                & flat unification               \\
                     & $\mid$ & $R_{n}\left( x_1, \ldots, x_{n} \right), x_{i}\in V, i \neq j \Rightarrow x_i \neq x_j$ & flat call                      \\

        % $\Delay$ & $:$ &  $\text{Delay} \mid \text{NoDelay} $ &
    \end{tabular}
    \caption{Abstract syntax of \micro in the normal form}
    \label{fig:miniKanren}
\end{figure*}



\begin{figure*}[t!]
  \centering
  \vspace{1.5mm}
\begin{subfigure}[b]{0.45\textwidth}
    \begin{lstlisting}[frame=tb]
  let double$^o$ x$^{g \rightarrow g}$ r$^{f \rightarrow g}$ =
    addo$^o$ x$_1^{g \rightarrow g}$ x$_2^{g \rightarrow g}$ r$^{f \rightarrow g}$ /\
    x$_1^{g \rightarrow g}$ ===    x$_2^{g \rightarrow g}$

  let rec add$^o$ x$^{g \rightarrow g}$ y$^{g \rightarrow g}$ z$^{f \rightarrow g}$ =
    (x$^{g \rightarrow g}$ ===    O /\ y$^{g \rightarrow g}$ ===    z$^{f \rightarrow g}$) \/
    (x$^{g \rightarrow g}$ ===    S x$_1^{f \rightarrow g}$ /\
     add$^o$ x$_1^{g \rightarrow g}$ y$^{g \rightarrow g}$ z$_1^{f \rightarrow g}$ /\
     z$^{f \rightarrow g}$ ===    S z$_1^{g \rightarrow g}$)
    \end{lstlisting}
    \caption{Forward direction}
    \label{fig:double_fwd}
  \end{subfigure}
\hfill
  \begin{subfigure}[b]{0.45\textwidth}
    \begin{lstlisting}[frame=tb]
  let double$^o$ x$^{f \rightarrow g}$ r$^{g \rightarrow g}$ =
    addo$^o$ x$_1^{f \rightarrow g}$ x$_2^{f \rightarrow g}$ r$^{g \rightarrow g}$ /\
    x$_1^{g \rightarrow g}$ ===    x$_2^{g \rightarrow g}$

  let rec add$^o$ x$^{f \rightarrow g}$ y$^{f \rightarrow g}$ z$^{g \rightarrow g}$ =
    (x$^{f \rightarrow g}$ ===    O /\ y$^{f \rightarrow g}$ ===    z$^{g \rightarrow g}$) \/
    (z$^{f \rightarrow g}$ ===    S z$_1^{g \rightarrow g}$ /\
     add$^o$ x$_1^{f \rightarrow g}$ y$^{f \rightarrow g}$ z$_1^{g \rightarrow g}$ /\
     x$^{f \rightarrow g}$ ===    S x$_1^{g \rightarrow g}$)
    \end{lstlisting}
    \caption{Backward direction}
    \label{fig:double_bw}
  \end{subfigure}
  \caption{Normalized doubling and addition relations with mode annotations}
  \label{fig:double_modded}
\end{figure*}




Relational programming as a subfield of conventional logic programming which is focused on using purely relational specifications only.
Using extra-logical features such as ``cuts'' and side-effects common in \prolog as well as the knowledge of the particular direction the relation is supposed to be run is discouraged.
Since the search in \mk is complete~\cite{kiselyov2005backtracking,rozplokhas2020certified}, all answers to the query will eventually be found without the programmer taking into account a particular search strategy used in the language implementation.
It also means that the way in which a program is structured has no effect on which answers are found, only on the order in which they are computed.

In this paper we use a core \mk language usually referred to as \micro.
In its syntax, a relation is a goal comprised of disjunctions (\lstinline{\/}) or conjunctions (\lstinline{/\}) of other goals.
A base goal can be either an explicit unification of two terms (\lstinline{===}) or a call of a relation.
An example program in \mk is shown in Figure~\ref{fig:addo_mk}.
It relates triples of Peano natural numbers \lstinline{x}, \lstinline{y}, \lstinline{z} such that \lstinline{x + y = z}.
We use a syntax notation such that constructors are denoted by identifiers which start with the upper-case letters, while identifiers which start with the lower-case letters are used as variable names.
The~superscript ``\lstinline{$^o$}'' denotes a relation name while the keyword \lstinline{fresh} introduces fresh variables into the scope.
To execute a relation, one should provide a query to run.
For example, the query \lstinline{run q (add$^o$ q q (S (S O)))} finds a number which, doubled, is 2 in Peano representation, namely \lstinline{S O}.
Some queries can compute values of several variables, and there may be infinitely many of them.
For example, the query \lstinline{run q (fresh y,z in q == (y, z) /\ add$^o$ (S O) y z)} finds all \lstinline{y} and \lstinline{z} such that $1 + y = z$.
These answers are \lstinline{(O, S O)}, \lstinline{(S O, S (S O))} and so forth.


To simplify the functional conversion scheme, we consider \mk relations to be in the superhomogeneous normal form used in the \merc programming language~\cite{somogyi1996execution}.
Converting an arbitrary \mk relation into the normal form is a simple syntactic transformation, which we omit.

In the normal form, a term is either a variable or a constructor application which is flat and linear.
Linearity means that arguments of a constructor are distinct variables.
To be flat, a term should not contain any nested constructors.
Each constructor has a fixed arity $n$.
Below is the abstract syntax of the term language over the set of variables $V$:

\[  \FlatTerm_{V} = V \cup \{\Cons_{n}\left( x_1, \ldots, x_{n} \right) \mid x_{i}\in V; i \neq j \Rightarrow x_i \neq x_j\} \]


Whenever a term which does not adhere to this form is encountered in a unification or as an argument of a call, it is transformed into a conjunction of several unifications, as illustrated by the examples in Figure~\ref{fig:superhomo}.

Unification in the normal form is restricted to always unify a variable with a term.
We also prohibit using disjunctions inside conjunctions.
The normalization procedure declares a new relation whenever this is encountered.
This is done to limit the number of possible permutations one has to consider when doing the mode inference.

The complete abstract syntax of the \mk language used in this paper is presented in Figure~\ref{fig:miniKanren}.
