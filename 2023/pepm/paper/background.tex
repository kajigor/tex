\section{Background}

In this section, we give the abstract syntax of \mk version used in this paper and describe the concept of modes which was developed earlier for other logic languages.

\subsection{Normal Form Abstract Syntax of \mk}

In this paper we use core \mk usually referred to as \micro.
In its syntax, a relation is a goal comprised of disjunctions or conjunctions of other goals.
A base goal can be either an explicit unification of two terms or a call to a relation.
An example program in \mk is shown in Figure~\ref{fig:addo_mk}.
It relates triples of Peano natural numbers \lstinline{x}, \lstinline{y}, \lstinline{z} such that \lstinline{x + y = z}.
We use identifiers which start with the upper-case letters as constructors, and identifiers which start with the lower-case letters as variable names.
A superscript \lstinline{$^o$} denotes a relation name while the keyword \lstinline{fresh} introduces fresh variables into the scope.
To execute a relation, one should provide a query to run, for example the query \lstinline{run q (add$^o$ q q (S (S O)))} finds a number which, doubled, is 2 in Peano representation.


\begin{figure}[t]
  \centering
    \begin{lstlisting}[frame=tb]
  let rec add$^o$ x y z =
    (x === O /\ y === z) \/
    (fresh x$_1$, z$_1$ in
       x === S x$_1$ /\
       add$^o$ x$_1$ y z$_1$ /\
       z === S z$_1$)
    \end{lstlisting}
  \caption{Addition relation in \mk}
  \label{fig:addo_mk}
\end{figure}



To simplify the functional conversion scheme, we consider \mk relations to be in the superhomogeneous normal form used in the \merc programming language~\cite{somogyi1996execution}.
Converting an arbitrary \mk relation into the normal form is a simple syntactic transformation, which we omit.

In the normal form, a term is either a variable or a constructor application which is flat and linear.
Linearity means that arguments of a constructor are distinct variables.
To be flat, a term should not contain any nested constructors.
Each constructor has a fixed arity $n$.
Below is the abstract syntax of the term language over the set of variables $V$.
\[  \FlatTerm_{V} = V \cup \{\Cons_{n}\left( x_1, \ldots, x_{n} \right) \mid x_{i}\in V; i \neq j \Rightarrow x_i \neq x_j\} \]


Whenever a term which does not adhere to this form is encountered in a unification or as an argument of a call, it is transformed into a conjunction of several unifications, as illustrated by the following examples:
\[
    \begin{array}{rll}
        C\left( x_1, x_2 \right)                      & \equiv C\left( C\left( y_1, y_2 \right), y_3 \right)                                          & \Rightarrow \\
                                                      & \multicolumn{2}{l}{\quad\Rightarrow x_1 \equiv C\left( y_1, y_2 \right) \land x_2 \equiv y_3}               \\
        C\left( C\left( x_1, x_2 \right), x_3 \right) & \equiv C\left( C\left( y_1, y_2 \right), y_3 \right)                                          & \Rightarrow \\
                                                      & \multicolumn{2}{l}{\quad\Rightarrow x_1 \equiv y_1 \land x_2 \equiv y_2 \land x_3 \equiv y_3}               \\
        x \equiv C \left(y, y \right)                 & \multicolumn{2}{l}{\quad\Rightarrow x \equiv C \left(y_1, y_2\right) \land y_1 \equiv y_2}                  \\
        add^o\left( x, x, z \right)                   & \multicolumn{2}{l}{\quad\Rightarrow add^o\left( x_1, x_2, z \right) \land x_1 \equiv x_2}
    \end{array}
\]

Unification in the normal form is restricted to always unify a variable with a term.
We also prohibit using disjunctions inside conjunctions.
The normalization procedure declares a new relation whenever this is encountered.
This is done to limit the number of possible permutations one has to consider when doing the mode inference.

The complete abstract syntax of the \mk language used in this paper is presented in Figure~\ref{fig:miniKanren}.

\label{sec:mode}
\subsection{Modes}

\begin{figure*}[t!]
  \centering
  \vspace{1.5mm}
\begin{subfigure}[b]{0.45\textwidth}
    \begin{lstlisting}[frame=tb]
  let double$^o$ x$^{g \rightarrow g}$ r$^{f \rightarrow g}$ =
    addo$^o$ x$_1^{g \rightarrow g}$ x$_2^{g \rightarrow g}$ r$^{f \rightarrow g}$ /\
    x$_1^{g \rightarrow g}$ ===    x$_2^{g \rightarrow g}$

  let rec add$^o$ x$^{g \rightarrow g}$ y$^{g \rightarrow g}$ z$^{f \rightarrow g}$ =
    (x$^{g \rightarrow g}$ ===    O /\ y$^{g \rightarrow g}$ ===    z$^{f \rightarrow g}$) \/
    (x$^{g \rightarrow g}$ ===    S x$_1^{f \rightarrow g}$ /\
     add$^o$ x$_1^{g \rightarrow g}$ y$^{g \rightarrow g}$ z$_1^{f \rightarrow g}$ /\
     z$^{f \rightarrow g}$ ===    S z$_1^{g \rightarrow g}$)
    \end{lstlisting}
    \caption{Forward direction}
    \label{fig:double_fwd}
  \end{subfigure}
\hfill
  \begin{subfigure}[b]{0.45\textwidth}
    \begin{lstlisting}[frame=tb]
  let double$^o$ x$^{f \rightarrow g}$ r$^{g \rightarrow g}$ =
    addo$^o$ x$_1^{f \rightarrow g}$ x$_2^{f \rightarrow g}$ r$^{g \rightarrow g}$ /\
    x$_1^{g \rightarrow g}$ ===    x$_2^{g \rightarrow g}$

  let rec add$^o$ x$^{f \rightarrow g}$ y$^{f \rightarrow g}$ z$^{g \rightarrow g}$ =
    (x$^{f \rightarrow g}$ ===    O /\ y$^{f \rightarrow g}$ ===    z$^{g \rightarrow g}$) \/
    (z$^{f \rightarrow g}$ ===    S z$_1^{g \rightarrow g}$ /\
     add$^o$ x$_1^{f \rightarrow g}$ y$^{f \rightarrow g}$ z$_1^{g \rightarrow g}$ /\
     x$^{f \rightarrow g}$ ===    S x$_1^{g \rightarrow g}$)
    \end{lstlisting}
    \caption{Backward direction}
    \label{fig:double_bw}
  \end{subfigure}
  \caption{Normalized doubling and addition relations with mode annotations}
  \label{fig:double_modded}
\end{figure*}



A mode generalizes the concept of a direction; this terminology is commonly used in the conventional logic programming community.
In its most primitive form, a mode specifies which arguments of a relation will be known at runtime (input) and which are expected to be computed (output).
Several logic programming languages have mode systems used for optimizations, with \merc standing out among them.
\merc\footnote{Website of the \merc programming language: \url{https://mercurylang.org/}} is a modern functional-logic programming language with a complicated mode system capable not only of describing directions, but also specifying if a relation in the given mode is deterministic, among other things~\cite{overton2002constraint}.

Given an annotation for a relation, mode inference determines modes of each variable of the relation.
For some modes, conjunctions in the body of a relation may need reordering to ensure that consumers of computed values come after the producers of said values so that a variable is never used before it is bound to some value.
In this project, we employed the least complicated mode system, in which variables may only have an \inm or \outm mode.
A mode maps variables of a relation to a pair of the initial and final instantiations.
The mode \inm stands for $g \rightarrow g$, while \outm stands for $f \rightarrow g$.
The instantiation $f$ represents an unbound, or \emph{free}, variable, when no information about its possible values is available.
When the variable is known to be \emph{ground}, its instantiation is $g$.

In this paper, we call a pair of instantiations a mode of a variable.
Figure~\ref{fig:double_modded} shows examples of the normalized \mk relations with modes inferred for the forward and backward directions.
We use superscript annotation for variables to represent their modes visually.
Notice the different order of conjuncts in the bodies of the \lstinline{add$^o$} relation in different modes.



