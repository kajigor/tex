\documentclass[xcolor=table, aspectratio=169]{beamer}
\usepackage{beamerthemesplit}
\usepackage{wrapfig}
\usetheme{SPbGU}
\usepackage{pdfpages}
\usepackage{amsmath}
\usepackage{amssymb}
\usepackage{cmap}
\usepackage[T2A]{fontenc}
\usepackage[utf8]{inputenc}
\usepackage[english]{babel}
\usepackage{indentfirst}
\usepackage{tikz}
\usetikzlibrary{shapes,arrows,automata,positioning,quotes,backgrounds,decorations.text,decorations.pathmorphing}
\usepackage{multirow}
\usepackage[noend]{algpseudocode}
\usepackage{algorithm}
\usepackage{algorithmicx}
\usepackage{fancyvrb}
\usepackage[linguistics]{forest}
\usepackage{listings}
\usepackage{multicol}
\usepackage{comment}
\usepackage{xspace}
\usepackage{adjustbox}
\usepackage{makecell}
\usepackage{ stmaryrd }
\usepackage{ulem}


\newcommand{\backupbegin}{
   \newcounter{finalframe}
   \setcounter{finalframe}{\value{framenumber}}
}
\newcommand{\backupend}{
   \setcounter{framenumber}{\value{finalframe}}
}

\newcommand{\makenote}[1]{\hfill \footnotesize{#1}}
\newcommand{\strikeoutnote}[1]{\makenote{\strikethrough{#1}}}
\newcommand{\strikethrough}[1]{\sout{#1}}

\newcommand{\lststrikethrough}[1]{\ttfamily\sout{#1}}

\newcolumntype{A}{>{\hb@xt@\z@\bgroup\hss}r<{\egroup}}
\newcolumntype{B}{>{\hb@xt@\z@\bgroup}l<{\hss\egroup}}

\setbeamertemplate{itemize item}[circle]
\setbeamertemplate{enumerate items}[circle]

\lstdefinelanguage{ocanren}{
keywords={where, case, run, conde, fresh, let, match, with, when, class, type,
object, method, of, rec, repeat, until, while, \begin{comment}not,\end{comment} do, done, as, val, inherit,
new, module, sig, deriving, datatype, struct, if, then, else, open, private, virtual, include, success, failure,
true, false, mplus},
sensitive=true,
commentstyle=\small\itshape\ttfamily,
keywordstyle=\color{blue},
identifierstyle=\ttfamily,
basewidth={0.5em,0.5em},
columns=flexible,
mathescape=true,
escapechar=~,
fontadjust=true,
literate={fun}{{$\lambda$}}1 {function}{function}8 {->}{{$\to$}}3 {<-}{{$\leftarrow$}}3 {===}{{$\equiv$}}1 {=/=}{{$\not\equiv$}}1 {|>}{{$\triangleright$}}3 {\\/}{{$\vee$}}2 {/\\}{{$\wedge$}}2 {^}{{$\uparrow$}}1,
morecomment=[s]{(*}{*)},
 moredelim=**[is][\color{red}]{@!}{@}
}

\tikzstyle{processTree} = [
  ->,
  sibling distance=15em,
  scale=0.6,
  every node/.style = {
    shape=rectangle,
    rounded corners=0.05cm,
    draw,
    align=center,
    minimum size=5mm,
    scale=0.6,},
  %level 1/.style={sibling distance=100em}
  ]


\tikzstyle{program} = [
  draw=black,
  thick,
  rectangle,
  rounded corners=1pt,
  inner sep=5pt,
  inner ysep=5pt
  ]

\tikzstyle{goal} = [
  draw=black,
  rectangle,
  rounded corners=1pt,
  inner ysep=0pt,
  ]

\tikzstyle{input} = [
  draw=none,
  rectangle,
  rounded corners=1pt,
  inner sep=2pt,
  inner ysep=2pt,
  fill=green!10,
  minimum height=5mm
  ]


\tikzstyle{transparent} = [
  draw=none,
  inner ysep=3pt
  ]

\lstset{
language=ocanren
}


\DeclareMathOperator{\Term}{\mathcal{T}}
\DeclareMathOperator{\FlatTerm}{\mathcal{FT}}
\DeclareMathOperator{\Var}{\mathbf{Var}}
\DeclareMathOperator{\Cons}{\mathcal{C}}
\DeclareMathOperator{\Kan}{\mathcal{G}}
\DeclareMathOperator{\Fresh}{\mathbf{Fresh}}
\DeclareMathOperator{\Delay}{\mathbf{Delay}}
\DeclareMathOperator{\Cll}{\mathbf{Call}}
\DeclareMathOperator{\Def}{\mathcal{D}}
\DeclareMathOperator{\Base}{\mathbf{Base}}
\DeclareMathOperator{\Conj}{\mathbf{Conj}}
\DeclareMathOperator{\free}{\mathbf{free}}
\DeclareMathOperator{\ground}{\mathbf{ground}}
\DeclareMathOperator{\In}{\mathbf{In}}
\DeclareMathOperator{\Out}{\mathbf{Out}}
\DeclareMathOperator{\Fun}{\mathcal{F}}
\DeclareMathOperator{\Rtrn}{\mathbf{Return}}
\DeclareMathOperator{\Bind}{\mathbf{Bind}}
\DeclareMathOperator{\Match}{\mathbf{Match}}
\DeclareMathOperator{\Sum}{\mathbf{Sum}}
\DeclareMathOperator{\Guard}{\mathbf{Guard}}
\DeclareMathOperator{\Gen}{\mathbf{Gen}}
\DeclareMathOperator{\Stream}{\mathit{Stream}}
\DeclareMathOperator{\vars}{vars}
\DeclareMathOperator{\inmode}{\mathtt{I}}
\DeclareMathOperator{\outmode}{\mathtt{O}}
\DeclareMathOperator{\whatmode}{\mathtt{?}}
% \DeclareMathOperator{\inmode}{g \rightarrow g}
% \DeclareMathOperator{\outmode}{f \rightarrow g}
% \DeclareMathOperator{\mode1}{mode}
% \DeclareMathOperator{\Mode1}{\mathcal{M}}
\newcommand{\KanN}{\mathcal{K}^{N}}
\newcommand{\tran}[1]{\left\llbracket #1 \right\rrbracket}
\newcommand{\LIST}[1]{\left[ #1 \right]}
\renewcommand{\emptyset}{\varnothing}
\newcommand{\mk}{\textsc{miniKanren}\xspace}
\newcommand{\ocaml}{\textsc{OCaml}\xspace}
\newcommand{\haskell}{\textsc{Haskell}\xspace}
\renewcommand{\and}{$\&$\xspace}
\newcommand{\rel}[2]{\texttt{#1}$^o$ #2}
\newcommand{\subst}[1]{$\langle$#1$\rangle$}
\newcommand{\sem}[1]{\llbracket #1 \rrbracket}

\beamertemplatenavigationsymbolsempty

\title[Functional Conversion for miniKanren]{A Case Study in Functional Conversion and Mode Inference in miniKanren}
\institute[JetBrains Research]{
JetBrains Research

\vspace{1cm}
PEPM @ POPL 2024
}

\author[Igor]{Kate Verbitskaia, \emph{Igor Engel}, Daniil Berezun}

\date{16.01.2024}

\definecolor{orange}{RGB}{179,36,31}

\begin{document}
{
\begin{frame}[fragile]
   \begin{center}
      \includegraphics[height=1.5cm]{pictures/jetbrainsResearch.pdf}
    \end{center}
  \titlepage
\end{frame}
}
\begin{frame}[fragile]
    \frametitle{Program inversion}
    \begin{center}
        Many complicated programgs are inverse of simple ones
    \end{center}
    \vfill
    \begin{center}
        Type inference or habitation is inverse of type checking
    \end{center}
    \vfill
    \begin{center}
        Program inversion: Given a program $f$, produce inverse porgram  $f^{-1}$
    \end{center}
    \vfill
    \begin{center}
        Given $\text{typecheck}(\text{program}, \text{types}) = true$, produce $\text{typecheck}^{-1}(\text{program}, true) = \text{types}$.
    \end{center}
\end{frame}
\begin{frame}[fragile]
    \frametitle{Relational inversion}
    \begin{center}
        A program is a relation between its inputs and outputs
    \end{center}
    \vfill
    \begin{center}
        miniKanren can evaluate relations in both directions
    \end{center}
    \vfill
    \begin{center}
        Write a simple verifier, convert to miniKanren, get a solver
    \end{center}
    \vfill
    \begin{center}
        Problem: miniKanren may be slow. So convert it back!
    \end{center}
\end{frame}

\begin{frame}[fragile]
  \frametitle{Relational Programming}

    \begin{center}
        Subset of logic programming, focus on pure relations
    \end{center}
    \vfill
    \begin{center}
        Extra-logical features (cuts, side-effects, search order manipulation) discouraged
    \end{center}
    \vfill
    \begin{center}
        Interleaving search guarantees that all answers are found
    \end{center}
\end{frame}

\begin{frame}[fragile]
  \frametitle{miniKanren}
    \begin{center}
        miniKanren is a simple relational language designed to be implemented as shallow embedding.
    \end{center}
    \begin{center}
\begin{tikzpicture}[
  scale=0.8,
  every node/.style = {
    shape=rectangle,
    rounded corners=0.05cm,
    draw,
    align=center,
    minimum size=5mm,
    scale=0.8,},
  node distance=1.3em,
  anchor=center
]
  \node[inner sep=10pt] (code) at (0,0) {\begin{lstlisting}
 let rec add$^o$ x y z = conde [
   (x === O /\ y === z);
   (fresh (x$_1$ z$_1$)
     (x === S x$_1$ /\
      add$^o$ x$_1$ y z$_1$ /\
      z === S z$_1$) ) ]
    \end{lstlisting}};
    \node (rel) [draw=none, above=of code, xshift=1cm] {relation};
    \draw [red,<-] ($(code.north)+(-0.45,-0.4)$) to [out=90,in=180] node[below, draw=none, red] {} ($(rel.west)+(0,0)$);
    \pause
    \node (disj) [draw=none, right=of code, yshift=2cm] {disjunction};
    \node (conj) [draw=none, right=of code, yshift=-0.5cm] {conjunction};
    \node (uni) [draw=none, left=of code, yshift=1.2cm] {term unification};
    \node (fresh) [draw=none, left=of code, yshift=-1cm] {variable introduction};
    \node (call) [draw=none, below=of code, xshift=-2cm] {relation call};

    \draw [red,<-] ($(code.south)+(-1.7,1.1)$) to [out=-180,in=90] node[below, draw=none, red] {} ($(call.north)+(0,0)$);

    \draw [red,<-] ($(code.west)+(1.65,0.8)$) to [out=150,in=-45] node[below, draw=none, red] {} ($(uni.east)+(0,0)$);

    \draw [red,<-] ($(code.west)+(1.4,0)$) to [out=230,in=0] node[below, draw=none, red] {} ($(fresh.east)+(0,0)$);


    \draw [red,<-] ($(code.east)+(-1.2,1.5)$) to [out=90,in=-180] node[below, draw=none, red] {} ($(disj.west)+(0,0)$);

    \draw [red,<-] ($(code.east)+(-2,-0.75)$) to [out=0,in=-180] node[below, draw=none, red] {} ($(conj.west)+(0,0)$);
    \draw [red,<-] ($(code.east)+(-2.65,-0.3)$) to [out=0,in=-190] node[below, draw=none, red] {} ($(conj.west)+(0,0)$);

\end{tikzpicture}
\end{center}

\end{frame}

\begin{frame}[fragile]
    \frametitle{Stream}
    \begin{center}
        \emph{Stream} is a list-like structure, representing nondetermistic computation with interleaving search
    \end{center}
    \begin{center}
        \[ \left[1, 2, 3\right] >>= f = f(1) \left<\mid\right> f(2) \left<\mid\right> f(3)\] 
        \[ \left[1, 2, 3\right] \left<\mid\right> \left[a, b, c\right] = \left[1, a, 2, b, 3, c\right] \] 
    \end{center}

    \begin{center}
        miniKanren implementation - \emph{Stream} of substitutions

        Conversion to functional language - $\emph{Stream}$ of values
    \end{center}
\end{frame}

\begin{frame}[fragile]
  \frametitle{Example: Addition in the Forward Direction}
\begin{columns}
  \begin{column}[t]{0.49\textwidth}
    \begin{figure}[!t]
  \centering
  \begin{minipage}{\columnwidth}
    \begin{lstlisting}[label={add},
                      %  caption={Addition relation},
                       captionpos=b,
                       frame=tb]
let rec add$^o$ x y z = conde [
  (x === O /\ y === z);
  (fresh (x' z')
    (x === S x' /\
     z === S z' /\
     add$^o$ x' y z') ) ]
    \end{lstlisting}
  \end{minipage}
\end{figure}
    \[ \text{add}^{\circ}\ 1\ 2\ z = \{z \to 3\}\]
  \end{column}
  \begin{column}[t]{0.49\textwidth}
    \begin{figure}[!t]
  \centering
  \begin{minipage}{\columnwidth}
    \begin{lstlisting}[label={add_x_y},
                      %  caption={Function for \lstinline{addo in in out} direction},
                       captionpos=b,
                       frame=tb]
addIIO :: Nat -> Nat -> Nat
addIIO x y =
  case x of
    O -> y
    S x$_1$ -> S (addIIO x$_1$ y)
    \end{lstlisting}
  \end{minipage}
\end{figure}

    \[ \text{addIIO}\ 1\ 2 = 3 \] 
  \end{column}
\end{columns}
\end{frame}

\begin{frame}[fragile]
  \frametitle{Functional Conversion}
\begin{center}
  Given a relation and a principal direction, construct a functional program that generates the same answers as \mk would
\end{center}

\vfill

\begin{center}
  Preserve the completeness of the search
\end{center}

\vfill

\begin{center}
Both inputs and outputs are expected to be ground
\end{center}

\vfill

\begin{center}
    Speed improvement: up to $3$ orders of magnitude on benchmark of mulltiplication
\end{center}

\end{frame}

\begin{frame}[fragile]
  \frametitle{Addition in the Backward Direction: Nondeterminism}
\begin{columns}
  \begin{column}[t]{0.49\textwidth}
    \begin{figure}[!t]
  \centering
  \begin{minipage}{\columnwidth}
    \begin{lstlisting}[label={add},
                      %  caption={Addition relation},
                       captionpos=b,
                       frame=tb]
let rec add$^o$ x y z = conde [
  (x === O /\ y === z);
  (fresh (x' z')
    (x === S x' /\
     z === S z' /\
     add$^o$ x' y z') ) ]
    \end{lstlisting}
  \end{minipage}
\end{figure}
    \begin{equation*}
        \begin{split}        
        \left<\text{fresh}\ x,y\ \text{in}\right.&\left.\text{add}^{\circ}\ x\ y\ 2\right> = [\\
            &\{x \to 0, y \to 2\},\\
            &\{x \to 1, y \to 1\},\\
            &\{x \to 2, y \to 0\}\\
            &]
        \end{split}
    \end{equation*}
  \end{column}
  \begin{column}[t]{0.49\textwidth}
    \begin{figure}[!t]
  \centering
  \begin{minipage}{\columnwidth}
    \begin{lstlisting}[frame=tb]
addOOI :: Nat -> Stream (Nat, Nat)
addOOI z =
  return (O, z) `mplus`
  case z of
    O -> Empty
    S z$_1$ -> do
      (x$_1$, y) <- addOOI z$_1$
      return (S x$_1$, y)
    \end{lstlisting}
  \end{minipage}
\end{figure}
    \begin{equation*}
        \text{addOOI}\ 2 = [\left( 0, 2 \right),\left( 1, 1 \right),\left( 2, 0 \right)]
    \end{equation*}
  \end{column}
\end{columns}
\end{frame}

\begin{frame}[fragile]
  \frametitle{Free Variables in Answers: Generators}
\begin{columns}
  \begin{column}[t]{0.49\textwidth}
    \begin{figure}[!t]
  \centering
  \begin{minipage}{\columnwidth}
    \begin{lstlisting}[label={add},
                      %  caption={Addition relation},
                       captionpos=b,
                       frame=tb]
let rec add$^o$ x y z = conde [
  (x === O /\ y === z);
  (fresh (x' z')
    (x === S x' /\
     z === S z' /\
     add$^o$ x' y z') ) ]
    \end{lstlisting}
  \end{minipage}
\end{figure}
    \[ \left<\text{fresh}\ y,z\ \text{in}\ \text{add}^{\circ}\ 1\ y\ z\right> = \left[\{z \to S\ y\}\right]\]
    \[ \text{genNat} = \left[0, 1, 2, 3, \ldots\right] \] 
    \[ \text{addIOO}\ 1 = \left[(0, 1), (1, 2), (2, 3), \ldots\right] \] 
  \end{column}
  \begin{column}[t]{0.49\textwidth}
    \begin{figure}[h]
\centering
\begin{minipage}{0.85\columnwidth}
  \begin{lstlisting}[frame=tb]
 addIOO $::$ Nat -> Stream (Nat, Nat)
 addIOO x =
   case x of
     O -> do
       z <- genNat
       return (z, z)
     S x$_1$ -> do
       (y, z$_1$) <- addIOO x$_1$
       return (y, S z$_1$)

 genNat $::$ Stream Nat
 genNat = Mature O (S <$\$$> genNat)
  \end{lstlisting}
\end{minipage}
\end{figure}
  \end{column}
\end{columns}
\end{frame}




\lstset{basicstyle=\small}

% \begin{frame}[fragile]
%   \frametitle{Predicates}
% \begin{columns}
%   \begin{column}[t]{0.48\textwidth}
%     \begin{figure}[!t]
  \centering
  \begin{minipage}{\columnwidth}
    \begin{lstlisting}[label={add},
                      %  caption={Addition relation},
                       captionpos=b,
                       frame=tb]
let rec add$^o$ x y z = conde [
  (x === O /\ y === z);
  (fresh (x' z')
    (x === S x' /\
     z === S z' /\
     add$^o$ x' y z') ) ]
    \end{lstlisting}
  \end{minipage}
\end{figure}
%   \end{column}
%   \begin{column}[t]{0.5\textwidth}
%     \begin{figure}[!t]
  \centering
  \begin{minipage}{\columnwidth}
    \begin{lstlisting}[label={add_x_y_z}, caption={Function for \lstinline{addo in in in} direction}, captionpos=b, frame=tb]
addXYZ :: Nat -> Nat -> Nat -> Stream ()
addXYZ x y z =
  case x of
    O | y == z -> return ()
      | otherwise -> Empty
    S x' ->
      case z of
        O -> Empty
        S z' -> addXYZ x' y z'
    \end{lstlisting}
  \end{minipage}
\end{figure}
%   \end{column}
% \end{columns}
% \end{frame}

\begin{frame}[fragile]
  \frametitle{Conversion Scheme}
  \begin{center}
    \begin{minipage}{0.4\textwidth}
      \begin{enumerate}
          \item Normalization
          \item Mode analysis
          \item Functional conversion
      \end{enumerate}
    \end{minipage}
  \end{center}
\end{frame}


\begin{frame}[fragile]
  \frametitle{Normalization: Flat Term}
\begin{center}

Eliminate nested constructors and repeated variables

\end{center}

  \[  \FlatTerm_{V} = V \cup \{\Cons \, x_0 \ldots \, x_{k} \mid x_{j}\in V, x_j - distinct \} \]

\vfill

\begin{center}
\begin{tabular}{rcl}
 $C\left( x, y \right) \equiv C\left( C\left( a, b \right), c \right)$ & $\iff$ &  $x \equiv C\left( a, b \right) \land y \equiv c$ \\
 $add^{\circ}(x, x, z)$ & $\iff$ & $add^{\circ}\left( x_1, x_2, z \right) \land x_1 \equiv x_2$
\end{tabular}
\end{center}

\end{frame}

\begin{frame}[fragile]
  \frametitle{Normalization: Goal}
\begin{center}
\begin{tabular}{rclll}
$\KanN_{V}$ & $=$ & $c_1 \vee \ldots \vee c_{n}$ & $c_{i}\in \Conj_{V}$ & normal form \\
$\Conj_{V}$ & $=$ & $g_1 \wedge \ldots \wedge g_n$ & $ g_{i}\in \Base_{V}$ & normal conjunction \\
$\Base_{V}$ & $=$ & $V \equiv \FlatTerm_{V}$ & & flat unification \\
            & $\mid$ & $R \, x_1 \ldots \, x_{k} $ & $ x_{j}\in V, x_j - distinct$ & flat call \\
\end{tabular}
\end{center}


\vfill

\strikeoutnote{Disjunctions within conjunctions}

\strikeoutnote{Empty disjunctions and conjunctions}

\strikeoutnote{Constructors as arguments of relation calls}

\end{frame}


\begin{frame}[fragile]
  \frametitle{Mode of a Variable}
\begin{center}

Instantiation describes whether at a given point a variable has a known value:

\begin{tabular}{rll}
  \emph{Ground} term & no fresh variables & \lstinline|Cons O (Cons (S O) Nil)| \\
  \emph{Free} variable & a fresh variable & \lstinline|_.0|
\end{tabular}

\vfill

Once we know that a variable is \emph{ground}, it stays \emph{ground} in later conjuncts
\end{center}

\vfill

\begin{center}
Mode is a transition between instantiations, associated with each use of a variable

\vfill

\begin{tabular}{rl}
  Mode \lstinline|I|: & \lstinline|ground| $\rightarrow$ \lstinline|ground| \\
  Mode \lstinline|O|: & \lstinline|free| $\rightarrow$ \lstinline|ground|
\end{tabular}

Taken together, modes represent data flow.
\end{center}

\vfill

\hfill \footnotesize Mercury uses more complicated modes

\end{frame}

\begin{frame}[fragile]
  \frametitle{Modded Unification Types}

\begin{center}
\begin{tabular}{rl}
  assignment & $x^{\outmode} \equiv \Term^{\inmode} $ \\
  assignment & $x^{\inmode}  \equiv y^{\outmode}    $ \\
  guard      & $x^{\inmode}  \equiv \Term^{\inmode} $ \\
  match      & $x^{\inmode}  \equiv \Term           $ \\
  generator  & $x^{\outmode} \equiv \Term           $
\end{tabular}
\end{center}

\hfill \footnotesize $\Term$ contains at least one $f$ variable
\end{frame}



\begin{frame}[fragile]
  \frametitle{Order in Conjunctions: Slow Version}
\begin{columns}
  \begin{column}[t]{0.4\textwidth}
    \begin{figure}[h]
  \centering
  \begin{minipage}{0.87\columnwidth}
    \begin{lstlisting}[frame=tb]
 let rec mult$^o$ x y z = conde [
   (fresh (x$_1$ r$_1$)
     (x === S x$_1$) /\
     (add$^o$ y r$_1$ z) /\
     (mult$^o$ x$_1$ y r$_1$));
  ...]
    \end{lstlisting}
  \end{minipage}
\end{figure}

  \end{column}
  \begin{column}[t]{0.55\textwidth}
    \begin{figure}[!t]
  \centering
  \begin{minipage}{\columnwidth}
    \begin{lstlisting}[label={mult_slow}, caption={Inefficient implementation of \lstinline{multo in in out} direciton}, captionpos=b, frame=tb]
multXY' :: Nat -> Nat -> Stream Nat
multXY' O y = return O
multXY' x O = return O
multXY' (S O) y = return y
multXY' x (S O) = return x
multXY' (S x') y = do
  (r', r) <- addX y
  multXYZ x' y r'
  return r

multXYZ :: Nat -> Nat -> Nat -> Stream ()
multXYZ O y O = return ()
multXYZ x O O = return ()
multXYZ (S O) y z | y == z = return ()
multXYZ x (S O) z | x == z = return ()
multXYZ (S x') y z = do
  z' <- multXY' x' y
  addXYZ y z' z
multXYZ _ _ _ = Empty
    \end{lstlisting}
  \end{minipage}
\end{figure}
  \end{column}
\end{columns}

\begin{center}
  Premature grounding of \lstinline{z$_1$} leads to the \emph{generate-and-test} behavior

  $\Omega(x!)$ complexity.
\end{center}

\end{frame}


\begin{frame}[fragile]
  \frametitle{Order in Conjunctions: Faster Version}

\begin{columns}
  \begin{column}[t]{0.4\textwidth}
    \begin{figure}[h]
  \centering
  \begin{minipage}{0.87\columnwidth}
    \begin{lstlisting}[frame=tb]
 let rec mult$^o$ x y z = conde [
   (fresh (x$_1$ r$_1$)
     (x === S x$_1$) /\
     (add$^o$ y r$_1$ z) /\
     (mult$^o$ x$_1$ y r$_1$));
  ...]
    \end{lstlisting}
  \end{minipage}
\end{figure}

  \end{column}
  \begin{column}[t]{0.55\textwidth}
    \begin{figure}[!t]
  \centering
  \begin{minipage}{\columnwidth}
    \begin{lstlisting}[label={mult_fast},
                      %  caption={Efficient implementation of \lstinline{multo in in out} direciton},
                       captionpos=b,
                       frame=tb]
multIIO :: Nat -> Nat -> Stream Nat
...
multIIO (S x') y = do
  r' <- multIIO x' y
  addXY y r'
    \end{lstlisting}
  \end{minipage}
\end{figure}

  \end{column}
\end{columns}
\begin{center}
    $O(xy)$ complexity, $10$x faster than relational version
\end{center}
\end{frame}

\begin{frame}[fragile]
  \frametitle{Mode Inference: Conjunction}
\begin{center}
Priority:
\end{center}

\vfill

\begin{center}
  \begin{minipage}{0.4\textwidth}
    \begin{enumerate}
      \item Guard
      \item Assignment
      \item Match
      \item Recursion, same direction
      \item Call, some ground
      \item Unification-generator
      \item Call, all free
    \end{enumerate}
  \end{minipage}
\end{center}


\end{frame}

\begin{frame}[fragile]
  \frametitle{Functional Conversion: Intermediate Language}
\begin{center}
\begin{tabular}{lcll}
    $\Fun_{V}$ & $=$ & $\Sum\LIST{\Fun_{V}}$ & interleaving\\
               & $\mid$ & $\Bind\LIST{\left(\LIST{V}, \Fun_{V}\right)} $ & monadic bind on streams\\
               & $\mid$ & $\Rtrn \LIST{\Term_{V}}$ & return a tuple of terms\\
               & $\mid$ & $\Guard\left( V, \Term_{V}\right)$ & equality check\\
               & $\mid$ &  $\Match_{V} \left( \Term_{V}, \Fun_{V} \right)$& match a variable against a pattern\\
               & $\mid$ & $R_{i}(\LIST{V}, \LIST{G})$ & function call\\
               & $\mid$ & $\Gen_{G}$ & generator
\end{tabular}
\end{center}
\end{frame}

\begin{frame}[fragile]
  \frametitle{Functional Conversion into Intermediate Language}
\begin{center}
\begin{tabular}{rcl}
  Disjunction   & $\rightarrow$ & $\Sum\LIST{\Fun_{V}}$ \\ && \\
  Conjunction   & $\rightarrow$ & $ \Bind\LIST{\left(\LIST{V}, \Fun_{V}\right)}$ \\ && \\
  Relation call & $\rightarrow$ & $ R_{i}(\LIST{V}, \LIST{G})$ \\ && \\
  Unification   & $\rightarrow$ & $\Rtrn \LIST{\Term_{V}}$ \\
                & $|$           & $\Match_{V} \left( \Term_{V}, \Fun_{V} \right)$ \\
                & $|$           & $\Guard\left( V, \Term_{V}\right)$ \\
                & $|$           & $\Gen_{G}$
\end{tabular}
\end{center}
\end{frame}


\begin{frame}[fragile]
  \frametitle{Functional Conversion: Generators}
\begin{center}
  In the untyped miniKanren it is only possible to generate \emph{all terms}
\end{center}

\vfill

\begin{center}
  Instead pass generators to functions as additional arguments


  It is up to the user what generator to pass
\end{center}

\begin{figure}[!t]
  \centering
  \begin{minipage}{\columnwidth}
    \begin{lstlisting}[frame=tb]
addIOO :: Nat -> Stream Nat -> Stream (Nat, Nat)
addIOO x gen$_z$ = case x of
  O -> do
    z <- gen$_z$
    return (z, z)
  S x$_1$ -> do
    (y, z$_1$) <- addIOO x$_1$ gen$_z$
    return (y, S z$_1$)
    \end{lstlisting}
  \end{minipage}
\end{figure}

\end{frame}

\begin{frame}[fragile]
  \frametitle{Functional Conversion: Generators}
\begin{center}
We pass a generator for every variable in \emph{rhs} of a unification-generator
\end{center}

\begin{center}
Generators used in calls should be passed to the parent function
\end{center}

\begin{center}
In a typed version, it should be possible to automatically derive generators from types
\end{center}

    \begin{figure}[!t]
  \centering
  \begin{minipage}{\columnwidth}
    \begin{lstlisting}[label={mult_x_gen},
                      %  caption={Efficient implementation of \lstinline{multo in in out} direciton},
                       captionpos=b,
                       frame=tb]
multOIO :: Nat -> Stream Nat -> Stream Nat
multOIO y gen_add$_z$ =
    return (O, O) `mplus`
    do
        (z$_1$, z) <- addIOO y gen_add$_z$
        x <- muloOII y z$_1$
        return (S x, z)
    \end{lstlisting}
  \end{minipage}
\end{figure}

\end{frame}

\begin{frame}[fragile]
  \frametitle{Functional Conversion into the Target Languages}
\begin{columns}
  \begin{column}[t]{0.45\textwidth}

\begin{center}
  \haskell
\end{center}

\begin{center}
  TemplateHaskell to generate code
\end{center}

\begin{center}
  Stream monad
\end{center}

\begin{center}
  do-notation
\end{center}

\begin{center}

\end{center}

  \end{column}
  \begin{column}[t]{0.45\textwidth}
\begin{center}
  \ocaml
\end{center}
\begin{center}
  Hand-crafted (not so) pretty-printer
\end{center}

\begin{center}
  Stream monad
\end{center}

\begin{center}
  let*
\end{center}

\begin{center}
  Taking extra care to ensure laziness
\end{center}

  \end{column}
\end{columns}

\end{frame}
\begin{frame}[fragile]
  \frametitle{Relational Sort}
\begin{columns}
  \begin{column}[t]{0.49\textwidth}
    \begin{figure}[h]
  \centering
  \begin{minipage}{0.87\columnwidth}
    \begin{lstlisting}[frame=tb]
 let rec sort$^o$ x y = conde [
    (x === [] /\ y === []);
    (fresh (s xs xs$_1$)
      y === s :: xs$_1$ /\
      smallest$^o$ x s xs /\
      sort$^o$ xs xs$_1$)]
    \end{lstlisting}
  \end{minipage}
\end{figure}


    \vfill

    \begin{center}
      Only good for sorting:

      \lstinline{run q (sort$^o$ xs q)}
    \end{center}

  \end{column}
  \begin{column}[t]{0.49\textwidth}
    \begin{figure}[h]
  \centering
  \begin{minipage}{0.87\columnwidth}
    \begin{lstlisting}[frame=tb]
 let rec sort$^o$ x y = conde [
    (x === [] /\ y === []);
    (fresh (s xs xs$_1$)
      y === s :: xs$_1$ /\
      sorto xs xs$_1$ /\
      smallesto x s xs)]
    \end{lstlisting}
  \end{minipage}
\end{figure}


    \vfill

    \begin{center}
      Only good permutation generation:

      \lstinline{run q (sort$^o$ q xs)}
    \end{center}


  \end{column}
\end{columns}
\end{frame}

\begin{frame}[fragile]
  \frametitle{Relational Sort: Sorting}
    \begin{table}[]
\begin{tabular}{l||c|r||r}
                & \multicolumn{2}{c||}{Relation} & \multicolumn{1}{c}{\multirow[t]{2}{*}{Function}} \\ \cline{2-3}
                & \multicolumn{1}{l|}{\begin{tabular}[c]{@{}l@{}}\texttt{sorto}\\ \texttt{smallesto}\end{tabular}} & \multicolumn{1}{l||}{\begin{tabular}[c]{@{}l@{}}\texttt{smallesto}\\ \text{sorto}\end{tabular}} & \multicolumn{1}{c}{}                            \\ \hline
\texttt{{[}3;2;1;0{]}}   & \multicolumn{1}{r|}{0.077s}                                                    & 0.004s                                                                         & 0.000s                                          \\
\texttt{{[}4;3;2;1;0{]}} & \timeout timeout                                                                       & 0.005s                                                                         & 0.000s                                          \\
\texttt{{[}31;...;0{]}}  & \timeout timeout                                                                       & 1.058s                                                                         & 0.006s                                          \\
\texttt{{[}262;...;0{]}} & \timeout timeout                                                                       & \multicolumn{1}{c||}{\timeout timeout}                                                   & 1.045s
\end{tabular}
\end{table}


\end{frame}


\begin{frame}[fragile]
  \frametitle{Relational Sort: Generating Permutations}
    \begin{table}[]
\begin{tabular}{l||c|r||r}
                & \multicolumn{2}{c||}{Relation} & \multicolumn{1}{c}{\multirow[t]{2}{*}{Function}} \\ \cline{2-3}
                & \multicolumn{1}{l|}{\begin{tabular}[c]{@{}l@{}}\texttt{smallesto}\\ \texttt{sorto}\end{tabular}} & \multicolumn{1}{l||}{\begin{tabular}[c]{@{}l@{}}\texttt{sorto}\\ \text{smallesto}\end{tabular}} & \multicolumn{1}{c}{}                            \\ \hline
\texttt{{[}0;1;2{]}}   & \multicolumn{1}{r|}{0.013s}                                                    & 0.004s                                                                         & 0.004s                                          \\
\texttt{{[}0;1;2;3{]}} & timeout                                                                       & 0.005s                                                                         & 0.005s                                          \\
\texttt{{[}0;...;6{]}}  & timeout                                                                       & 0.999s                                                                         & 0.021s                                          \\
\texttt{{[}0;...;8{]}} & timeout                                                                       & \multicolumn{1}{c||}{timeout}                                                   & 1.543s
\end{tabular}
\end{table}

\end{frame}

\begin{frame}[fragile]
  \frametitle{Conclusion}
Conclusion
  \begin{itemize}
    \item We presented a functional conversion scheme
    \item The conversion speeds up implementations considerably
    \item We implemented the conversion scheme in Haskell
  \end{itemize}

\vfill

We are currently working on
  \begin{itemize}
    \item Determinism check
    \item Integration with partial deduction
    \item Integration into the framework of using relational interpreters for solving
  \end{itemize}
\end{frame}





\appendix
\backupbegin

\begin{frame}[fragile]
  \frametitle{\lstinline[basicstyle=\Large]{Maybe} for Semi-Determinism}
\begin{center}
  \begin{minipage}{0.43\textwidth}
    \begin{figure}[!t]
  \centering
  \begin{minipage}{\columnwidth}
    \begin{lstlisting}[frame=tb]
muloOII :: Nat -> Nat -> Stream Nat
muloOII x1 x2 = zero `mplus` positive
  where
    zero = do
      guard (x2 == O)
      return O
    positive = do
      x4 <- addoIOI x1 x2
      S <$\$$> muloOII x1 x4
    \end{lstlisting}
  \end{minipage}
\end{figure}
  \end{minipage}
\end{center}
\end{frame}


\begin{frame}[noframenumbering]
  \frametitle{\lstinline[basicstyle=\Large]{Maybe} for Semi-Determinism}
  \begin{center}
  \begin{minipage}{0.43\textwidth}
    \lstset{moredelim=[is][\sout]{|}{|}}

\begin{figure}[!t]
  \centering
  \begin{minipage}{\columnwidth}
    \begin{lstlisting}[]
 ~\texttt{muloOII $::$ Nat $\to$ Nat $\to$ Maybe Nat}~
 ~\lststrikethrough{muloOII $::$ Nat $\to$ Nat $\to$ Stream Nat}~
 muloOII x1 x2 =
     zero <$\mid$> positive
   where
     zero = do
       guard (x2 == O)
       return O
     positive = do
       x4 <- addoIOI x1 x2
       S <$\$$> muloOII x1 x4
    \end{lstlisting}
  \end{minipage}
\end{figure}

  \end{minipage}
\end{center}
\end{frame}

\begin{frame}[fragile]
  \frametitle{Need for Determinism Check}
\begin{center}
  Simply replacing the type of monad from \texttt{Stream} to \texttt{Maybe} improves performance 10 times for~relations on natural numbers
\end{center}

\begin{center}
  Pure (no monad) version is even faster
\end{center}

\vfill

\begin{center}
  Use determinism check to figure out when replacing \texttt{Stream} is feasible
\end{center}

\end{frame}

\begin{frame}[fragile]
  \frametitle{Need for Partial Deduction}

\begin{center}
Running a relational interpreter backwards fixes some arguments
\end{center}

\begin{center}
\begin{minipage}{0.3\textwidth}
  \lstinline{run q (eval$^o$ q true)}
\end{minipage}

\vfill

\begin{center}
  Augmenting functional conversion with partial deduction must be beneficial
\end{center}
\end{center}


\end{frame}
\begin{frame}[fragile]
    \frametitle{Functional Conversion: Example}
    \begin{columns}
        \begin{column}[t]{0.45\textwidth}
            \begin{figure}[!t]
  \centering
  \begin{minipage}{\columnwidth}
    \begin{lstlisting}[label={add},
                      %  caption={Addition relation},
                       captionpos=b,
                       frame=tb]
let rec add$^o$ x y z = conde [
  (x === O /\ y === z);
  (fresh (x' z')
    (x === S x' /\
     z === S z' /\
     add$^o$ x' y z') ) ]
    \end{lstlisting}
  \end{minipage}
\end{figure}
        \end{column}
        \begin{column}[t]{0.55\textwidth}
            \begin{figure}[h]
\centering
\begin{minipage}{0.9\columnwidth}
  \begin{lstlisting}[frame=tb]
data Term = O | S Term
addoIIO :: Term -> Term -> Stream Term
addoIIO x y = msum [
    do {
        guard (x == O);
        z <- return y;
        return z
    },
    do {
        S $x_1$ <- return x;
        $z_1$ <- addoIIO $x_1$ y;
        z <- return (S $z_1$);
        return z
    }
]
  \end{lstlisting}
\end{minipage}
\end{figure}


        \end{column}
    \end{columns}
\end{frame}

\backupend

\end{document}
