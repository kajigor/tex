\documentclass[xcolor=table]{beamer}
\usepackage{beamerthemesplit}
\usepackage{wrapfig}
\usetheme{SPbGU}
\usepackage{pdfpages}
\usepackage{amsmath}
\usepackage{amssymb}
\usepackage{cmap}
\usepackage[T2A]{fontenc}
\usepackage[utf8]{inputenc}
\usepackage[english]{babel}
\usepackage{indentfirst}
\usepackage{tikz}
\usetikzlibrary{shapes,arrows,automata,positioning,quotes,backgrounds,decorations.text,decorations.pathmorphing}
\usepackage{multirow}
\usepackage[noend]{algpseudocode}
\usepackage{algorithm}
\usepackage{algorithmicx}
\usepackage{fancyvrb}
\usepackage[linguistics]{forest}
\usepackage{listings}
\usepackage{multicol}
\usepackage{comment}
\usepackage{xspace}
\usepackage{adjustbox}
\usepackage{makecell}
\usepackage{ stmaryrd }

\setbeamertemplate{itemize items}[circle]
\setbeamertemplate{enumerate items}[circle]

\lstdefinelanguage{ocanren}{
keywords={run, conde, fresh, let, match, with, when, class, type,
object, method, of, rec, repeat, until, while, \begin{comment}not,\end{comment} do, done, as, val, inherit,
new, module, sig, deriving, datatype, struct, if, then, else, open, private, virtual, include, success, failure,
true, false},
sensitive=true,
commentstyle=\small\itshape\ttfamily,
keywordstyle=\textbf,%\ttfamily\underline,
identifierstyle=\ttfamily,
basewidth={0.5em,0.5em},
columns=fixed,
mathescape=true,
fontadjust=true,
literate={fun}{{$\lambda$}}1 {function}{function}8 {->}{{$\to$}}3 {<-}{{$\leftarrow$}}3 {===}{{$\equiv$}}1 {=/=}{{$\not\equiv$}}1 {|>}{{$\triangleright$}}3 {\\/}{{$\vee$}}2 {/\\}{{$\wedge$}}2 {^}{{$\uparrow$}}1,
morecomment=[s]{(*}{*)},
 moredelim=**[is][\color{red}]{@!}{@}
}

\tikzstyle{processTree} = [
  ->,
  sibling distance=15em,
  scale=0.6,
  every node/.style = {
    shape=rectangle,
    rounded corners=0.05cm,
    draw,
    align=center,
    minimum size=5mm,
    scale=0.6,},
  %level 1/.style={sibling distance=100em}
  ]


\tikzstyle{program} = [
  draw=black,
  thick,
  rectangle,
  rounded corners=1pt,
  inner sep=5pt,
  inner ysep=5pt
  ]

\tikzstyle{goal} = [
  draw=black,
  rectangle,
  rounded corners=1pt,
  inner ysep=0pt,
  ]

\tikzstyle{input} = [
  draw=none,
  rectangle,
  rounded corners=1pt,
  inner sep=2pt,
  inner ysep=2pt,
  fill=green!10,
  minimum height=5mm
  ]


\tikzstyle{transparent} = [
  draw=none,
  inner ysep=3pt
  ]

\lstset{
language=ocanren
}


\DeclareMathOperator{\Term}{\mathcal{T}}
\DeclareMathOperator{\FlatTerm}{\mathcal{FT}}
\DeclareMathOperator{\Var}{\mathbf{Var}}
\DeclareMathOperator{\Cons}{\mathcal{C}}
\DeclareMathOperator{\Kan}{\mathcal{G}}
\DeclareMathOperator{\Fresh}{\mathbf{Fresh}}
\DeclareMathOperator{\Delay}{\mathbf{Delay}}
\DeclareMathOperator{\Cll}{\mathbf{Call}}
\DeclareMathOperator{\Def}{\mathcal{D}}
\DeclareMathOperator{\Base}{\mathbf{Base}}
\DeclareMathOperator{\Conj}{\mathbf{Conj}}
\DeclareMathOperator{\free}{\mathbf{free}}
\DeclareMathOperator{\ground}{\mathbf{ground}}
\DeclareMathOperator{\In}{\mathbf{In}}
\DeclareMathOperator{\Out}{\mathbf{Out}}
\DeclareMathOperator{\Fun}{\mathcal{F}}
\DeclareMathOperator{\Rtrn}{\mathbf{Return}}
\DeclareMathOperator{\Bind}{\mathbf{Bind}}
\DeclareMathOperator{\Match}{\mathbf{Match}}
\DeclareMathOperator{\Sum}{\mathbf{Sum}}
\DeclareMathOperator{\Guard}{\mathbf{Guard}}
\DeclareMathOperator{\Gen}{\mathbf{Gen}}
\DeclareMathOperator{\Stream}{\mathit{Stream}}
\DeclareMathOperator{\vars}{vars}
\DeclareMathOperator{\inmode}{in}
\DeclareMathOperator{\outmode}{out}
% \DeclareMathOperator{\inmode}{g \rightarrow g}
% \DeclareMathOperator{\outmode}{f \rightarrow g}
% \DeclareMathOperator{\mode1}{mode}
% \DeclareMathOperator{\Mode1}{\mathcal{M}}
\newcommand{\KanN}{\mathcal{K}^{N}}
\newcommand{\tran}[1]{\left\llbracket #1 \right\rrbracket}
\newcommand{\LIST}[1]{\left[ #1 \right]}
\renewcommand{\emptyset}{\varnothing}
\newcommand{\mk}{\textsc{miniKanren}\xspace}
\renewcommand{\and}{$\&$\xspace}
\newcommand{\rel}[2]{\texttt{#1}$^o$ #2}
\newcommand{\subst}[1]{$\langle$#1$\rangle$}
\newcommand{\sem}[1]{\llbracket #1 \rrbracket}

\beamertemplatenavigationsymbolsempty

\title[Functional Conversion for microKanren]{Semi-Automated Direction-Driven Functional Conversion}
\institute[JetBrains Research]{
JetBrains Research, Programming Languages and Tools Lab

\vspace{1cm}

miniKanren workshop @ ICFP 2023
}

\author[Kate, Igor]{Kate Verbitskaia, Igor Engel, Daniil Berezun}

\date{08.09.2023}

\definecolor{orange}{RGB}{179,36,31}

\begin{document}
{
\begin{frame}[fragile]
   \begin{center}
      \includegraphics[height=1.5cm]{pictures/jetbrainsResearch.pdf}
    \end{center}
  \titlepage
\end{frame}
 }


\begin{frame}[fragile]
  \frametitle{Relational Programming}
\begin{center}
One relation to solve many problems
\end{center}

\begin{center}
Nondeterminism
\end{center}

\begin{center}
Completeness of search
\end{center}

\end{frame}

\begin{frame}[fragile]
  \frametitle{Relational Conversion: Easy}
Given a function
\begin{figure}[!t]
  \centering
  \begin{minipage}{\columnwidth}
    \begin{lstlisting}[frame=tb]
let rec add x y =
  match x with
  | O -> y
  | S x$_1$ -> S (add x$_1$ y)
    \end{lstlisting}
  \end{minipage}
\end{figure}

generate miniKanren relation
\begin{figure}[!t]
  \centering
  \begin{minipage}{\columnwidth}
    \begin{lstlisting}[label={add},
                      %  caption={Addition relation},
                       captionpos=b,
                       frame=tb]
let rec add$^o$ x y z = conde [
  (x === O /\ y === z);
  (fresh (x' z')
    (x === S x' /\
     z === S z' /\
     add$^o$ x' y z') ) ]
    \end{lstlisting}
  \end{minipage}
\end{figure}
\end{frame}


\begin{frame}[fragile]
  \frametitle{Principal Directions of \mk Relations}
\begin{center}
  Every argument of a relation can be either \lstinline{in} or \lstinline{out}
\end{center}

\begin{center}
  For addition relation \lstinline{add$^o$ x y z} there are 8 directions:
\end{center}

\begin{itemize}
  \item \emph{Forward} direction: \lstinline{add$^o$ in in out} --- addition
  \item \emph{Backward} direction: \lstinline{add$^o$ out out in} --- decomposition
  \item \emph{Predicate}: \lstinline{add$^o$ in in in}
  \item \emph{Generator}: \lstinline{add$^o$ out out out}
  \item \lstinline{add$^o$ in out in} --- subtraction
  \item \lstinline{add$^o$ out in in} --- subtraction
  \item \lstinline{add$^o$ out in out}
  \item \lstinline{add$^o$ in out out}
\end{itemize}
\end{frame}


\begin{frame}[fragile]
  \frametitle{Each Direction is a Function \pause (kind of)}
Straightforward functions:
\begin{itemize}
  \item \emph{Forward} direction: \lstinline{add$^o$ in in out} --- addition
  \item \lstinline{add$^o$ in out in} --- subtraction
  \item \lstinline{add$^o$ out in in} --- subtraction
  \item \emph{Predicate}: \lstinline{add$^o$ in in in}
\end{itemize}

\vfill

Relations:
\begin{itemize}
  \item \emph{Backward} direction: \lstinline{add$^o$ out out in} --- decomposition
  \item \emph{Generator}: \lstinline{add$^o$ out out out}
  \item \lstinline{add$^o$ out in out}
  \item \lstinline{add$^o$ in out out}
\end{itemize}
These relations are functions which return multiple answers (list monad)
\end{frame}

\begin{frame}[fragile]
  \frametitle{\mk Comes with an Overhead}
  \begin{center}
    Unifications
  \end{center}

  \begin{center}
    Occurs-check
  \end{center}

  \begin{center}
    Scheduling complexity
  \end{center}
\end{frame}

\begin{frame}[fragile]
  \frametitle{Functional Conversion}
\begin{center}
  Given a relation and a principal direction, construct a functional program which generates the same answers as \mk would
\end{center}

\vfill

\begin{center}
  Preserve completeness of the search
\end{center}

\vfill

\begin{center}
Both inputs and outputs are expected to be ground
\end{center}
\end{frame}

\lstset{basicstyle=\small}

\begin{frame}[fragile]
  \frametitle{Example: Addition in the Forward Direction}
\begin{figure}[!t]
  \centering
  \begin{minipage}{\columnwidth}
    \begin{lstlisting}[label={add},
                      %  caption={Addition relation},
                       captionpos=b,
                       frame=tb]
let rec add$^o$ x y z = conde [
  (x === O /\ y === z);
  (fresh (x' z')
    (x === S x' /\
     z === S z' /\
     add$^o$ x' y z') ) ]
    \end{lstlisting}
  \end{minipage}
\end{figure}

\begin{figure}[!t]
  \centering
  \begin{minipage}{\columnwidth}
    \begin{lstlisting}[label={add_x_y},
                      %  caption={Function for \lstinline{addo in in out} direction},
                       captionpos=b,
                       frame=tb]
addIIO :: Nat -> Nat -> Nat
addIIO x y =
  case x of
    O -> y
    S x$_1$ -> S (addIIO x$_1$ y)
    \end{lstlisting}
  \end{minipage}
\end{figure}

\end{frame}

\begin{frame}[fragile]
  \frametitle{Addition in the Backward Direction: Nondeterminism}
\begin{figure}[!t]
  \centering
  \begin{minipage}{\columnwidth}
    \begin{lstlisting}[label={add},
                      %  caption={Addition relation},
                       captionpos=b,
                       frame=tb]
let rec add$^o$ x y z = conde [
  (x === O /\ y === z);
  (fresh (x' z')
    (x === S x' /\
     z === S z' /\
     add$^o$ x' y z') ) ]
    \end{lstlisting}
  \end{minipage}
\end{figure}

\begin{figure}[!t]
  \centering
  \begin{minipage}{\columnwidth}
    \begin{lstlisting}[frame=tb]
addOOI :: Nat -> Stream (Nat, Nat)
addOOI z =
  return (O, z) `mplus`
  case z of
    O -> Empty
    S z$_1$ -> do
      (x$_1$, y) <- addOOI z$_1$
      return (S x$_1$, y)
    \end{lstlisting}
  \end{minipage}
\end{figure}
\end{frame}

\begin{frame}[fragile]
  \frametitle{Free Variables in Answers: Generators}
\input{fig/add/add_one_line.tex}
% \end{frame}

% \begin{frame}[fragile]
%   \frametitle{Free Variables in Answers: Generators}
\begin{figure}[h]
\centering
\begin{minipage}{0.85\columnwidth}
  \begin{lstlisting}[frame=tb]
 addIOO $::$ Nat -> Stream (Nat, Nat)
 addIOO x =
   case x of
     O -> do
       z <- genNat
       return (z, z)
     S x$_1$ -> do
       (y, z$_1$) <- addIOO x$_1$
       return (y, S z$_1$)

 genNat $::$ Stream Nat
 genNat = Mature O (S <$\$$> genNat)
  \end{lstlisting}
\end{minipage}
\end{figure}
\end{frame}

\begin{frame}[fragile]
  \frametitle{Predicates}
  \begin{figure}[!t]
  \centering
  \begin{minipage}{\columnwidth}
    \begin{lstlisting}[label={add},
                      %  caption={Addition relation},
                       captionpos=b,
                       frame=tb]
let rec add$^o$ x y z = conde [
  (x === O /\ y === z);
  (fresh (x' z')
    (x === S x' /\
     z === S z' /\
     add$^o$ x' y z') ) ]
    \end{lstlisting}
  \end{minipage}
\end{figure}

  \begin{figure}[!t]
  \centering
  \begin{minipage}{\columnwidth}
    \begin{lstlisting}[label={add_x_y_z}, caption={Function for \lstinline{addo in in in} direction}, captionpos=b, frame=tb]
addXYZ :: Nat -> Nat -> Nat -> Stream ()
addXYZ x y z =
  case x of
    O | y == z -> return ()
      | otherwise -> Empty
    S x' ->
      case z of
        O -> Empty
        S z' -> addXYZ x' y z'
    \end{lstlisting}
  \end{minipage}
\end{figure}
\end{frame}



\begin{frame}[fragile]
  \frametitle{Conversion Scheme}
  \begin{itemize}
    \item Normalization
    \item Mode analysis
    \item Functional conversion
  \end{itemize}
\end{frame}


\begin{frame}[fragile]
  \frametitle{Normalization: Flat Term}

Flat terms: a var or a constructor which takes \emph{distinct} vars as arguments:

  \[  \FlatTerm_{V} = V \cup \{\Cons_{i}\left( x_1, \ldots, x_{k_{i}} \right) \mid x_{j}\in V, x_j - distinct \} \]

Examples:

\begin{equation*}
\begin{split}
 C\left( x_1, x_2 \right) \equiv C\left( C\left( y_1, y_2 \right), y_3 \right) & \iff x_1 \equiv C\left( y_1, y_2 \right) \land x_2 \equiv y_3 \\
 C\left( C\left( x_1, x_2 \right), x_3 \right) \equiv C\left( C\left( y_1, y_2 \right), y_3 \right) & \iff x_1 \equiv y_1 \land x_2 \equiv y_2 \land x_3 \equiv y_3 \\
 x \equiv C\left( y, y \right) & \iff x \equiv C\left( y_1, y_2 \right)\land y_1 \equiv y_2
\end{split}
\end{equation*}
\end{frame}



\begin{frame}[fragile]
  \frametitle{Normalization: Goal}
\begin{tabular}{lcll}
$\KanN_{V}$ & $=$ & $\bigvee\left( c_1, \ldots, c_{n} \right), c_{i}\in \Conj_{V}$ & normal form \\
$\Conj_{V}$ & $=$ & $\bigwedge\left( g_1, \ldots, g_n \right), g_{i}\in \Base_{V}$ & normal conjunction \\
$\Base_{V}$ & $=$ & $V \equiv \FlatTerm_{V}$ & flat unification \\
            & $\mid$ & $R_{i}\left( x_1, \ldots, x_{k_{i}} \right), x_{j}\in V, x_j - distinct$ & flat call\\
\end{tabular}
\end{frame}


\begin{frame}[fragile]
  \frametitle{Mode of a Variable}
Mode of a variable: mapping between its instantiations

\vfill

\emph{Ground} term contains no fresh variables

\emph{Free} variable: a fresh variable, no info about its instantiation

\vfill

Once we know that a variable is \emph{ground}, it stays \emph{ground} in subsequent conjuncts

\vfill

Mode $in$: $ground \rightarrow ground$

Mode $out$: $free \rightarrow ground$

\vfill

Mercury uses more complicated modes

\end{frame}

\begin{frame}[fragile]
  \frametitle{Modded Goal}
Assign mode to every variable, make sure they are consistent
\end{frame}

\begin{frame}[fragile]
  \frametitle{Modded Unification Types}


\begin{equation*}
\begin{split}
  assignment &: x^{\outmode}  \equiv \Term^{\inmode}$ and $x^{\inmode} \equiv y^{\outmode} \\
  guard &: x^{\inmode} \equiv \Term^{\inmode} \\
  match &: x^{\inmode} \equiv \Term$ ($\Term$ contains both \emph{in} and \emph{out} variables) \\
  generator &: x^{\outmode} \equiv \Term
\end{split}
\end{equation*}
\end{frame}

\begin{frame}[fragile]
  \frametitle{Mode Inference: Initialization }

\begin{itemize}
  \item Input variables: $ground \rightarrow ground $
  \item Output variables: $free \rightarrow ground$
  \item Other variables: $free \rightarrow \ ?$
\end{itemize}

\begin{figure}[!t]
  \centering
  \begin{minipage}{\columnwidth}
    \begin{lstlisting}[frame=tb]
let rec add$^o$ x$^{g \to g}$ y$^{g \to g}$ z$^{f \to g}$ = conde
  (x$^{g \to g}$  ===    O /\ y$^{g \to g}$ ===     z$^{f \to g}$);
  (x$^{g \to g}$ ===     S x$_1^{f \to ?}$ /\
   add$^o$ x$_1^{f \to ?}$ y$^{g \to g}$ z$_1^{f \to ?}$ /\
   z$^{f \to g}$ ===     S z$_1^{f \to ?}$)
    \end{lstlisting}
  \end{minipage}
\end{figure}
\end{frame}

\begin{frame}[fragile]
  \frametitle{Mode Inference: Disjunction }
Run inference on each disjunct independently

\begin{figure}[!t]
  \centering
  \begin{minipage}{0.65\columnwidth}
    \begin{lstlisting}[frame=tb]
 x$^{\inmode}$  === O /\ y$^{\inmode}$ ===  z$^{\outmode}$
    \end{lstlisting}
  \end{minipage}
\end{figure}
\begin{figure}[!t]
  \centering
  \begin{minipage}{\columnwidth}
    \begin{lstlisting}[label={add},
                      %  caption={Addition relation},
                       captionpos=b,
                       frame=tb]
  x$^{g \to g}$ ===     S x$_1^{f \to ?}$ /\
  add$^o$ x$_1^{f \to ?}$ y$^{g \to g}$ z$_1^{f \to ?}$ /\
  z$^{f \to g}$ ===     S z$_1^{f \to ?}$
    \end{lstlisting}
  \end{minipage}
\end{figure}
\end{frame}


\begin{frame}[fragile]
  \frametitle{Mode Inference: Unification}
Propagate the groundness information according to the 4 types of modded unifications


\begin{figure}[!t]
  \centering
  \begin{minipage}{\columnwidth}
    \begin{lstlisting}[frame=tb]
  x$^{g \to g}$ ===     S x$_1^{f \to ?}$ $\Rightarrow$ x$^{g \to g}$ ===          S x$_1^{f \to g}$
    \end{lstlisting}
  \end{minipage}
\end{figure}
\begin{figure}[!t]
  \centering
  \begin{minipage}{\columnwidth}
    \begin{lstlisting}[label={add},
                      %  caption={Addition relation},
                       captionpos=b,
                       frame=tb]
  z$^{f \to g}$ ===     S z$_1^{f \to ?}$ $\Rightarrow$ z$^{f \to g}$ ===          S z$_1^{f \to g}$
    \end{lstlisting}
  \end{minipage}
\end{figure}
\end{frame}

\begin{frame}[fragile]
  \frametitle{Mode Inference: Conjunction}
Pick a conjunct according to the priority, propagate groundness
\begin{itemize}
  \item Always deterministic
    \item Guard
    \item Assignment
    \item Matches
    \item Calls with some ground arguments
    \item Unifications-generators
    \item Calls with all free arguments

\end{itemize}
\end{frame}

\begin{frame}[fragile]
  \frametitle{Mode Inference: Conjunction}
\begin{figure}[!t]
  \centering
  \begin{minipage}{\columnwidth}
    \begin{lstlisting}[label={add},
                      %  caption={Addition relation},
                       captionpos=b,
                       frame=tb]
  add$^o$ x$_1^{f \to ?}$ y$^{g \to g}$ z$_1^{f \to ?}$ /\
  x$^{g \to g}$ ===     S x$_1^{f \to ?}$ /\
  z$^{f \to g}$ ===     S z$_1^{f \to ?}$
    \end{lstlisting}
  \end{minipage}
\end{figure} \pause
\begin{figure}[!t]
  \centering
  \begin{minipage}{0.57\columnwidth}
    \begin{lstlisting}[frame=tb]
 x$^{\inmode}$ ===  S x$_1^{\outmode}$ /\
 add$^o$ x$_1^{\inmode}$ y$^{\inmode}$ z$_1^{\whatmode}$ /\
 z$^{\outmode}$ ===  S z$_1^{\whatmode}$
    \end{lstlisting}
  \end{minipage}
\end{figure} \pause
\begin{figure}[!t]
  \centering
  \begin{minipage}{\columnwidth}
    \begin{lstlisting}[label={add},
                      %  caption={Addition relation},
                       captionpos=b,
                       frame=tb]
  (((x, g->g) === S (x', f->g) /\
    add$^o$ (x', f->g) (y, g->g) (z', f->g) /\
    (z,f->g) === S (z', f->g)))
    \end{lstlisting}
  \end{minipage}
\end{figure}
\end{frame}

\begin{frame}[fragile]
  \frametitle{Order in Conjunctions}
  \begin{figure}[h]
  \centering
  \begin{minipage}{0.87\columnwidth}
    \begin{lstlisting}[frame=tb]
 let rec mult$^o$ x y z = conde [
   (fresh (x$_1$ r$_1$)
     (x === S x$_1$) /\
     (add$^o$ y r$_1$ z) /\
     (mult$^o$ x$_1$ y r$_1$));
  ...]
    \end{lstlisting}
  \end{minipage}
\end{figure}

\end{frame}

\begin{frame}[fragile]
  \frametitle{Order in Conjunctions: Slow Version}
  \begin{figure}[!t]
  \centering
  \begin{minipage}{\columnwidth}
    \begin{lstlisting}[label={mult_slow}, caption={Inefficient implementation of \lstinline{multo in in out} direciton}, captionpos=b, frame=tb]
multXY' :: Nat -> Nat -> Stream Nat
multXY' O y = return O
multXY' x O = return O
multXY' (S O) y = return y
multXY' x (S O) = return x
multXY' (S x') y = do
  (r', r) <- addX y
  multXYZ x' y r'
  return r

multXYZ :: Nat -> Nat -> Nat -> Stream ()
multXYZ O y O = return ()
multXYZ x O O = return ()
multXYZ (S O) y z | y == z = return ()
multXYZ x (S O) z | x == z = return ()
multXYZ (S x') y z = do
  z' <- multXY' x' y
  addXYZ y z' z
multXYZ _ _ _ = Empty
    \end{lstlisting}
  \end{minipage}
\end{figure}

  Premature grounding of \lstinline{z$_1$} leads to generate-and-test behavior
\end{frame}


\begin{frame}[fragile]
  \frametitle{Order in Conjunctions: Faster Version}
  \begin{figure}[!t]
  \centering
  \begin{minipage}{\columnwidth}
    \begin{lstlisting}[label={mult_fast},
                      %  caption={Efficient implementation of \lstinline{multo in in out} direciton},
                       captionpos=b,
                       frame=tb]
multIIO :: Nat -> Nat -> Stream Nat
...
multIIO (S x') y = do
  r' <- multIIO x' y
  addXY y r'
    \end{lstlisting}
  \end{minipage}
\end{figure}

\end{frame}

\begin{frame}[fragile]
  \frametitle{Functional Conversion: Intermediate Language}
\begin{tabular}{llll}
    $\Fun_{V}$ & $=$ &  $\Rtrn \LIST{\Term_{V}}$ & return a tuple of terms\\
               & $\mid$ &  $\Match_{V} \left( \Term_{V}, \Fun_{V} \right)$& match a variable against a pattern\\
               & $\mid$ & $\Bind\LIST{\left(\LIST{V}, \Fun_{V}\right)} $ & monadic bind on streams\\
               & $\mid$ & $\Sum\LIST{\Fun_{V}}$ & concatenation of streams\\
               & $\mid$ & $\Guard\left( V, V \right)$ & equality check\\
               & $\mid$ & $\Gen_{G}$ & generator\\
               & $\mid$ & $R_{i}(\LIST{V}, \LIST{G})$ & function call
\end{tabular}
\end{frame}

\begin{frame}[fragile]
  \frametitle{Functional Conversion into Intermediate Language}
\begin{itemize}
  \item Disjunction $\rightarrow \Sum\LIST{\Fun_{V}}$
  \item Conjunction $\rightarrow \Bind\LIST{\left(\LIST{V}, \Fun_{V}\right)}$
  \item Relation call $\rightarrow R_{i}(\LIST{V}, \LIST{G})$
  \item Unification $\rightarrow$
    \begin{itemize}
      \item $\Rtrn \LIST{\Term_{V}}$
      \item $\Match_{V} \left( \Term_{V}, \Fun_{V} \right)$
      \item $\Guard\left( V, V \right)$
      \item $\Gen_{G}$
    \end{itemize}
\end{itemize}
\end{frame}

\begin{frame}[fragile]
  \frametitle{Functional Conversion into Haskell}
  \begin{itemize}
    \item TemplateHaskell to generate code
    \item Stream monad
    \item do-notation
  \end{itemize}
\end{frame}

\begin{frame}[fragile]
  \frametitle{Functional Conversion into OCaml}
  \begin{itemize}
    \item Hand-crafted (not so) pretty-printer
    \item Stream monad
    \item let*
    \item Taking extra care to employ laziness
  \end{itemize}
\end{frame}

\begin{frame}[fragile]
  \frametitle{Evaluation}
  \begin{center}
    We converted relational interpreters and measured execution time
  \end{center}

  \begin{itemize}
    \item Logic formulas generation
    \begin{itemize}
      \item Inverse computation of an evaluator of logic formulas
      \item Generating formulas which evaluate to \lstinline{true}
    \end{itemize}
    \item Multiplication relation
    \begin{itemize}
      \item Forward direction: multiplication
      \item Backward direction: division
      \item Generation
    \end{itemize}
  \end{itemize}
\end{frame}

\begin{frame}[fragile]
  \frametitle{Generation of Logic Formulas: \lstinline[basicstyle=\Large]{evalo  [true; false; true] q true}}
  \begin{center}
    \includegraphics[height=0.85\textheight]{figures/propIOI.pdf}
  \end{center}
\end{frame}


\begin{frame}[fragile]
  \frametitle{Multiplication: \lstinline[basicstyle=\Large]{mulo n 10 q}}
  \begin{center}
    \includegraphics[height=0.85\textheight]{figures/muloIIO.pdf}
  \end{center}
\end{frame}


\begin{frame}[fragile]
  \frametitle{Division: \lstinline[basicstyle=\Large]{mulo (n/10) q n}}
  \begin{center}
    \includegraphics[height=0.85\textheight]{figures/muloIOI.pdf}
  \end{center}
\end{frame}


\begin{frame}[fragile]
  \frametitle{Multiplication Generation: \lstinline[basicstyle=\Large]{take n (mulo 10 q r)}}
  \begin{center}
    \includegraphics[height=0.85\textheight]{figures/muloIOI.pdf}
  \end{center}
\end{frame}

\begin{frame}[fragile]
  \frametitle{Need for Determinism Check: \lstinline[basicstyle=\Large]{mulo q 10 1000}}
  \begin{center}
    \includegraphics[height=0.85\textheight]{figures/maybe.pdf}
  \end{center}
\end{frame}

\begin{frame}[fragile]
  \frametitle{Need for Determinism Check}
  \begin{itemize}
    \item Just replacing the monad Stream with the monad Maybe improves performance about 10 times for relations on natural numbers
    \begin{itemize}
      \item The implementation stays the same!
    \end{itemize}
    \item Pure (no monad) version is even faster
    \item Use determinism check to figure out when replacing Stream is feasible
    \item How to combine different monads naturally?
  \end{itemize}
\end{frame}

\begin{frame}[fragile]
  \frametitle{Need for Partial deduction}

\begin{center}
\mk can run a verifier backwards to get solver
\end{center}

\begin{center}
\begin{minipage}{0.3\textwidth}
  \lstinline{run q (eval$^o$ q true)}
\end{minipage}

\begin{center}
  Augmenting functional conversion with partial deduction must be beneficial
\end{center}
\end{center}


\end{frame}


\begin{frame}[fragile]
  \frametitle{Conclusion}
Conclusion
  \begin{itemize}
    \item We presented a functional conversion scheme
    \item The conversion speeds up implementations considerably
    \item We implemented the conversion scheme in Haskell
    \item We found some way to order conjuncts
  \end{itemize}

\vfill

We are currently working on
  \begin{itemize}
    \item The integration with partial deduction
    \item The integration into the framework of using relational interpreters for solving
  \end{itemize}
\end{frame}



\end{document}
