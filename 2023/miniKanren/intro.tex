\section{Introduction}
One of the most attractive applications of relational programming is program inversion.
It comes in handy, when the program being inverted is a relational interpreter of some sorts: this way an interpreter for a programming language may be used for program synthesis, a type checker --- to solve type inhabitation problem \todo{(cite relational interpreters for search)}.
Building relational interpreters out of functional implementations can be done automatically \todo{(Cite Lozov's conversion)}, but the resulting relations are often rather slow. 
Expertise and effort are required to manually create a relational interpreter with optimal performance.
Utilizing program transformations to improve performance of the generated relational interpreters may be a better way to achieve better inversions.  

Relational programs do not exist on their own: they are a part of a host program, which utilizes query results in some way. 
Since host languages are rarely logic, they are not expected to be able to process logic variables, nondeterminism and other aspects of relational computations. 
The host program usually only deals with a finite subset of answers, which have been reified into a ground representation, meaning they do not include logic variables. 

When a relation is expected to produce ground answers, and the direction in which it is intended to be run is known, then it becomes possible to convert it into a function which may execute significantly faster than its relational counterpart. 
Performance improvement comes from replacing expensive unifications with considerably faster equality checks, assignments and pattern matches of a host language. 
An informal functional conversion scheme was introduced in the paper \todo{(cite last year's \mk paper)}. 
We are building upon this research effort, presenting a semi-automatic functional conversion algorithm and implementation for a minimal core relational programming language \micro. 
This paper focuses on converting to the target languages of \haskell and \ocaml, although other languages can also be considered as potential target languages.

\todo{(Some evaluation numbers)}