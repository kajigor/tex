\section{Introduction}
One of the most attractive applications of relational programming is \emph{inverse computation}.
It comes in handy, when the program being inverted is a relational interpreter of some sort: this way an interpreter for a programming language may be used for program synthesis, a type checker --- to solve type inhabitation problem and so on~\cite{Untagged, lozov2019relational}.
Constructing relational interpreters out of functional implementations can be done automatically by relational conversion~\cite{lozov2018typed}.
\mk along with relational conversion are capable of inverse computations.
However, it is important to note that inverse computations exhibit lower performance compared to directly executing an inversion of the original program due to the interpretation overhead~\cite{RevURA,SemanticsModifiers1}.
% As an alternative, we propose employing a direction-driven functional conversion for creating inversions instead of relying on an inverter.

Relational programs do not exist on their own: they are a part of a host program, which utilizes query results in some way.
The host languages are not expected to be able to process logic variables, nondeterminism and other aspects of relational computations.
The host program usually only deals with a finite subset of answers, which have been reified into a ground representation, meaning they do not include any logic variables.

When a relation is expected to produce ground answers, and the direction in which it is intended to be run is known, then it becomes possible to convert it into a function which may execute significantly faster than its relational counterpart.
Performance improvement comes from reducing interpretation overhead as well as replacing expensive unifications with considerably faster equality checks, assignments and pattern matches of a host language.
An informal functional conversion scheme was introduced in the paper~\cite{verbitskaia2022direction}.
We are building upon this research effort, presenting a semi-automatic functional conversion algorithm and implementation for a minimal core relational programming language \micro.
This paper focuses on converting to the target languages of \haskell and \ocaml, although other languages can also be considered as potential target languages.
Our evaluation showed performance improvement of $2.5$ times for propositional formulas synthesis and up to $3$ degrees of magnitude improvement for relations over Peano numbers. 