\documentclass[12pt]{article}
\usepackage[left=2cm,right=2cm,top=2cm,bottom=2cm,bindingoffset=0cm]{geometry}
\usepackage{fontspec}
\usepackage{polyglossia}
\setdefaultlanguage{russian}
\setmainfont[Mapping=tex-text]{CMU Serif}

\begin{document}
\begin{center} 
{\LARGE Формальные языки}

{\Large домашнее задание до 23:59 12.11}
\end{center}

\enumerate
{
   \item Для грамматики скобочных последовательностей $ S \rightarrow  ( S ) \, | \, S S  \, | \, \varepsilon$:
    
    \begin{enumerate}
        \item Построить LR(0) автомат и LR(0) таблицу.
        \item Если не удалось, построить SLR(1) таблицу для той же грамматики.
        \item Если не удалось, построить CLR(1) автомат и таблицу для той же грамматики.
        \item Если не удалось, подумать и написать, почему так вышло. 
        \item Если какую-нибудь таблицу построить все-таки удалось, промоделировать с ней разбор строк \verb!()(())! и \verb!(()!: предоставить историю изменения стека и дерево разбора.
    \end{enumerate} 
    \item Для языка арифметических выражений с вычитанием ($-$), делением ($/$) и скобками над алфавитом $\{0, 1\}$ с правильным приоритетом операций и ассоциативностью:
    \begin{enumerate}
        \item Привести грамматику, подходящую для анализа алгоритмом из семейства LR.
        \item Построить LR(0) автомат и LR(0) таблицу.
        \item Если не удалось, построить SLR(1) таблицу для той же грамматики.
        \item Если не удалось, построить CLR(1) автомат и таблицу для той же грамматики.
        \item Промоделировать с ней разбор строк \verb!0-1/(0-1)! и \verb!1//0!: предоставить историю изменения стека и дерево разбора.
    \end{enumerate} 
}

  Автомат можно не рисовать в виде графа: достаточно указать, из каких LR-item-ов состоят состояния, и предоставить таблицу переходов между состояниями. 
\end{document}