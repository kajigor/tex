\documentclass[12pt]{article}
\usepackage{fontspec}
\usepackage{polyglossia}
\usepackage[left=2cm,right=2cm,top=2cm,bottom=2cm,bindingoffset=0cm]{geometry}
\setdefaultlanguage{russian}
\setmainfont[Mapping=tex-text]{CMU Serif}
\usepackage{tikz}
\usetikzlibrary{automata,positioning}
\usepackage{hyperref}


\begin{document}
\begin{center} {\LARGE Формальные языки}

 {\Large домашнее задание до 23:59 11.10} \end{center} 
\bigskip

Репозиторий с парсером: \url{https://github.com/kajigor/hse-parsing}

\enumerate
{
  \item 
  {
    Сделайте так, чтобы парсер игнорировал пробельные символы. Для этого удобно написать версию комбинаторов \verb!>>=! и \verb!|>!, которые пропускают произвольную последовательность пробелов между двумя парсерами.
  }
  
  \item 
  {
    Модифицируйте верхнеуровневую функцию \verb!parse! таким образом, чтобы она не использовала конструкторов \verb!Success! и \verb!Error!. Можно ли это сделать с использованием имеющихся комбинаторов? Если нет, можно добавить новые.  
  }
  
  \item 
  {   
    Модифицируйте текущую реализацию, чтобы допускать произвольные натуральные числа и нормальные идентификаторы (\verb!var = 13 * 42!). Добавьте унарный минус (\verb!var = -1!) и возведение в степень (\verb!2^3!).
    
  }

  \item 
  {
    Модифицируйте язык, дерево и анализатор таким образом, чтобы можно было обрабатывать несколько выражений, выписанных через \verb!;!. Последнее выражение точкой с запятой завершаться не должно
    
    \begin{verbatim} 
x = 13; 
y = z = 42 + 6;
777
    \end{verbatim}
  }

  \item
  {
    Добавьте в язык списки чисел, идентификаторов и выражений. Элементы разделяются запятыми, список взят в квадратные скобки. Над списками существует операция~\verb!++!, их конкатенирующая. Списки можно присваивать в идентификаторы; конкатенировать можно списки и идентификаторы, но не числа и выражения. Списки могут быть произвольной вложенности. Проверять, что все элементы списков одного типа, не надо. 
    
    \begin{verbatim}
a = [] ; 
[b = 13, [z], 42 + 6]  ++ a ++ [31, 25];
777
    \end{verbatim}
  }
}
\end{document}
