\documentclass[12pt]{article}
\usepackage{fontspec}
\usepackage{polyglossia}
\usepackage[left=2cm,right=2cm,top=2cm,bottom=2cm,bindingoffset=0cm]{geometry}
\setdefaultlanguage{russian}
\setmainfont[Mapping=tex-text]{CMU Serif}
\usepackage{tikz}
\usetikzlibrary{automata,positioning}
\usepackage{hyperref}


\begin{document}
\begin{center} {\LARGE Формальные языки}

 {\Large домашнее задание до 23:59 04.10} \end{center} 
\bigskip

Репозиторий с парсером: \url{https://github.com/kajigor/hse-parsing}

\enumerate
{
  \item 
  {   
    Модифицируйте токенизатор, чтобы допускать произвольные натуральные числа и нормальные идентификаторы (\verb!var = 13 * 42!).
    
  }
  \item
  {
    Добавьте в язык унарный минус (\verb!var = -1!). Самый простой способ --- сделать унарный минус наиболее приоритетной операцией на уровне синтаксического анализа. Можно выбрать любой другой способ, но специфицируйте, как именно вы решили это делать.
  }
  \item
  {
    Добавьте в язык возведение в степень (\verb!2^3!). Возведение в степень правоассоциативно.
  }
  \item
  {
    Реализуйте преобразователь нашего дерева с неправильной ассоциативностью для \verb!+-*/! в дерево, где эти операции левоассоциативны.
  }
}
\end{document}
