\documentclass[12pt]{article}
\usepackage{fontspec}
\usepackage{polyglossia}
\usepackage[left=2cm,right=2cm,top=2cm,bottom=2cm,bindingoffset=0cm]{geometry}
\setdefaultlanguage{russian}
\setmainfont[Mapping=tex-text]{CMU Serif}
\usepackage{tikz}
\usetikzlibrary{automata,positioning}


\begin{document}
\centering {\LARGE Формальные языки}

{\Large домашнее задание до 23:59 11.03}
\bigskip

\enumerate
{
  \item 
  {   
     На странной планете живут животные трех видов: $a, b, c$. Размножаются они тоже странно: скрещиваются особи двух разных видов, при этом родительские особи погибают, и на свет появляются 2 особи третьего вида (при скрещивании особей видов $a$ и $b$ появляются 2 особи вида $c$, родители погибают). Если на планете остаются особи только одного вида, дальнейшее размножение оказывается невозможным, что эквивалентно вымиранию. Для $n \in [4..7]$, где $n$ --- начальное количество всех животных на планете, определить стратегии размножения $(a)$ гарантирующие вымирание всех животных, $(b)$ гарантирующие, наоборот, что вымирание не наступит.
  }
  \item
  {
    Перечислить слова языка $L_1 \cap L_2,$ где $L_1 = \{ (ab)^n \mid n \geq 0 \}$ и $L_2 = \{ a^m b^m \mid m \geq 0 \}$. Доказать, что других цепочек в пересечении нет. 
  }
  \item
  {
    Пусть $V_T = \{a,b,c\}$. Равны ли языки $L_1 = \{ (abc)^n a \mid n \geq 2 \}$ и $L_2 = \{ ab (cab)^n ca \mid n \geq 1 \}$? Привести аргументацию точки зрения. 
  }
  \item
  {
    Доказать или опровергнуть: 
    \begin{enumerate}
      \item алгоритм Томпсона детерминизации конечных автоматов работает на неполных автоматах так же, как на дополненных ``стоком'';
      \item алгоритм минимизации конечных автоматов работает на неполных автоматах так же, как на дополненных ``стоком''.
    \end{enumerate}
  }
  \item { Детерминировать и минимизировать следующий автомат: 
  
  \begin{tikzpicture}[shorten >=1pt,node distance=2cm,on grid,auto] 
   \node[state,initial] (q_0)   {$A$}; 
   \node[state] (q_1) [right=of q_0] {$B$}; 
   \node[state] (q_2) [right=of q_1] {$C$}; 
   \node[state,accepting](q_3) [right=of q_2] {$D$};
    \path[->] 
    (q_0) edge  node {a} (q_1)
          edge [loop above] node {a} ()
          edge [loop below] node {b} ()
    (q_1) edge  node  {a,b} (q_2)
    (q_2) edge  node  {a,b} (q_3);
\end{tikzpicture}

  
  }
  \item { Построить минимальный детерминированный конечный автомат для языка идентификаторов без ключевых слов.
  \begin{itemize}
      \item Идентификаторы: слова над алфавитом $\{\_, 0, \dots, 9, a, \dots, z \}$, не начинающиеся с цифры
      \item Ключевые слова: $\{ by, val, var \}$
  \end{itemize}}
}
\end{document}
