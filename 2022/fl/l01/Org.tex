\documentclass{beamer}
\usepackage{beamerthemesplit}
\usepackage{wrapfig}
\usetheme{SPbGU}
\usepackage{pdfpages}
\usepackage{amsmath}
\usepackage{cmap}
\usepackage[T2A]{fontenc}
\usepackage[utf8]{inputenc}
\usepackage[english,russian]{babel}
\usepackage{indentfirst}
\usepackage{amsmath}
\usepackage{tikz}
\usepackage{multirow}
\usepackage[noend]{algpseudocode}
\usepackage{algorithm}
\usepackage{algorithmicx}
\usetikzlibrary{shapes,arrows}
\usepackage{fancyvrb}
\newtheorem{rutheorem}{Теорема}
\newtheorem{ruproof}{Доказательство}
\newtheorem{rudefinition}{Определение}
\newtheorem{rulemma}{Лемма}
\setbeamertemplate{itemize item}[circle]
\beamertemplatenavigationsymbolsempty

\title[]{Теория автоматов и формальных языков}
\subtitle[]{Организационные моменты}
\institute[]{
НИУ-ВШЭ\\
}

\author[]{Екатерина Вербицкая}

\date{5 сентября 2022}

\definecolor{orange}{RGB}{179,36,31}

\begin{document}
{
\begin{frame}
  \titlepage
\end{frame}

}

\begin{frame}[fragile]
  \transwipe[direction=90]
  \frametitle{Контакты}
  \begin{itemize}
    \item Екатерина [Андреевна] Вербицкая
    \item JetBrains Programming Languages and Tools Lab
    \item email: \href{mailto:kajigor@gmail.com}{kajigor@gmail.com}
    \item email: \href{mailto:ekaterina.verbitskaya@jetbrains.com}{ekaterina.verbitskaya@jetbrains.com}
    \item Телеграм: kajigor
    \item Материалы: \url{https://bit.ly/3CUQXuw}
  \end{itemize}
\end{frame}

\begin{frame}[fragile]
  \transwipe[direction=90]
  \frametitle{Занятия}
  \begin{itemize}
    \item Занятия по понедельникам и четвергам
    \item Разделение на лекции-практики по обстоятельствам
    \item Домашние задания выполняются дома, сдаются онлайн
    \begin{itemize}
      \item Сроки сдачи необходимо строго соблюдать
      \item Типичный срок сдачи: до 23:59 воскресенья перед занятием
    \end{itemize}
    \item Проект
    \item Экзамен
  \end{itemize}
\end{frame}

\begin{frame}[fragile]
  \transwipe[direction=90]
  \frametitle{Условия успешной сдачи}
  \begin{itemize}
    \item Домашние задания не обязательные, но сложность проекта зависит от процента сданных домашек
    \item Проект дает автомат по курсу
  \end{itemize}
  \begin{itemize}
    \item Оценку можно будет улучшить на очном экзамене
  \end{itemize}
\end{frame}

\begin{frame}[fragile]
  \transwipe[direction=90]
  \frametitle{Литература и материалы}
  \begin{itemize}
    \item Dick Grune and Ceriel J.H. Jacobs, Parsing Techniques - A Practical Guide, Second Edition (2008): \url{https://bit.ly/2lxP4y2}
    \item Хопкрофт Дж. Э., Мотвани Р., Ульман Дж. Д. Введение в теорию автоматов, языков и вычислений, 2-е изд./Пер. с англ. – М.: Вильямс, 2002.
    \item Ахо А., Ульман Дж. Теория синтаксического анализа, перевода и компиляции. Т.1/ Пер. с англ. – М.: Мир, 1978
    \item Вики-конспект ИТМО по аналогичному курсу: \url{http://goo.gl/we8kvB}
  \end{itemize}
\end{frame}


\end{document}
