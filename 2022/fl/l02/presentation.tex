\documentclass{beamer}
\usepackage{beamerthemesplit}
\usepackage{wrapfig}
\usetheme{SPbGU}
\usepackage{pdfpages}
\usepackage{amsmath}
\usepackage{mathtools}
\usepackage{cmap}
\usepackage[T2A]{fontenc}
\usepackage[utf8]{inputenc}
\usepackage[english,russian]{babel}
\usepackage{indentfirst}
\usepackage{amsmath}
\usepackage{tikz}
\usepackage{multirow}
\usepackage{multicol}
\usepackage{array}
\usepackage[noend]{algpseudocode}
\usepackage{algorithm}
\usepackage{algorithmicx}
\usepackage{minted}
\usepackage{verbatim}
\usepackage[linguistics]{forest}
\usetikzlibrary{shapes,arrows}
\usepackage{fancyvrb}
\newtheorem{rutheorem}{Теорема}
\newtheorem{ruproof}{Доказательство}
\newtheorem{rudefinition}{Определение}
\newtheorem{rulemma}{Лемма}
\beamertemplatenavigationsymbolsempty
\setbeamertemplate{itemize item}[circle]
\setbeamertemplate{enumerate item}[circle]
\newcommand{\derives}[1][*]{\xRightarrow[]{#1}}


\title[]{Теория автоматов и формальных языков}
\subtitle[]{Формальные грамматики}
\institute[]{
НИУ-ВШЭ\\
}

\author[]{Екатерина Вербицкая}

\date{5 сентября 2022}

\definecolor{orange}{RGB}{179,36,31}

\begin{document}
{
  \begin{frame}
    \titlepage
  \end{frame}
}

\begin{frame}[fragile]
  \transwipe[direction=90]
  \frametitle{В предыдущей серии}
  \begin{itemize}
    \item Формальные языки повсюду. Язык --- множество строк над алфавитом
    \item Существует множество способов описать язык
    \item Задачи теории формальных языков
    \begin{itemize}
      \item Как представить язык?
      \item Какие есть характеристики у разных представлений языка?
      \item Как определить, принадлежит ли строка данному языку?
    \end{itemize}
  \end{itemize}
\end{frame}

\begin{frame}[fragile]
  \transwipe[direction=90]
  \frametitle{Метаязык}
  \begin{itemize}
    \item Язык, на котором дано описание языка
    \begin{itemize}
      \item Естественный язык
      \item Язык металингвистических формул Бэкуса (БНФ)
      \item Синтаксические диаграммы
      \item \textbf{Грамматики}
      \item $\dots$
    \end{itemize}
  \end{itemize}
\end{frame}

\begin{frame}[fragile]
  \transwipe[direction=90]
  \frametitle{Описание языка: формальная грамматика}
  \begin{itemize}
    \item \textbf{Порождающая грамматика $G$} --- это четверка $\langle V_T, V_N, P, S \rangle$

    \begin{itemize}
      \item $V_T$ --- алфавит терминальных  символов (терминалов)
      \item $V_N$ --- алфавит нетерминальных  символов (нетерминалов)
      \begin{itemize}
        \item $V_T \cap V_N = \emptyset$
        \item $V ::= V_T \cup V_N$
      \end{itemize}
      \item P --- конечное множество правил вида $\alpha \rightarrow \beta$
      \begin{itemize}
        \item $\alpha \in V^* V_N V^*$
        \item $\beta \in V^*$
      \end{itemize}
      \item S --- начальный нетерминал грамматики,\begin{itemize}
        \item $S  \in N$
      \end{itemize}
    \end{itemize}
  \end{itemize}
\end{frame}

\begin{frame}[fragile]
  \transwipe[direction=90]
  \frametitle{Пример: язык чисел в двоичной системе счисления}

\begin{center}
  $V_T = \{ 0, 1, - \} \ \ \  V_N = \{ S, N, A \}$
\end{center}

\begin{tabular}{p{3cm} | p{4cm} | p{3cm}}

\[
\begin{array}{rcl}
S& \rightarrow & 0 \\
S& \rightarrow & N \\
S& \rightarrow & - N \\
N& \rightarrow & 1 A \\
A& \rightarrow & 0 A \\
A& \rightarrow & 1 A \\
A& \rightarrow & \varepsilon \\
\end{array}
\]

& \pause

\[
\begin{array}{rcl}
S& \rightarrow & 0 \mid N \mid - N  \\
N& \rightarrow & 1 A \\
A& \rightarrow & 0 A \mid 1 A  \mid \varepsilon\\
\end{array}
\]

& \pause

\[
\begin{array}{rcl}
S& \rightarrow & 0 \mid [-] N  \\
N& \rightarrow & 1 A \\
A& \rightarrow & (0 \mid 1) A  \mid \varepsilon\\
\end{array}
\]

\end{tabular}
\end{frame}

\begin{frame}[fragile]
  \transwipe[direction=90]
  \frametitle{Отношение непосредственной выводимости}
  \begin{itemize}
    \item $\alpha \rightarrow \beta \in P$
    \item $\gamma, \delta \in V^*$
    \item $\gamma \alpha \delta \Rightarrow \gamma \beta \delta$: $\gamma \beta \delta$ \textbf{непосредственно выводится} из $\gamma \alpha \delta$ при помощи правила $\alpha \rightarrow \beta$
  \end{itemize}
\end{frame}

\begin{frame}[fragile]
  \transwipe[direction=90]
  \frametitle{Отношение непосредственной выводимости: пример}
  \[
  \begin{array}{rcl}
  S& \rightarrow & 0 \mid N \mid - N  \\
  N& \rightarrow & 1 A \\
  A& \rightarrow & 0 A \mid 1 A  \mid \varepsilon\\
  \end{array}
  \]

\vspace{10pt}

\[S \derives[] -N\]

\[-N \derives[] -1A\]

\[-1A \derives[] -11A\]
\end{frame}


\begin{frame}[fragile]
  \transwipe[direction=90]
  \frametitle{Отношение выводимости}

  \begin{center}
    \textbf{Отношение выводимости} является рефлексивно-транзитивным замыканием отношения непосредственной выводимости
  \end{center}

  \begin{itemize}
    \item $\alpha_0, \alpha_1, \alpha_2, \dots, \alpha_n \in V^*$
    \item $\alpha_0 \Rightarrow \alpha_1 \Rightarrow \alpha_2 \Rightarrow \dots \Rightarrow \alpha_n$
    \item $\alpha_0 \derives \alpha_n$: $\alpha_n$ \textbf{выводится} из $\alpha_0$
  \end{itemize}
\end{frame}


\begin{frame}[fragile]
  \transwipe[direction=90]
  \frametitle{Отношение выводимости: пример}

  \[
  \begin{array}{rcl}
  S& \rightarrow & 0 \mid N \mid - N  \\
  N& \rightarrow & 1 A \\
  A& \rightarrow & 0 A \mid 1 A  \mid \varepsilon\\
  \end{array}
  \]

  \[ S \Rightarrow - N \Rightarrow - 1 A \Rightarrow - 1 1 A \derives - 1 1 0 1 A \Rightarrow - 1 1 0 1 \]
\end{frame}

\begin{frame}[fragile]
  \transwipe[direction=90]
  \frametitle{Отношение выводимости: свойства}

  \begin{itemize}
    \item Транзитивность: $\forall \alpha, \beta, \gamma \in V^*: \ \alpha \derives \beta, \beta \derives \gamma \text{ следовательно } \alpha \derives \gamma$
    \item Рефлексивность: $\forall \alpha \in V^*: \ \alpha \derives \alpha$
  \end{itemize}

  \begin{itemize}
    \item $\alpha_0 \derives[+] \alpha_n$: вывод использует хотя бы одно правило грамматики
    \item $\alpha_0 \derives[k] \alpha_n$: вывод происходит за $k$ шагов
  \end{itemize}
\end{frame}


\begin{frame}[fragile]
  \transwipe[direction=90]
  \frametitle{Левосторонний вывод}

\begin{center}
  На каждом шагу заменяем самый \textbf{левый} нетерминал
\end{center}

\[
  \begin{array}{rcl}
  S& \rightarrow & A A \mid s  \\
  A& \rightarrow & A A \mid B b \mid a \\
  B& \rightarrow & c \mid d
  \end{array}
\]

\[ \boldsymbol{S} \derives[] \boldsymbol{A} A \derives[] \boldsymbol{B} b A \derives[] c b \boldsymbol{A} \derives[] c b \boldsymbol{A} A \derives[] c b a \boldsymbol{A} \derives[] c b a a \]

\begin{center}
  Аналогично определяется \textbf{правосторонний} вывод
\end{center}

\end{frame}


\begin{frame}[fragile]
  \transwipe[direction=90]
  \frametitle{Язык, порождаемый грамматикой $G = \langle V_T, V_N, P, S \rangle$}
\[ L(G) = \{ \omega \in V_T^* \mid S \derives \omega \} \]
\end{frame}

\begin{frame}[fragile]
  \transwipe[direction=90]
  \frametitle{Эквивалентность грамматик}
\begin{center}
  Грамматики $G_1$ и $G_2$ \textbf{эквивалентны}, если $L(G_1) = L(G_2)$
\end{center}

\pause

\begin{tabular}{p{0.5\textwidth} | p{0.5\textwidth}}

\[
  \begin{array}{rcl}
  V_T &=& \{ 0, 1, - \} \\
  V_N &=& \{ S, N, A \} \\~\\
  S& \rightarrow & 0 \mid N \mid - N  \\
  N& \rightarrow & 1 A \\
  A& \rightarrow & 0 A \mid 1 A  \mid \varepsilon\\
  \end{array}
\]

&

\[
  \begin{array}{rcl}
  V_T &=& \{ 0, 1, - \} \\
  V_N &=& \{ S, A \} \\~\\
  S& \rightarrow & 0 \mid 1 A  \mid - 1 A  \\
  A& \rightarrow &  0 A \mid 1 A  \mid \varepsilon\\
  \end{array}
\]
\end{tabular}

\end{frame}

\begin{frame}[fragile]
  \transwipe[direction=90]
  \frametitle{Контекстно-свободная грамматика}
    \begin{center}
      \textbf{Контекстно-свободная грамматика} --- грамматика, все правила которой имеют вид $A \rightarrow \alpha, A \in V_N, \alpha \in V^*$
    \end{center}
\end{frame}

\begin{frame}[fragile]
  \transwipe[direction=90]
  \frametitle{Дерево вывода}

  Дерево является \textbf{деревом вывода} для $G = \langle V_N, V_T, P, S\rangle$, если:
  \begin{itemize}
    \item Каждый узел помечен символом из алфавита $V$
    \item Метка корня --- $S$
    \item Листья помечены терминалами, остальные узлы --- нетерминалами
    \item Если узлы $n_0, \dots, n_k$ --- прямые потомки узла $n$, перечисленные слева направо, с метками $A_0, \dots, A_k$; метка $n$ --- $A$, то $A \rightarrow A_0 \dots A_k \in P$
  \end{itemize}
\end{frame}

\begin{frame}[fragile]
  \transwipe[direction=90]
  \frametitle{Пример дерева вывода}
\begin{center}

\[ G = \langle \{ S, A \}, \{ a, b \}, \{ S \rightarrow a A S \mid a, A \rightarrow S b A \mid b a \mid SS \}, S\rangle \]

\[ S \Rightarrow aAS \Rightarrow a S b A S \Rightarrow a a b A S \Rightarrow a a b b a S \Rightarrow a a b b a a \]

\begin{forest}
  [S
    [a]
    [A
      [S
        [a]
      ]
      [b]
      [A
        [b]
        [a]
      ]
    ]
    [S
      [a]
    ]
  ]
\end{forest}
\end{center}
\end{frame}

\begin{frame}[fragile]
  \transwipe[direction=90]
  \frametitle{Вывод и дерево вывода}
  \begin{rutheorem}[]
    Пусть $G = \langle V_N, V_T, P, S \rangle$ --- КС-грамматика

    Вывод $S \derives \alpha$, где $\alpha \in V^*, \alpha \neq \varepsilon$ существует $\Leftrightarrow$ существует дерево вывода в грамматике G с результатом $\alpha$
  \end{rutheorem}

\begin{center}
  Упражнение: доказать теорему
\end{center}
\end{frame}

\end{document}
