\section{Introduction}

The \kan family of embedded programming languages is designed to bring the power of logic programming into any host language \todo{cite minikanren.org}.
The smallest core language, \muk, is comprised of as little as 4 combinators and can be implemented in 40 lines of code \todo{cite}.
Using this minimal core, one can implement \emph{relations}: programs which describe how values relate to each other in mathematical sense.
\kan languages are intended to be easily modifiable, including extensions for nominal and constraint logic programming \todo{cite, cite}.

Consider an addition relation \lstinline{add$^o$ x y z} which specifies that \z equals to \lstinline{x + y} (Listing~\ref{add}).
Having a relation, one can \emph{run} it in some \emph{direction} to compute useful data.
Running \lstinline{add$^o$} with the specific known \x and \y produces their sum as a result, implementing addition.
It is also possible to run the same relation \lstinline{add$^o$} and pass only \z and this produces all pairs of numbers which sum up to the given \z.

\begin{figure}[!t]
  \centering
  \begin{minipage}{0.7\columnwidth}
    \begin{lstlisting}[frame=tb]
 let rec add$^o$ x y z = conde [
   (x === O /\ y === z);
   (fresh (x$_1$ z$_1$)
     (x === S x$_1$ /\
      add$^o$ x$_1$ y z$_1$ /\
      z === S z$_1$) ) ]
    \end{lstlisting}
  \end{minipage}
\end{figure}

Relational programming is a step towards truly declarative programming, in which user only states \emph{what} is to be done and not \emph{how}.
Being an attractive technique on its own, it also enables \todo{list applications}.

Automatic translation from a general purpose programming \todo{cite Lozov, unnesting} makes it possible to create relational specifications which then may be run in a direction of choice and thus do more than original program.
As an example, one may implement a simple functional verifier which checks that some candidate is indeed a solution for a search problem.
When translated into \mk, this verifier may be used to actually solve search problems with no deep knowledge required from a programmer. \todo{cite rel.interpreters}.

Unfortunately, \mk comes with a lot of overhead, which may hinder using it in practical setting.
One source of overhead comes from \emph{unification} --- the basic operation which is at the core of all \kan languages.
Unification involves traversing terms being unified along with a list of substitutions and doing occurs-check all of which may be redundant when there is a specific execution direction in mind.
The other source of overhead comes from the planning complexity \todo{cite Rozplohos}.
\todo{We do not do anything about this, do we?}

Directions fix at compile-time which arguments of a relation are always going to be known and ground at runtime.
Having this information, it is possible to specialize a relation for the direction \todo{cite Verbitskaia} and reduce some of the overhead.
Moreover, unifications may prove to be redundant and be replaced with much simpler pattern-matching and equality checks.

In this paper we present a scheme of translation of \mk programs into a host functional programming language as a sequence of examples.
Examples start from simplest translations and evolve to introduce different features of \mk which influence translation.
Currently translation is not automated: everything is done manually.
We believe the translation may be semi-automated: some decisions will still be left up to a programmer.
Although this project is at the early state, evaluation demonstrates its usefulness by significantly speeding up such programs as computing a topological sorting of a graph and generating logic formulas which evaluate to the given value.



