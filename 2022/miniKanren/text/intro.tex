\section{Introduction}

It is well-known that some programs are easier to implement as inversions of other, simpler programs~\cite{Abramov2002}. One of the notable
cases is \emph{verifiers} vs. \emph{solvers}~\cite{lozov2019relational}: it is rather easy to implement a verification procedure which tests
if a given candidate is indeed a solution of a certain problem, and the inversion of this procedure delivers a solver. 
There are a few aproaches to program inversion, for example, universal resolving algorithm~\cite{abramov2002principles} and logic and
relational programming. In latter case, inversion comes with a lot of overhead which may be eliminated.

One source of overhead in relational programming comes from \emph{unification}~--- the basic operation which is at the core of \mk.
Unification involves traversing terms being unified along with a list of substitutions and doing occurs-check all of which may be redundant when there is a specific execution \emph{direction} in mind.
Directions fix at compile-time which arguments of a relation are always going to be known and ground at runtime.
Having this information, it is possible to specialize a relation for the direction~\cite{EPTCS341.5} and get rid of some of the overhead.
In this case, unifications may prove to be redundant and be replaced with much simpler pattern-matching and equality checks.

In this paper we study a conversion of \mk programs into a host functional programming language in a sequence of examples.
Examples start from the simplest conversions and evolve to introduce different features of \mk which influence conversion.
Currently the conversion is not automated: everything is done manually. We believe the conversion can be semi-automated, leaving
some decisions up to a programmer. Although this project is at the early state, the evaluation demonstrates its usefulness by
significantly speeding up such programs as computing a topological sorting of a graph and generating logic formulas which evaluate
to a given value.

