\begin{abstract}
  Relational programming is known for its capability to provide a short and concise executable specifications
  for a wide range of interesting problems. Specifically, the nature of relational programming makes it possible
  to consider a single specification as a whole family of concrete programs. Individual programs of this family
  can be taken and run by placing free variables inside a top-level goal arguments. In particular, relational
  programming provides a very generic way to implement \emph{program inversion}, which opens a way for
  program synthesis via converting \emph{verifiers} into \emph{solvers}. However, acquired in such a way
  solvers often come with an overhead, originating from the very nature of relational computations
  with substitutions, unifications, interleaving, etc. In this paper we study a conversion of relational programs
  into functional form taking into account a concrete \emph{direction} of evaluation. The project is at an early stage,
  but the results so far are promising: converted functions run much faster than the original relations.
\end{abstract}
