\section{Related Work}

There are several research area relevant to our conversion.
First, semantic modifiers~\cite{abramov2001standard} and universal resolving algorithm~\cite{abramov2002principles} may be used to invert computations.
They do not guarantee termination in general, which is reasonable, given that the problem is undecidable.

Logic and relational programming languages inherently support inverse computations, but they often come with significant overhead.
Reducing such overhead may be done with such techniques as partial evaluation, or partial deduction.
Applying these techniques to \mk has not yet done successfully, although some speed ups were achieved~\cite{EPTCS341.5}.

Functional logic programming languages such as \curry and \mercury translate their logic subsets into a general programming language.
\mercury also comes with a sophisticated system of modes along with mode analysis which we would adapt to \mk as part of future work.
The search strategy in \mercury is not complete which limits its use.

\curry has several compilers including the one whose target language is \haskell~\cite{brassel2011kics2}.
\curry also comes with flexibility in choosing the search strategy~\cite{hanus2012search}.

There exist an automatic conversion from a subset of \ocaml into \ocanren~\cite{lozov2017typed}.
Coupling it with conversion from \mk back into \ocaml can be used for generating solvers from verifies.


