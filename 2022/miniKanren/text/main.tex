\documentclass[sigplan,screen]{acmart}

\usepackage{xspace}
\usepackage{listings}
\newcommand{\todo}[1]{{\color{red} #1}}


%% Rights management information.  This information is sent to you
%% when you complete the rights form.  These commands have SAMPLE
%% values in them; it is your responsibility as an author to replace
%% the commands and values with those provided to you when you
%% complete the rights form.
\setcopyright{acmcopyright}
\copyrightyear{2018}
\acmYear{2018}
\acmDOI{XXXXXXX.XXXXXXX}

%% These commands are for a PROCEEDINGS abstract or paper.
\acmConference[Conference acronym 'XX]{Make sure to enter the correct
  conference title from your rights confirmation emai}{June 03--05,
  2018}{Woodstock, NY}
\acmPrice{15.00}
\acmISBN{978-1-4503-XXXX-X/18/06}

%%
%% end of the preamble, start of the body of the document source.
\begin{document}

% TODO: better title needed
\title{\todo{Think of a Title}}

%%
%% The "author" command and its associated commands are used to define
%% the authors and their affiliations.
%% Of note is the shared affiliation of the first two authors, and the
%% "authornote" and "authornotemark" commands
%% used to denote shared contribution to the research.
\author{Ekaterina Verbitskaia}
\email{kajigor@gmail.com}
\orcid{0000-0002-6828-3698}
\affiliation{%
  \institution{JetBrains}
  \streetaddress{Smth Smth}
  \city{Belgrade}
  \country{Serbia}
  \postcode{11000}
}

\author{Daniil Berezun}
\affiliation{
  \institution{JetBrains}
  \streetaddress{Smth smth}
  \city{Amsterdam}
  \country{Netherlands}}
\email{example@example.com}

\author{Dmitry Boulytchev}
\affiliation{%
  \institution{SPbSU, Huawei}
  \city{Saint Petersburg}
  \country{Russia}
}

% \renewcommand{\shortauthors}{Verbitskaia et al.}

\begin{abstract}
Many programs which solve complicated problems can be seen as inversions of other, much simpler, programs.
One particular example is transforming verifiers into solvers which can be achieved with low effort by implementing the verifier in a relational language and then executing it in the backward direction.
Unfortunately, as it is common with inverse computations, interpretation overhead may lead to subpar performance compared to direct program inversion.
In this paper we discuss functional conversion aimed at improving relational \mk specifications with respect to a known fixed direction.
Our preliminary evaluation demonstrates a significant performance increase for some programs which exemplify the approach.
\end{abstract}

% %% TODO: WHAT IS THIS
% %% The code below is generated by the tool at http://dl.acm.org/ccs.cfm.
% %% Please copy and paste the code instead of the example below.
% %%
% \begin{CCSXML}
% <ccs2012>
%  <concept>
%   <concept_id>10010520.10010553.10010562</concept_id>
%   <concept_desc>Computer systems organization~Embedded systems</concept_desc>
%   <concept_significance>500</concept_significance>
%  </concept>
%  <concept>
%   <concept_id>10010520.10010575.10010755</concept_id>
%   <concept_desc>Computer systems organization~Redundancy</concept_desc>
%   <concept_significance>300</concept_significance>
%  </concept>
%  <concept>
%   <concept_id>10010520.10010553.10010554</concept_id>
%   <concept_desc>Computer systems organization~Robotics</concept_desc>
%   <concept_significance>100</concept_significance>
%  </concept>
%  <concept>
%   <concept_id>10003033.10003083.10003095</concept_id>
%   <concept_desc>Networks~Network reliability</concept_desc>
%   <concept_significance>100</concept_significance>
%  </concept>
% </ccs2012>
% \end{CCSXML}

% \ccsdesc[500]{Computer systems organization~Embedded systems}
% \ccsdesc[300]{Computer systems organization~Redundancy}
% \ccsdesc{Computer systems organization~Robotics}
% \ccsdesc[100]{Networks~Network reliability}

\keywords{relational programming, functional programming, cps }

\maketitle


\section{Introduction}

Implementing a program is often significantly easier than its inversion.
For example, integer multiplication is much simpler than factoring, while program evaluation is easier than program generation.
Although inversion is undecidable, there are approaches capable of inversing a computation in some cases, notably, universal resolving algorithm \todo{cite Gluck}, logic and relational programming.
Inversion comes with a lot of overhead which may be reduced in some circumstances.
% The focus of this paper is to reduce overhead when implementing inversions of  programs written in \mk by translating them into functional counterparts.

One source of overhead in relational programming comes from \emph{unification} --- the basic operation which is at the core of \mk.
Unification involves traversing terms being unified along with a list of substitutions and doing occurs-check all of which may be redundant when there is a specific execution \emph{direction} in mind.
Directions fix at compile-time which arguments of a relation are always going to be known and ground at runtime.
Having this information, it is possible to specialize a relation for the direction \todo{cite Verbitskaia} and get rid of some of the overhead.
In this case, unifications may prove to be redundant and be replaced with much simpler pattern-matching and equality checks.

In this paper we present a scheme of translation of \mk programs into a host functional programming language as a sequence of examples.
Examples start from the simplest translations and evolve to introduce different features of \mk which influence translation.
Currently translation is not automated: everything is done manually.
We believe the translation can be semi-automated, leaving some decisions up to a programmer.
Although this project is at the early state, evaluation demonstrates its usefulness by significantly speeding up such programs as computing a topological sorting of a graph and generating logic formulas which evaluate to the given value.


\section{Preliminaries}

Consider an addition relation \lstinline{add$^o$ x y z} which specifies that \z equals to \lstinline{x + y} (Listing~\ref{add}).
Having this relation, one can \emph{run} it in some \emph{direction} to compute useful data.
Running \lstinline{add$^o$} with the specific known \x and \y produces their sum as a result, implementing addition.
It is also possible to run the same relation \lstinline{add$^o$} and pass only \z and this produces all pairs of numbers which sum up to the given \z.
\begin{figure}[!t]
  \centering
  \begin{minipage}{0.7\columnwidth}
    \begin{lstlisting}[frame=tb]
 let rec add$^o$ x y z = conde [
   (x === O /\ y === z);
   (fresh (x$_1$ z$_1$)
     (x === S x$_1$ /\
      add$^o$ x$_1$ y z$_1$ /\
      z === S z$_1$) ) ]
    \end{lstlisting}
  \end{minipage}
\end{figure}
% \section{Modes}


Any relation comes with a multitude of \emph{modes}, or directions.
The mode, among other things, specifies which arguments of the relation are input, and which values are to be computed.
In some systems, such as \mercury, modes also specify the determinism of a relation: whether or not it is supposed to compute one or many values.
There are many other different types of determinism, one can \todo{read more about this here}.

\begin{figure}[!t]
  \centering
  \begin{minipage}{0.28\textwidth}
    \begin{lstlisting}[label={add}, caption={Addition relation}, captionpos=b, frame=tb]
let rec add$^o$ x y z = conde [
  (x === zero /\ y === z);
  (fresh (x' z')
    (x === succ x' /\
     z === succ z' /\
     add$^o$ x' y z') ) ]
    \end{lstlisting}
  \end{minipage}
\end{figure}

A relation with a specified mode can be viewed as a function which maps its bound variables into its free variables.
Consider a relation \lstinline{add$^o$ x y z} in Listing~\ref{add} with the mode \lstinline{(in, in, out) is det}.
This mode means that the arguments \x and \y are known (are input), the value of \z is computed, and there must be exactly one value of \z for every distinct pair of values of \x and \y.
This mode corresponds to the addition.
The mode \lstinline{(out, out, in) is nondet} means that the value of \z is known, the values of \x and \y are computed.
We use \lstinline{nondet} here, since there may be a multitude of possible pairs of \x and \y which sum up to a given value \z.
Notice that at least one such pair exist for any value of \z, but we do not explicitly specify it.

Modes cannot be computed exactly, only over-approximated \todo{find a citation for some Mercury paper}.
Mercury allows to only specify modes for top-level predicates and functions, and the uses abstract interpretation to inference other modes.
It is not unusual to have several modes for a single predicate.
A predicate with a given mode may directly or indirectly use the same predicate with different modes.
In this case, several functions are generated for each mode of the predicate.

% \section{Translation by Examples}

In this section we present a scheme for translation of a \mk relation with a given mode into pure functions.
We start by exploring examples which are translated straightforwardly, and then consider aspects of relational programming which complicate translation considerably.
We will then describe how translation scheme must be adjusted to incorporate these complicated features.

\subsection{The Simplest Case}

For a relation \lstinline{add$^o$ x y z} with the mode \lstinline{(in, in, out) is det} we can easily construct a pure function which has the same semantics: see Listing~\ref{addXY}.
We construct this function using the following thinking.
The relation \lstinline{add$^o$} is comprised of a single disjunction (\conde).
Each disjunct involves a unification of a known (\lstinline{in}) variable \x.
These unifications of \x are disjunct: there may only be one succeeding disjunct in the \conde when computed with a particular ground value of \x.
This naturally translates into a pattern matching on \x with two possible branches: either \x is \lstinline{zero} or a \lstinline{succ} of some other value.
The bodies of both branches are then generated using the remaining conjuncts in the conde branches.
The unifications of \z rule how the result can be constructed when other variables are known.
The body of the first branch of the pattern match is thus just \y.
The second branch of the disjunction involves a recursive call to the \lstinline{add$^o$} relation and a unification of \z.
A recursive call to \lstinline{add$^o$} relation is done in the mode \lstinline{(in, in, out)} since \y is known and \lstinline{x'} is a unification with a known variable.
This means that here we can do a recursive call to the function \lstinline{add_x_y}.
The only thing left is constructing a resulting value which corresponds to \z by applying \lstinline{succ} to the result of the recursive call.

\begin{figure*}[!t]
  \centering
  \begin{minipage}{0.68\textwidth}
    \begin{lstlisting}[label={addXY}, caption={Functional representation of \lstinline{addo x y z} with the mode \lstinline{(in, in, out) is det}}, captionpos=b, frame=tb]
let rec add_x_y x y =
  match x with
  | zero -> y
  | succ x' -> succ (add_x_y x' y)
    \end{lstlisting}
  \end{minipage}
\end{figure*}

\subsection{Nondeterminism}

The simplest translation scheme works, but in a very small number of potential modes.
First of all, the mode must be deterministic for that translation scheme to work.
If there are multiple answers, then we have to express nondeterminism somehow.
One natural way is to use the simplest nondeterminism monad: a list, to represent the resulting value.

Consider the same relation \lstinline{add$^o$ x y z}, but with the mode \lstinline{(out, out, in) is nondet}.
This case is about finding all pairs of values which sum up to the given \z.
Querying this relation with the value of \lstinline{z = succ zero} must compute two answers for \x and \y: \lstinline{(zero, succ zero)} and \lstinline{(succ zero, zero)}.
The first answer comes from the first branch of the \conde, when \z is unified with \y, and \x is zero.
The second answer is computed in the second branch of \conde, after a recursive call to \lstinline{add$^o$} is done with the argument equal to \lstinline{zero}.

Notice that the value of \z do not discriminate the two branches of \conde.
We know that \z cannot be \lstinline{zero} in the second branch, but the first branch do not restrict the value of \z at all.
This means that when \z is zero, answers should be generated from both branches of \conde.
One way of implementing this is shown in Listing~\ref{addZ}.
Here we concatenate the results which are generated by the first and the second branches of \conde via a list concatenation operator \lstinline{@}.
The first branch universally provides a single answer \lstinline{(0, z)}, while the other branch does a recursive call to the \lstinline{add_z} function and then applies \lstinline{succ} constructor to \x.

\begin{figure*}[!t]
  \centering
  \begin{minipage}{0.68\textwidth}
    \begin{lstlisting}[label={addZ}, caption={Functional representation of \lstinline{addo x y z} with the mode \lstinline{(out, out, in) is nondet}}, captionpos=b, frame=tb]
let rec add_z z =
  [(0, z)] @
  match z with
  | succ z' -> List.map (fun x y -> (succ x, y)) (add_z (z - 1))
  | _ -> []
    \end{lstlisting}
  \end{minipage}
\end{figure*}

\subsection{Infinitely Many Answers}

Lists serves well as an abstraction to capture nondeterminism.
However, there are infinitely many answers sometimes which may pose a problem in case of eager host languages such as \ocaml.
Consider \lstinline{add$^o$ x y z} with the mode \lstinline{(out, in, out) is nondet}.
Here we compute all values \x and \z that \lstinline{x + y = z} when \y is given.
Although the implementation of this function is straightforward (see Listing~\ref{addY}), it contains infinite recursion which generates infinite number of answers.
When the host language is lazy, one can force only the first $n$ answers, while in an eager host language additional care must be taken.

\begin{figure*}[!t]
  \centering
  \begin{minipage}{0.68\textwidth}
    \begin{lstlisting}[label={addY}, caption={Functional representation of \lstinline{addo x y z} with the mode \lstinline{(out, in, out) is nondet}}, captionpos=b, frame=tb]
let rec add_y y =
  [(0, y)] @  List.map (fun x z -> (succ x, succ z)) (add_y y)
    \end{lstlisting}
  \end{minipage}
\end{figure*}

\todo{Move to a subsection about miniKanren streams.}
The problem of the infinite number of answers gets more complicated when the user expects only a single answer, but computing it involves intermediate infinite list.
In this case there is no way to just force only the first answers, they all have to be considered. \todo{Example}.

\subsection{Generators}

\todo{This all is wrong, should be rewritten with a more suitable example which really involves a generator}
Some relations do not restrict some of the variables at all.
Consider the relation \lstinline{append$^o$ xs ys zs} which concatenates the lists \lstinline{xs} and \lstinline{ys} to get the result \lstinline{zs} (Listing~\ref{append}).
Whatever the mode of this relation is, it never restricts the value of the fresh variable \lstinline{h}.
Our objective is to implement a function which computes \emph{ground} answers, but since \lstinline{h} is never restricted in any way, there is no way to provide ground elements of lists.
One to solve this problem is by asking the end user to provide a generator of the values which can be used as elements of the list.
Sometimes, this information can be derived from the task or from the type annotations, if the host language is typed.

\begin{figure}[!t]
  \centering
  \begin{minipage}{0.28\textwidth}
    \begin{lstlisting}[label={append}, caption={List concatenation relation}, captionpos=b, frame=tb]
let rec append$^o$ xs ys zs = conde [
  (x === [] /\ y === z);
  (fresh (h t z')
    (x === h :: t /\
     z === h :: z' /\
     append$^o$ t y z') ) ]
    \end{lstlisting}
  \end{minipage}
\end{figure}








\section{Related Work}

Functional logic programming languages such as \merc translate the logic subset of the language into a functional programming language much like we do.
\merc utilizes a strong system of modes and types to guide the compilation\todo{cite mode system paper}.
Many \mk languages are embedded into host languages which are not typed and thus we cannot rely on type information in our conversion.

It is hard to compare our implementation with \merc because of the difference in semantics.
\merc uses a system of prescriptive modes which means that the semantics is defined by the mode assigned to a program.
We use modes in a descriptive fashion, and the semantics of a \mk program is the same set of answers regardless of the order of subgoals.
This allows us to compare runtime of differently modded relational programs.

There are two papers which mention doing mode analysis in the same fashion we do.




\section{Future Work}

Since this project is in active phase of development, there are many directions for future work.

First of all, we need to research how to best order calls within a conjunction.
Since the order of calls greatly influences the efficiency of the converted function, this research direction is of upmost importance.
Annotations of variables with \lstinline{in} and \lstinline{out} are also affected by the order of calls and thus we need to adapt the mode analysis to take it into account.

Second of all, we plan to formalize the conversion and prove its correction.

Third of all, the conversion should be implemented either as a standalone tool or  integrated into some of the major \mk implementations.

Finally, after all these building blocks are done, we would like to integrate the conversion into a relational interpreters framework.
This would made a fullstack solution for the program inversion problem.
\documentclass[crop=false]{standalone}
\usepackage[subpreambles=true]{standalone}
\usepackage{import}
\usepackage{blindtext}
\usepackage[table]{xcolor}


\begin{document}

This concludes the thesis.

\end{document}


\begin{acks}
Here is where acknowledgments come
\end{acks}

\bibliographystyle{ACM-Reference-Format}
\bibliography{main}

\appendix

\section{Principal Directions of the Addition Relation}

\begin{figure}[!t]
  \centering
  \begin{minipage}{\columnwidth}
    \begin{lstlisting}[label={add_all}, caption={Function for \lstinline{addo out out out} direction}, captionpos=b, frame=tb]
add :: Stream (Nat, Nat, Nat)
add =
    disj1 `mplus` disj2
  where
    disj1 = do
      z <- genNat
      return (O, z, z)
    disj2 = do
      (x', y, z') <- add
      return (S x', y, S z')
    \end{lstlisting}
  \end{minipage}
\end{figure}

\begin{figure}[!t]
  \centering
  \begin{minipage}{\columnwidth}
    \begin{lstlisting}[label={add_y}, caption={Function for \lstinline{addo out in out} direction}, captionpos=b, frame=tb]
addY :: Nat -> Stream (Nat, Nat)
addY y =
  return (O, y) `mplus`
  do
    (x', z') <- addY y
    return (S x', S z')
    \end{lstlisting}
  \end{minipage}
\end{figure}

\begin{figure}[!t]
  \centering
  \begin{minipage}{\columnwidth}
    \begin{lstlisting}[label={add_x_z}, caption={Function for \lstinline{addo in out in} direction}, captionpos=b, frame=tb]
addXZ :: Nat -> Nat -> Stream Nat
addXZ x z =
  case x of
    O -> return z
    S x' ->
      case z of
        O -> Empty
        S z' ->
          addXZ x' z'
    \end{lstlisting}
  \end{minipage}
\end{figure}

\begin{figure}[!t]
  \centering
  \begin{minipage}{\columnwidth}
    \begin{lstlisting}[label={add_y_z}, caption={Function for \lstinline{addo out in in} direction}, captionpos=b, frame=tb]
addYZ :: Nat -> Nat -> Stream Nat
addYZ y z =
  if y == z
  then return O
  else
    case z of
      S z' -> do
        x <- addYZ y z'
        return (S x)
      O -> Empty
    \end{lstlisting}
  \end{minipage}
\end{figure}

\end{document}
\endinput
