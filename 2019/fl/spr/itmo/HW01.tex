\documentclass{article}

\usepackage[left=2cm,right=2cm,top=2cm,bottom=2cm,bindingoffset=0cm]{geometry}
\usepackage{listings}
\usepackage{indentfirst}
\usepackage{verbatim}
\usepackage{amsmath, amsthm, amssymb}
\usepackage{stmaryrd}
\usepackage{graphicx}
\usepackage{euscript}
\usepackage{hyperref}

\usepackage[utf8]{inputenc}
\usepackage[english,russian]{babel}
\usepackage[T2A]{fontenc}

\begin{document}

\begin{center} {\LARGE Формальные языки} \end{center}

\begin{center} {\Large домашнее задание до 23:59 13.02} \end{center}
\bigskip

\begin{enumerate}
  \item 
  { 
    Найти описание лексической структуры вашего второго самого любимого языка программирования. Составить регулярные выражения для трех языков: языка идентификаторов, языка ключевых слов (можно какого-нибудь их конечного подмножества) и языка каких-нибудь чисел (целых в десятичной счисления, с плавающей точкой, любых других). В отчете указать, где и какую спецификацию читали и привести регулярные выражения (их должно быть три). 
  }
  \item 
  {
    Реализовать функцию токенизации \verb!tokenize :: String -> [Token]!
    \begin{itemize}
        \item Предполагайте, что все токены разделены пробелами.
        \item Пока в случае ошибки токенизации допустимо ругаться \verb!error!-ом.
        \item Не забудьте написать тесты.
        \item Репозиторий с шаблоном живет тут: \href{https://github.com/kajigor/fl_ifmo_2019_spr/tree/HW01}{https://github.com/kajigor/fl\_ifmo\_2019\_spr/tree/HW01}
        \item Если вы не хотите писать на Haskell, пожалуйста. Тогда: 
        \begin{itemize}
            \item Сделать консольное  приложение, принимающее на вход путь к файлу, содержащему строку, производящее токенизацию и печатающее ее результат в файл.
            \item Код должен быть размещен на гитхабе, собираться одним скриптом, содержать инструкцию по сборке и запуску собранного приложения, собираться на чистой Ubuntu 18.04 или Windows 10. Все зависимости, в случае их отсутствия в системе, должны доставляться скриптом.
            \item Инструкция по запуску должна содержать информацию о том, где находится бинарник, как именно его полагается запускать, какой формат входных данных, куда пишется результат.
        \end{itemize}
    \end{itemize}
  }
\end{enumerate}
\end{document}

