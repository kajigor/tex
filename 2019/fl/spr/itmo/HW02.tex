\documentclass{article}

\usepackage[left=2cm,right=2cm,top=2cm,bottom=2cm,bindingoffset=0cm]{geometry}
\usepackage{listings}
\usepackage{indentfirst}
\usepackage{verbatim}
\usepackage{amsmath, amsthm, amssymb}
\usepackage{stmaryrd}
\usepackage{graphicx}
\usepackage{euscript}
\usepackage{hyperref}

\usepackage[utf8]{inputenc}
\usepackage[english,russian]{babel}
\usepackage[T2A]{fontenc}

\begin{document}

\begin{center} {\LARGE Формальные языки} \end{center}

\begin{center} {\Large домашнее задание до 23:59 20.02} \end{center}
\bigskip

Библиотечка парсер-комбинаторов, которую мы реализовали на паре живет в соответствующей ветке в репозитории: \href{https://github.com/kajigor/fl_ifmo_2019_spr/tree/HW02}{https://github.com/kajigor/fl\_ifmo\_2019\_spr/tree/HW02}

\begin{enumerate}
  \item 
  {
    Реализовать те комбинаторы, которые не реализованы (\verb! = undefined!).
  }
  \item
  {
    Реализовать парсер-комбинатор для ключевых слов, который принимает список ключевых слов и возвращает парсер, который завершается успехом, если префикс входной строки --- одно из ключевых слов. 
    
    \begin{itemize}
        \item Если у нескольких ключевых слов общий префикс, то парсер при работе не должен этот общий префикс разбирать несколько раз.
        \item Если хочется, можно еще и общие суффиксы строк не разбирать несколько раз. 
        \item Имеет смысл сжимать список ключевых слов в бор или минимальный конечный автомат --- на ваш выбор.
    \end{itemize}
  }
  \item
  {
    Переписать токенизацию ключевых слов, чисел и идентификаторов вашего второго самого любимого языка, используя наши парсер-комбинаторы.
  }
\end{enumerate}


    Если вы не хотите писать на Haskell, реализуйте похожую на нашу библиотечку комбинаторов и используйте ее. Правила оформления работы остаются с прошлого домашнего задания.
\end{document}

