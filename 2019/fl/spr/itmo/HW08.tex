\documentclass[12pt]{article}
\usepackage[left=2cm,right=2cm,top=2cm,bottom=2cm,bindingoffset=0cm]{geometry}
\usepackage{fontspec}
\usepackage{polyglossia}
\setdefaultlanguage{russian}
\setmainfont[Mapping=tex-text]{CMU Serif}

\begin{document}
\begin{center}
\LARGE {Формальные языки}

\Large {домашнее задание до 23:59 10.04}
\end{center} 

\bigskip

Шаблон кода для задания живет в соответсвующей ветке: 

\url{https://github.com/kajigor/fl\_ifmo\_2019\_spr/tree/HW08}

\begin{enumerate}
  \item Реализовать универсальный комбинатор для выражений. Принимает список спецификаций для бинарных операций, упорядоченный от низшего приоритета к высшему, и парсер для элементарного выражения. Для каждого уровня приоритета специфицируется ассоциативность --- одинаковая для всех. Каждый бинарный оператор определен парсером для самого значка оператора и семантической функцией, применяемой к аргументам. 
  
  Пример спецификации операторов: 
  
\begin{verbatim}
[ (LAssoc, [ (char '+', BinOp Sum)
           , (char '-', BinOp Minus)      
           ])
, (RAssoc, [ (char '^', BinOp Pow) ])
]    
\end{verbatim}

  
  \item 
  Реализовать парсер арифметических выражений из прошлой домашки, используя универсальный парсер-комбинатор выражения. Результатом является абстрактное синтаксическое дерево (то же, что и в прошлой домашке). Если тип результата парсера не совпадает с тем, что указан в шаблоне, можете его изменить. Не забывайте про тесты.
    
  \item 
  Реализовать калькулятор, используя универсальный парсер-комбинатор выражения. Результатом является целочисленное значение (в случае с логическими значениями, требующими булевы значения, вести себя, как принято в языке C). Если тип результата парсера не совпадает с тем, что указан в шаблоне, можете его изменить. Не забывайте про тесты.\textbf{}
\end{enumerate}

\begin{table}[h]
\centering
\begin{tabular}{l|l|l}
Приоритет & Оператор & Ассоциативность \\ \hline
Высший & \verb!^! & Правоассоциативна \\
       & \verb!*!, \verb!/! & Левоассоциативна  \\
       & \verb!+!, \verb!-! & Левоассоциативна  \\
       & \verb!==!, \verb!/=!, \verb!<=!, \verb!<!, \verb!>=!, \verb!>! & Неассоциативна \\
       & \verb!&&! & Правоассоциативна \\
Низший & \verb!||! & Правоассоциативна 
\end{tabular}
\end{table}
\end{document}
