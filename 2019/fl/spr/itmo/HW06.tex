\documentclass[12pt]{article}
\usepackage[left=2cm,right=2cm,top=2cm,bottom=2cm,bindingoffset=0cm]{geometry}
\usepackage{fontspec}
\usepackage{polyglossia}
\setdefaultlanguage{russian}
\setmainfont[Mapping=tex-text]{CMU Serif}

\begin{document}
\begin{center}
\LARGE {Формальные языки}

\Large {домашнее задание до 23:59 20.03}
\end{center} 

\bigskip

\begin{enumerate}
  \item
  {
    Перечислить слова языка $L_1 \cap L_2,$ где $L_1 = \{ (ab)^n \mid n \geq 0 \}$ и $L_2 = \{ a^m b^m \mid m \geq 0 \}$. Доказать, что других цепочек в пересечении нет. 
  }
  \item 
  { Описать язык $L$, порождаемый грамматикой $<\{0, 1\}, \{S\}, \{S \to 0 1 \, | \, 0 S 1\}, S>$, 
    \begin{itemize}
        \item на естественном языке;
        \item как множество.
    \end{itemize}

  Привести три различных дерева вывода для трех цепочек языка $L$.
  }
  \item 
  {
  Привести контекстно-свободную грамматику для языка арифметических выражений с правильным приоритетом операций и ассоциативностью
  }
\end{enumerate}

\begin{table}[h]
\centering
\begin{tabular}{l|l|l}
Приоритет & Оператор & Ассоциативность \\ \hline
Высший & \verb!^! & Правоассоциативна \\
       & \verb!*!, \verb!/! & Левоассоциативна  \\
       & \verb!+!, \verb!-! & Левоассоциативна  \\
       & \verb!==!, \verb!/=!, \verb!<=!, \verb!<!, \verb!>=!, \verb!>! & Неассоциативна \\
       & \verb!&&! & Правоассоциативна \\
Низший & \verb!||! & Правоассоциативна 
\end{tabular}
\end{table}
\end{document}
