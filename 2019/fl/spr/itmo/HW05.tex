\documentclass{article}

\usepackage[left=2cm,right=2cm,top=2cm,bottom=2cm,bindingoffset=0cm]{geometry}
\usepackage{listings}
\usepackage{indentfirst}
\usepackage{verbatim}
\usepackage{amsmath, amsthm, amssymb}
\usepackage{stmaryrd}
\usepackage{graphicx}
\usepackage{euscript}
\usepackage{hyperref}

\usepackage[utf8]{inputenc}
\usepackage[english,russian]{babel}
\usepackage[T2A]{fontenc}

\begin{document}

\begin{center} {\LARGE Формальные языки} \end{center}

\begin{center} {домашнее задание до 23:59 13.03} \end{center}
\bigskip

Новую ветку создала, но сигнатуры функций прописывать не стала, чтобы дать вам простор для воображения: \href{https://github.com/kajigor/fl_ifmo_2019_spr/tree/HW05}{https://github.com/kajigor/fl\_ifmo\_2019\_spr/tree/HW05}

\begin{enumerate}
  \item
  {
    Реализовать преобразования для автоматов: 
    
    \begin{enumerate} 
      \item Преобразование неполного детерминированного в полный детерминированный. На недетерминированных пусть возвращает ошибку преобразования.
      \item Детерминизация автомата. На детерминированных работает как \verb!id!. Если в автомате есть $\varepsilon$-переходы, пусть возвращает ошибку преобразования.
      \item Эпсилон-замыкание автомата. На автоматах без $\varepsilon$ работает как \verb!id!
      \item Минимизация автомата. Неполные детерминированные должна сначала дополнять и удалять недостижимые состояния. На недетерминированных пусть возвращает ошибку преобразования.
    \end{enumerate}
  }
\end{enumerate}


    Не забывайте писать свои тесты, пожалуйста. 
\end{document}

